\section{Experiments}
\label{sec:experiments}

We use the environments introduced in \citet{plappert2018multi} for our experiments.
Broadly the environments fall in two categories, Fetch and Hand tasks.
Our results show that learning is possible across all environments
without the requirement of goal-reward. More specifically, the learning happens
even when reward given to our algorithm is agent is always ``-1'' as opposed to
the HER formulation where a special goal-reward of ``0'' is needed for learning
to happen.

The Fetch tasks involve a simulation of the Fetch robot's 7-DOF robotic arm. The four tasks are Reach, Push,
Slide and PickAndPlace.
In the Reach task the arm's end-effector is tasked to reach the a particular 3D coordinate. 
In the Push task a block on a table needs to be pushed to a given point on it.
In the Slide task a puck must be slid to a desired location.
In the PickAndPlace task a block on a table must be picked up and moved to a
3D coordinate.

The Hand tasks use a simulation of the Shadow's Dexterous Hand to manipulate objects of
different shapes and sizes. These tasks are HandReach,
HandManipulateBlockRotateXYZ, HandManipulateEggFull and HandManipulatePenRotate.
In HandReach the hand's fingertips need to reach a given configuration.
In the HandManipulateBlockRotateXYZ, the hand needs to rotate a cubic
block to a desired orientation.
In HandManipulateEggFull, the hand repeats this orientation task with an egg, and in
HandManipulatePenRotate, it does so with a pen.

Snapshots of all these tasks can be found in Figure~\ref{fig:envs}. Note that
these tasks use joint angles, not visual input.

\begin{figure}
\scalebox{0.5}{%
\begin{minipage}[t]{0.55\linewidth}
\begin{tikzpicture}
  \def\extrapropsaxisdistepoch{xlabel=Epochs}
  \def\extrapropsaxissuccepoch{xlabel=Epochs},
  \def\extrapropsaxisdistrew{xlabel=Reward Computations},
  \def\extrapropsaxissuccrew{xlabel=Reward Computations},
  \def\fwrlsuffix{fwrl}
  \def\envversion{v1}
  \def\distymin{0}
  \def\scopexshift{0.05\columnwidth}
  \def\scopeyshift{-0.3\columnwidth}
  \begin{scope}[local bounding box=fetchreach]
    %\def\pathprefix{media/res/6efc1de-path_reward_low_thresh_chosen-FetchReach}
\def\distymax{0.25}
  \begin{axis}[ymin=0,xmin=0, ymax=\distymax,xmax=16,
  name=FetchReachDistEpoch,
  xlabel=Epochs,
  width=1.0\columnwidth,
  height=0.70\columnwidth,
  ylabel=Distance from goal (m),
  legend pos=north east]
  \addplot table [x=epoch, y=test/ag_g_dist, col sep=comma] {\pathprefix PR-v1-dqst/progress.csv};
  \addlegendentry{Ours};
  \addplot table [x=epoch, y=test/ag_g_dist, col sep=comma] {\pathprefix -v1-ddpg/progress.csv};
  \addlegendentry{HER};
  \addplot table [x=epoch, y=test/ag_g_dist, col sep=comma] {\pathprefix -v1-fwrl/progress.csv};
  \addlegendentry{FWRL};
  \end{axis}
  \begin{axis}[at={(FetchReachDistEpoch.south east)},
name=FetchReachDistRew,
ymin=0,xmin=0,ymax=\distymax,xmax=5e4,
  xlabel=Reward Computations,
  width=1.0\columnwidth,
  height=0.70\columnwidth,
  ytick=\empty,
  legend pos=north east]
% these constants can be read from \pathprefix */params.json
\def\T{50}
\def\ncycles{10}
\def\rolloutB{2}
\def\batch{256}
  \addplot table [x expr={\thisrow{epoch}*\ncycles*\rolloutB*\T}, y=test/ag_g_dist, col sep=comma] {\pathprefix PR-v1-dqst/progress.csv};
  \addlegendentry{Ours};
  \addplot table [x expr={\thisrow{epoch}*\ncycles*\rolloutB*\T + \thisrow{epoch}*\ncycles*\batch}, y=test/ag_g_dist, col sep=comma] {\pathprefix -v1-ddpg/progress.csv};
  \addlegendentry{HER};
  \addplot table [x expr={\thisrow{epoch}*\ncycles*\rolloutB*\T + \thisrow{epoch}*\ncycles*\batch}, y=test/ag_g_dist, col sep=comma] {\pathprefix -v1-fwrl/progress.csv};
  \addlegendentry{FWRL};
\end{axis}
  \begin{axis}[at={($(FetchReachDistRew.south east) + (30,0)$)},
name=FetchReachSuccEpoch,
ymin=0,xmin=0,ymax=1,xmax=16,
  xlabel=Epoch,
ylabel=Success Rate (test),
  width=1.0\columnwidth,
  height=0.70\columnwidth,
  legend pos=south east]
% these constants can be read from \pathprefix PR-v1-dqst/params.json
\def\xcol{epoch}
\def\ycol{test/success_rate}
  \addplot table [x=\xcol, y=\ycol, col sep=comma] {\pathprefix PR-v1-dqst/progress.csv};
  \addlegendentry{Ours};
  \addplot table [x=\xcol, y=\ycol, col sep=comma] {\pathprefix -v1-ddpg/progress.csv};
  \addlegendentry{HER};
  \addplot table [x=\xcol, y=\ycol, col sep=comma] {\pathprefix -v1-fwrl/progress.csv};
  \addlegendentry{FWRL};
\end{axis}
  \begin{axis}[at={(FetchReachSuccEpoch.south east)},
    ymin=0,xmin=0,ymax=1,xmax=5e4,
    name=FetchReachSuccRew,
  xlabel=Reward Computations,
  width=1.0\columnwidth,
  height=0.70\columnwidth,
  ytick=\empty,
  legend pos=south east]
% these constants can be read from \pathprefix PR-v1-dqst/params.json
\def\T{50}
\def\ncycles{10}
\def\rolloutB{2}
\def\batch{256}
\def\xexprPR{\thisrow{epoch}*\ncycles*\rolloutB*\T}
\def\xexprGoalRew{\thisrow{epoch}*\ncycles*\rolloutB*\T + \thisrow{epoch}*\ncycles*\batch}
\def\ycol{test/success_rate}

  \addplot table [x expr=\xexprPR, y=\ycol, col sep=comma] {\pathprefix PR-v1-dqst/progress.csv};
  \addlegendentry{Ours};
  \addplot table [x expr=\xexprGoalRew, y=\ycol, col sep=comma] {\pathprefix -v1-ddpg/progress.csv};
  \addlegendentry{HER};
  \addplot table [x expr=\xexprGoalRew, y=\ycol, col sep=comma] {\pathprefix -v1-fwrl/progress.csv};
  \addlegendentry{FWRL};
\end{axis}
    \begingroup
    \def\distxmax{16}
    \def\succxmax{5e4}
    \def\succymax{1}
    \def\rewlegendpos{south east}
    \def\nameprefix{FetchReach}
    \def\fullname{Fetch Reach}
    \def\distymax{0.25}
    \def\pathprefix{media/res/6efc1de-path_reward_low_thresh_chosen-FetchReach}
    \def\axisheight{0.6\columnwidth}
\node [rotate=90,align=center,text width=0.4\columnwidth]
(\nameprefix Name) {\color{blue}\scriptsize \fullname};
\begin{axis}[at={($(\nameprefix Name.south west) + (30, 0)$)},
  ymin=\distymin,xmin=0, ymax=\distymax,xmax=\distxmax,
  name=\nameprefix DistEpoch,
  width=1.0\columnwidth,
  height=\axisheight,
  ylabel=Distance from goal (m),
  legend pos=north east,
  \extrapropsaxisdistepoch,
  ]
  \addplot table [x=epoch, y=test/ag_g_dist, col sep=comma] {\pathprefix PR-\envversion-dqst/progress.csv};
  \addlegendentry{Ours};
  \addplot table [x=epoch, y=test/ag_g_dist, col sep=comma] {\pathprefix -\envversion-ddpg/progress.csv};
  \addlegendentry{HER};
  \addplot table [x=epoch, y=test/ag_g_dist, col sep=comma] {\pathprefix -\envversion-\fwrlsuffix/progress.csv};
  \addlegendentry{FWRL};
  \end{axis}
  \begin{axis}[at={(\nameprefix DistEpoch.south east)},
name=\nameprefix DistRew,
ymin=\distymin,xmin=0,ymax=\distymax,xmax=\succxmax,
  \extrapropsaxisdistrew,
  width=1.0\columnwidth,
  height=\axisheight,
  ytick=\empty,
  legend pos=north east]
  \addplot table [x expr={\thisrow{epoch}*\ncycles*\rolloutB*\T}, y=test/ag_g_dist, col sep=comma] {\pathprefix PR-\envversion-dqst/progress.csv};
  \addlegendentry{Ours};
  \addplot table [x expr={\thisrow{epoch}*\ncycles*\rolloutB*\T + \thisrow{epoch}*\ncycles*\batch}, y=test/ag_g_dist, col sep=comma] {\pathprefix -\envversion-ddpg/progress.csv};
  \addlegendentry{HER};
  \addplot table [x expr={\thisrow{epoch}*\ncycles*\rolloutB*\T + \thisrow{epoch}*\ncycles*\batch}, y=test/ag_g_dist, col sep=comma] {\pathprefix -\envversion-\fwrlsuffix/progress.csv};
  \addlegendentry{FWRL};
\end{axis}
  \begin{axis}[at={($(\nameprefix DistRew.south east) + (30,0)$)},
name=\nameprefix SuccEpoch,
ymin=0,xmin=0,ymax=\succymax,xmax=\distxmax,
  \extrapropsaxissuccepoch,
ylabel=Success Rate (test),
  width=1.0\columnwidth,
  height=\axisheight,
  legend pos=\rewlegendpos]
% these constants can be read from \pathprefix PR-\envversion-dqst/params.json
\def\xcol{epoch}
\def\ycol{test/success_rate}
  \addplot table [x=\xcol, y=\ycol, col sep=comma] {\pathprefix PR-\envversion-dqst/progress.csv};
  \addlegendentry{Ours};
  \addplot table [x=\xcol, y=\ycol, col sep=comma] {\pathprefix -\envversion-ddpg/progress.csv};
  \addlegendentry{HER};
  \addplot table [x=\xcol, y=\ycol, col sep=comma] {\pathprefix -\envversion-\fwrlsuffix/progress.csv};
  \addlegendentry{FWRL};
\end{axis}
  \begin{axis}[at={(\nameprefix SuccEpoch.south east)},
ymin=0,xmin=0,ymax=\succymax,xmax=\succxmax,
name=\nameprefix SuccRew,
  \extrapropsaxissuccrew,
  width=1.0\columnwidth,
  height=\axisheight,
  ytick=\empty,
  legend pos=\rewlegendpos]
\def\xexprPR{\thisrow{epoch}*\ncycles*\rolloutB*\T}
\def\xexprGoalRew{\thisrow{epoch}*\ncycles*\rolloutB*\T + \thisrow{epoch}*\ncycles*\batch}
\def\ycol{test/success_rate}

  \addplot table [x expr=\xexprPR, y=\ycol, col sep=comma] {\pathprefix PR-\envversion-dqst/progress.csv};
  \addlegendentry{Ours};
  \addplot table [x expr=\xexprGoalRew, y=\ycol, col sep=comma] {\pathprefix -\envversion-ddpg/progress.csv};
  \addlegendentry{HER};
  \addplot table [x expr=\xexprGoalRew, y=\ycol, col sep=comma] {\pathprefix -\envversion-\fwrlsuffix/progress.csv};
  \addlegendentry{FWRL};
\end{axis}
    \endgroup
    \node[fit=(FetchReachName) (FetchReachSuccRew)] (fetchreach) {}; 
  \end{scope}
  \begin{scope}[shift={($(fetchreach.south west)+(0.05\columnwidth, \scopeyshift)$)},
    local bounding box=fetchpush]
    %\def\pathprefix{media/res/6efc1de-path_reward_low_thresh_chosen-FetchPush}
\def\distymax{0.25}
  \begin{axis}[ymin=0,xmin=0, ymax=\distymax,xmax=21,
  name=FetchPushDistEpoch,
  xlabel=Epochs,
  width=1.0\columnwidth,
  height=0.70\columnwidth,
  ylabel=Distance from goal (m),
  legend pos=north east]
  \addplot table [x=epoch, y=test/ag_g_dist, col sep=comma] {\pathprefix PR-v1-dqst/progress.csv};
  \addlegendentry{Ours};
  \addplot table [x=epoch, y=test/ag_g_dist, col sep=comma] {\pathprefix -v1-ddpg/progress.csv};
  \addlegendentry{HER};
  \addplot table [x=epoch, y=test/ag_g_dist, col sep=comma] {\pathprefix -v1-fwrl/progress.csv};
  \addlegendentry{FWRL};
  \end{axis}
  \begin{axis}[at={(FetchPushDistEpoch.south east)},
name=FetchPushDistRew,
ymin=0,xmin=0,ymax=\distymax,xmax=6e4,
  xlabel=Reward Computations,
  width=1.0\columnwidth,
  height=0.70\columnwidth,
  ytick=\empty,
  legend pos=north east]
% these constants can be read from \pathprefix */params.json
\def\T{50}
\def\ncycles{10}
\def\rolloutB{2}
\def\batch{256}
  \addplot table [x expr={\thisrow{epoch}*\ncycles*\rolloutB*\T}, y=test/ag_g_dist, col sep=comma] {\pathprefix PR-v1-dqst/progress.csv};
  \addlegendentry{Ours};
  \addplot table [x expr={\thisrow{epoch}*\ncycles*\rolloutB*\T + \thisrow{epoch}*\ncycles*\batch}, y=test/ag_g_dist, col sep=comma] {\pathprefix -v1-ddpg/progress.csv};
  \addlegendentry{HER};
  \addplot table [x expr={\thisrow{epoch}*\ncycles*\rolloutB*\T + \thisrow{epoch}*\ncycles*\batch}, y=test/ag_g_dist, col sep=comma] {\pathprefix -v1-fwrl/progress.csv};
  \addlegendentry{FWRL};
\end{axis}
  \begin{axis}[at={($(FetchPushDistRew.south east) + (30,0)$)},
name=FetchPushSuccEpoch,
ymin=0,xmin=0,ymax=1,xmax=21,
  xlabel=Epoch,
ylabel=Success Rate (test),
  width=1.0\columnwidth,
  height=0.70\columnwidth,
  legend pos=south east]
% these constants can be read from \pathprefix PR-v1-dqst/params.json
\def\xcol{epoch}
\def\ycol{test/success_rate}
  \addplot table [x=\xcol, y=\ycol, col sep=comma] {\pathprefix PR-v1-dqst/progress.csv};
  \addlegendentry{Ours};
  \addplot table [x=\xcol, y=\ycol, col sep=comma] {\pathprefix -v1-ddpg/progress.csv};
  \addlegendentry{HER};
  \addplot table [x=\xcol, y=\ycol, col sep=comma] {\pathprefix -v1-fwrl/progress.csv};
  \addlegendentry{FWRL};
\end{axis}
  \begin{axis}[at={(FetchPushSuccEpoch.south east)},
ymin=0,xmin=0,ymax=1,xmax=6e4,
name=FetchPushSuccRew,
  xlabel=Reward Computations,
  width=1.0\columnwidth,
  height=0.70\columnwidth,
  ytick=\empty,
  legend pos=south east]
% these constants can be read from \pathprefix PR-v1-dqst/params.json
\def\T{50}
\def\ncycles{10}
\def\rolloutB{2}
\def\batch{256}
\def\xexprPR{\thisrow{epoch}*\ncycles*\rolloutB*\T}
\def\xexprGoalRew{\thisrow{epoch}*\ncycles*\rolloutB*\T + \thisrow{epoch}*\ncycles*\batch}
\def\ycol{test/success_rate}

  \addplot table [x expr=\xexprPR, y=\ycol, col sep=comma] {\pathprefix PR-v1-dqst/progress.csv};
  \addlegendentry{Ours};
  \addplot table [x expr=\xexprGoalRew, y=\ycol, col sep=comma] {\pathprefix -v1-ddpg/progress.csv};
  \addlegendentry{HER};
  \addplot table [x expr=\xexprGoalRew, y=\ycol, col sep=comma] {\pathprefix -v1-fwrl/progress.csv};
  \addlegendentry{FWRL};
\end{axis}
LaTeX/MPS finished at Wed Nov  7 18:01:26

    \begingroup
    \def\rewlegendpos{south east}
    \def\distxmax{21}
    \def\succxmax{6e4}
    \def\succymax{1}
    \def\nameprefix{FetchPush}
    \def\fullname{Fetch Push}
    \def\distymax{0.25}
    \def\pathprefix{media/res/6efc1de-path_reward_low_thresh_chosen-FetchPush}
    \def\axisheight{0.6\columnwidth}
\node [rotate=90,align=center,text width=0.4\columnwidth]
(\nameprefix Name) {\color{blue}\scriptsize \fullname};
\begin{axis}[at={($(\nameprefix Name.south west) + (30, 0)$)},
  ymin=\distymin,xmin=0, ymax=\distymax,xmax=\distxmax,
  name=\nameprefix DistEpoch,
  width=1.0\columnwidth,
  height=\axisheight,
  ylabel=Distance from goal (m),
  legend pos=north east,
  \extrapropsaxisdistepoch,
  ]
  \addplot table [x=epoch, y=test/ag_g_dist, col sep=comma] {\pathprefix PR-\envversion-dqst/progress.csv};
  \addlegendentry{Ours};
  \addplot table [x=epoch, y=test/ag_g_dist, col sep=comma] {\pathprefix -\envversion-ddpg/progress.csv};
  \addlegendentry{HER};
  \addplot table [x=epoch, y=test/ag_g_dist, col sep=comma] {\pathprefix -\envversion-\fwrlsuffix/progress.csv};
  \addlegendentry{FWRL};
  \end{axis}
  \begin{axis}[at={(\nameprefix DistEpoch.south east)},
name=\nameprefix DistRew,
ymin=\distymin,xmin=0,ymax=\distymax,xmax=\succxmax,
  \extrapropsaxisdistrew,
  width=1.0\columnwidth,
  height=\axisheight,
  ytick=\empty,
  legend pos=north east]
  \addplot table [x expr={\thisrow{epoch}*\ncycles*\rolloutB*\T}, y=test/ag_g_dist, col sep=comma] {\pathprefix PR-\envversion-dqst/progress.csv};
  \addlegendentry{Ours};
  \addplot table [x expr={\thisrow{epoch}*\ncycles*\rolloutB*\T + \thisrow{epoch}*\ncycles*\batch}, y=test/ag_g_dist, col sep=comma] {\pathprefix -\envversion-ddpg/progress.csv};
  \addlegendentry{HER};
  \addplot table [x expr={\thisrow{epoch}*\ncycles*\rolloutB*\T + \thisrow{epoch}*\ncycles*\batch}, y=test/ag_g_dist, col sep=comma] {\pathprefix -\envversion-\fwrlsuffix/progress.csv};
  \addlegendentry{FWRL};
\end{axis}
  \begin{axis}[at={($(\nameprefix DistRew.south east) + (30,0)$)},
name=\nameprefix SuccEpoch,
ymin=0,xmin=0,ymax=\succymax,xmax=\distxmax,
  \extrapropsaxissuccepoch,
ylabel=Success Rate (test),
  width=1.0\columnwidth,
  height=\axisheight,
  legend pos=\rewlegendpos]
% these constants can be read from \pathprefix PR-\envversion-dqst/params.json
\def\xcol{epoch}
\def\ycol{test/success_rate}
  \addplot table [x=\xcol, y=\ycol, col sep=comma] {\pathprefix PR-\envversion-dqst/progress.csv};
  \addlegendentry{Ours};
  \addplot table [x=\xcol, y=\ycol, col sep=comma] {\pathprefix -\envversion-ddpg/progress.csv};
  \addlegendentry{HER};
  \addplot table [x=\xcol, y=\ycol, col sep=comma] {\pathprefix -\envversion-\fwrlsuffix/progress.csv};
  \addlegendentry{FWRL};
\end{axis}
  \begin{axis}[at={(\nameprefix SuccEpoch.south east)},
ymin=0,xmin=0,ymax=\succymax,xmax=\succxmax,
name=\nameprefix SuccRew,
  \extrapropsaxissuccrew,
  width=1.0\columnwidth,
  height=\axisheight,
  ytick=\empty,
  legend pos=\rewlegendpos]
\def\xexprPR{\thisrow{epoch}*\ncycles*\rolloutB*\T}
\def\xexprGoalRew{\thisrow{epoch}*\ncycles*\rolloutB*\T + \thisrow{epoch}*\ncycles*\batch}
\def\ycol{test/success_rate}

  \addplot table [x expr=\xexprPR, y=\ycol, col sep=comma] {\pathprefix PR-\envversion-dqst/progress.csv};
  \addlegendentry{Ours};
  \addplot table [x expr=\xexprGoalRew, y=\ycol, col sep=comma] {\pathprefix -\envversion-ddpg/progress.csv};
  \addlegendentry{HER};
  \addplot table [x expr=\xexprGoalRew, y=\ycol, col sep=comma] {\pathprefix -\envversion-\fwrlsuffix/progress.csv};
  \addlegendentry{FWRL};
\end{axis}
    \endgroup
    \node[fit=(FetchPushName) (FetchPushSuccRew)] (fetchpush) {}; 
  \end{scope}
  \begin{scope}[shift={($(fetchpush.south west)+(0.05\columnwidth, \scopeyshift)$)},
    local bounding box=fetchpickandplace]
    \begingroup
    \def\rewlegendpos{south east}
    \def\distxmax{33}
    \def\distymax{0.25}
    \def\succxmax{10e4}
    \def\succymax{1}
    \def\nameprefix{FetchPickAndPlace}
    \def\fullname{Fetch Pick And Place}
    \def\pathprefix{media/res/6efc1de-path_reward_low_thresh_chosen-FetchPickAndPlace}
    \def\axisheight{0.6\columnwidth}
\node [rotate=90,align=center,text width=0.4\columnwidth]
(\nameprefix Name) {\color{blue}\scriptsize \fullname};
\begin{axis}[at={($(\nameprefix Name.south west) + (30, 0)$)},
  ymin=\distymin,xmin=0, ymax=\distymax,xmax=\distxmax,
  name=\nameprefix DistEpoch,
  width=1.0\columnwidth,
  height=\axisheight,
  ylabel=Distance from goal (m),
  legend pos=north east,
  \extrapropsaxisdistepoch,
  ]
  \addplot table [x=epoch, y=test/ag_g_dist, col sep=comma] {\pathprefix PR-\envversion-dqst/progress.csv};
  \addlegendentry{Ours};
  \addplot table [x=epoch, y=test/ag_g_dist, col sep=comma] {\pathprefix -\envversion-ddpg/progress.csv};
  \addlegendentry{HER};
  \addplot table [x=epoch, y=test/ag_g_dist, col sep=comma] {\pathprefix -\envversion-\fwrlsuffix/progress.csv};
  \addlegendentry{FWRL};
  \end{axis}
  \begin{axis}[at={(\nameprefix DistEpoch.south east)},
name=\nameprefix DistRew,
ymin=\distymin,xmin=0,ymax=\distymax,xmax=\succxmax,
  \extrapropsaxisdistrew,
  width=1.0\columnwidth,
  height=\axisheight,
  ytick=\empty,
  legend pos=north east]
  \addplot table [x expr={\thisrow{epoch}*\ncycles*\rolloutB*\T}, y=test/ag_g_dist, col sep=comma] {\pathprefix PR-\envversion-dqst/progress.csv};
  \addlegendentry{Ours};
  \addplot table [x expr={\thisrow{epoch}*\ncycles*\rolloutB*\T + \thisrow{epoch}*\ncycles*\batch}, y=test/ag_g_dist, col sep=comma] {\pathprefix -\envversion-ddpg/progress.csv};
  \addlegendentry{HER};
  \addplot table [x expr={\thisrow{epoch}*\ncycles*\rolloutB*\T + \thisrow{epoch}*\ncycles*\batch}, y=test/ag_g_dist, col sep=comma] {\pathprefix -\envversion-\fwrlsuffix/progress.csv};
  \addlegendentry{FWRL};
\end{axis}
  \begin{axis}[at={($(\nameprefix DistRew.south east) + (30,0)$)},
name=\nameprefix SuccEpoch,
ymin=0,xmin=0,ymax=\succymax,xmax=\distxmax,
  \extrapropsaxissuccepoch,
ylabel=Success Rate (test),
  width=1.0\columnwidth,
  height=\axisheight,
  legend pos=\rewlegendpos]
% these constants can be read from \pathprefix PR-\envversion-dqst/params.json
\def\xcol{epoch}
\def\ycol{test/success_rate}
  \addplot table [x=\xcol, y=\ycol, col sep=comma] {\pathprefix PR-\envversion-dqst/progress.csv};
  \addlegendentry{Ours};
  \addplot table [x=\xcol, y=\ycol, col sep=comma] {\pathprefix -\envversion-ddpg/progress.csv};
  \addlegendentry{HER};
  \addplot table [x=\xcol, y=\ycol, col sep=comma] {\pathprefix -\envversion-\fwrlsuffix/progress.csv};
  \addlegendentry{FWRL};
\end{axis}
  \begin{axis}[at={(\nameprefix SuccEpoch.south east)},
ymin=0,xmin=0,ymax=\succymax,xmax=\succxmax,
name=\nameprefix SuccRew,
  \extrapropsaxissuccrew,
  width=1.0\columnwidth,
  height=\axisheight,
  ytick=\empty,
  legend pos=\rewlegendpos]
\def\xexprPR{\thisrow{epoch}*\ncycles*\rolloutB*\T}
\def\xexprGoalRew{\thisrow{epoch}*\ncycles*\rolloutB*\T + \thisrow{epoch}*\ncycles*\batch}
\def\ycol{test/success_rate}

  \addplot table [x expr=\xexprPR, y=\ycol, col sep=comma] {\pathprefix PR-\envversion-dqst/progress.csv};
  \addlegendentry{Ours};
  \addplot table [x expr=\xexprGoalRew, y=\ycol, col sep=comma] {\pathprefix -\envversion-ddpg/progress.csv};
  \addlegendentry{HER};
  \addplot table [x expr=\xexprGoalRew, y=\ycol, col sep=comma] {\pathprefix -\envversion-\fwrlsuffix/progress.csv};
  \addlegendentry{FWRL};
\end{axis}
    \endgroup
    \node[fit=(FetchPickAndPlaceName) (FetchPickAndPlaceSuccRew)] (fetchpickandplace) {}; 
  \end{scope}
  \begin{scope}[shift={($(fetchpickandplace.south west)+(0.05\columnwidth, \scopeyshift)$)},
    local bounding box=fetchslide]
    \begingroup
    \def\rewlegendpos{north west}
    \def\distxmax{200}
    \def\distymax{0.50}
    \def\succxmax{20e4}
    \def\succymax{0.15}
    \def\nameprefix{FetchSlide}
    \def\fullname{Fetch Slide}
    \def\pathprefix{media/res/6efc1de-path_reward_low_thresh_chosen-FetchSlide}
    \def\extrapropsaxisdistepoch{xlabel=Epochs}
    \def\extrapropsaxissuccepoch{xlabel=Epochs},
    \def\extrapropsaxisdistrew{xlabel=Reward Computations},
    \def\extrapropsaxissuccrew{xlabel=Reward Computations},
    \def\axisheight{0.6\columnwidth}
\node [rotate=90,align=center,text width=0.4\columnwidth]
(\nameprefix Name) {\color{blue}\scriptsize \fullname};
\begin{axis}[at={($(\nameprefix Name.south west) + (30, 0)$)},
  ymin=\distymin,xmin=0, ymax=\distymax,xmax=\distxmax,
  name=\nameprefix DistEpoch,
  width=1.0\columnwidth,
  height=\axisheight,
  ylabel=Distance from goal (m),
  legend pos=north east,
  \extrapropsaxisdistepoch,
  ]
  \addplot table [x=epoch, y=test/ag_g_dist, col sep=comma] {\pathprefix PR-\envversion-dqst/progress.csv};
  \addlegendentry{Ours};
  \addplot table [x=epoch, y=test/ag_g_dist, col sep=comma] {\pathprefix -\envversion-ddpg/progress.csv};
  \addlegendentry{HER};
  \addplot table [x=epoch, y=test/ag_g_dist, col sep=comma] {\pathprefix -\envversion-\fwrlsuffix/progress.csv};
  \addlegendentry{FWRL};
  \end{axis}
  \begin{axis}[at={(\nameprefix DistEpoch.south east)},
name=\nameprefix DistRew,
ymin=\distymin,xmin=0,ymax=\distymax,xmax=\succxmax,
  \extrapropsaxisdistrew,
  width=1.0\columnwidth,
  height=\axisheight,
  ytick=\empty,
  legend pos=north east]
  \addplot table [x expr={\thisrow{epoch}*\ncycles*\rolloutB*\T}, y=test/ag_g_dist, col sep=comma] {\pathprefix PR-\envversion-dqst/progress.csv};
  \addlegendentry{Ours};
  \addplot table [x expr={\thisrow{epoch}*\ncycles*\rolloutB*\T + \thisrow{epoch}*\ncycles*\batch}, y=test/ag_g_dist, col sep=comma] {\pathprefix -\envversion-ddpg/progress.csv};
  \addlegendentry{HER};
  \addplot table [x expr={\thisrow{epoch}*\ncycles*\rolloutB*\T + \thisrow{epoch}*\ncycles*\batch}, y=test/ag_g_dist, col sep=comma] {\pathprefix -\envversion-\fwrlsuffix/progress.csv};
  \addlegendentry{FWRL};
\end{axis}
  \begin{axis}[at={($(\nameprefix DistRew.south east) + (30,0)$)},
name=\nameprefix SuccEpoch,
ymin=0,xmin=0,ymax=\succymax,xmax=\distxmax,
  \extrapropsaxissuccepoch,
ylabel=Success Rate (test),
  width=1.0\columnwidth,
  height=\axisheight,
  legend pos=\rewlegendpos]
% these constants can be read from \pathprefix PR-\envversion-dqst/params.json
\def\xcol{epoch}
\def\ycol{test/success_rate}
  \addplot table [x=\xcol, y=\ycol, col sep=comma] {\pathprefix PR-\envversion-dqst/progress.csv};
  \addlegendentry{Ours};
  \addplot table [x=\xcol, y=\ycol, col sep=comma] {\pathprefix -\envversion-ddpg/progress.csv};
  \addlegendentry{HER};
  \addplot table [x=\xcol, y=\ycol, col sep=comma] {\pathprefix -\envversion-\fwrlsuffix/progress.csv};
  \addlegendentry{FWRL};
\end{axis}
  \begin{axis}[at={(\nameprefix SuccEpoch.south east)},
ymin=0,xmin=0,ymax=\succymax,xmax=\succxmax,
name=\nameprefix SuccRew,
  \extrapropsaxissuccrew,
  width=1.0\columnwidth,
  height=\axisheight,
  ytick=\empty,
  legend pos=\rewlegendpos]
\def\xexprPR{\thisrow{epoch}*\ncycles*\rolloutB*\T}
\def\xexprGoalRew{\thisrow{epoch}*\ncycles*\rolloutB*\T + \thisrow{epoch}*\ncycles*\batch}
\def\ycol{test/success_rate}

  \addplot table [x expr=\xexprPR, y=\ycol, col sep=comma] {\pathprefix PR-\envversion-dqst/progress.csv};
  \addlegendentry{Ours};
  \addplot table [x expr=\xexprGoalRew, y=\ycol, col sep=comma] {\pathprefix -\envversion-ddpg/progress.csv};
  \addlegendentry{HER};
  \addplot table [x expr=\xexprGoalRew, y=\ycol, col sep=comma] {\pathprefix -\envversion-\fwrlsuffix/progress.csv};
  \addlegendentry{FWRL};
\end{axis}
    \endgroup
    \node[fit=(FetchSlideName) (FetchSlideSuccRew)] (fetchpickandplace) {}; 
  \end{scope}
\end{tikzpicture}
%\end{figure}%
%\end{wrapfigure}%
\end{minipage}%
}
\caption{For the Fetch tasks, we compare our method (red) against HER (blue) ~\citep{andrychowicz2016learning}
  and FWRL (green) ~\citep{kaelbling1993learning} on the distance-from-goal
  and success rate metrics. Both metrics are plotted
  against two progress measures: the number of training epochs and the number of reward
  computations. Except for the Fetch Slide task, we achieve comparable or
  better performance across the metrics and progress measures. 
}%
\label{fig:fetch-results}%
\end{figure}

\subsection{Metrics}
Similar to prior work, we evaluate all experiments on two metrics: the success
rate and the average distance to the goal. The success rate is defined as the
fraction of episodes in which the agent is able to reach the goal within a
pre-defined threshold region.
% $\frac{1}{E}\sum_{e,t} \mathbbm{1}_{\|\goal_t - \goal^*\|_2 < \epsilon}$.
The metric \emph{distance of the goal} is the euclidean distance between
the achieved goal and the desired goal in meters.
% $\|\goal_t - \goal^*\|_2$.
These metrics are plotted against a standard progress measure, the
number of training epochs, showing
comparable results of our method to the baselines.

To emphasize that our method does not require goal-reward
and reward re-computation, we plot these metrics against another
progress measure, the number of reward computations used during
training. This includes both the episode rollouts and the reward recomputations
during HER sampling.

%

\subsection{Hyper-parameters choices} \label{sec:hyperparams}
Unless specified, all our hyper-parameters are identical to the ones
used in the HER
implementation~\citep{dhariwal2017baselines}. We note two main changes
to HER to make the comparison more fair. Firstly,
we use a smaller \emph{distance-threshold}.
The environment used for HER and FWRL returns the goal-reward when the
achieved goal is within this threshold of the desired goal. Because of
the absence of goal-rewards, the distance-threshold information is not used by our
method.
We reduce the threshold to 1cm which is reduction by a factor of 5 compared to
HER.

Secondly, we run all experiments on 6 cores each,
while HER uses 19. The batch size used is a function of the number of
cores and hence this parameter has a significant effect on learning. 

To ensure fair comparison, all experiments are run with the same hyper-parameters and
random seeds to ensure that variations in performance are purely due
to differences between the algorithms.

\subsection{Results}
% Due to space limitations and in the interest of clarity we show a subset our experiments across both platforms that emphasize both strengths and weaknesses of our algorithm.
All our experimental results are described below, highlighting the strengths and
weaknesses of our algorithm. Across all our experiments, the
distance-to-the-goal metric achieves comparable performance to HER
\emph{without requiring goal-rewards}. 

\paragraph{Fetch Tasks}

The experimental results for Fetch tasks are shown in
Figure~\ref{fig:fetch-results}. For the Fetch Reach and Push tasks, our
method achieves comparable performance to the baselines 
across both metrics in terms of training epochs and outperforms them in
terms of reward recomputations. Notably, the Fetch
Pick and Place task trains in significantly fewer epochs. For the Fetch
Slide task the opposite is true.
We conjecture that Fetch Slide is more sensitive to the
distance threshold information, which our method is unable to use.

\paragraph{Hand Tasks}

For the Hand tasks, the distance to the goal and the success rate show different trends.
We show the results in Figure~\ref{fig:hand-results}.
When the distance metric is plotted against epochs, we get comparable
performance for all tasks; when plotted against reward computations, we outperform
all baselines on all tasks except Hand Reach. The baselines perform
well enough on this task, leaving less scope for significant improvement.
These trends do not hold for the success rate metric, on which our method
consistently under-performs compared to the baselines across tasks. This
is surprising, as all algorithms average equally on the distance-from-goal metric.
We conjecture that this might be the result of
high-distance failure cases of the baselines, i.e. when the baselines
fail, they do so at larger distances from the goal. In contrast, we
assume our method's success and failure cases are closer together. 
\begin{figure}
\scalebox{0.5}{%
  \begin{minipage}{0.55\linewidth}
\begin{tikzpicture}
  \def\T{50}
  \def\ncycles{50}
  \def\rolloutB{2}
  \def\batch{256}
  \def\fwrlsuffix{fwrl}
  \def\envversion{v0}
  \def\scopexshift{0.05\columnwidth}
  \def\scopeyshift{-0.3\columnwidth}
  \def\succxmax{25e4}
  \def\succymax{1}
  \def\distxmax{49}
  \def\distymax{1}
  \def\distymin{0}
  \def\rewlegendpos{south east}
  \def\extrapropsaxisdistepoch{xlabel=Epochs}
  \def\extrapropsaxissuccepoch{xlabel=Epochs},
  \def\extrapropsaxisdistrew{xlabel=Reward Computations},
  \def\extrapropsaxissuccrew{xlabel=Reward Computations},
  \begin{scope}[local bounding box=handreach]
    \begingroup
    \def\distymax{0.099}
    \def\distymin{0.02}
    \def\nameprefix{HandReach}
    \def\fullname{Hand Reach}
    \def\pathprefix{DhBaSiCoICLR2019/media/res/6efc1de-path_reward_low_thresh_chosen-HandReach}
    \def\axisheight{0.6\columnwidth}
\node [rotate=90,align=center,text width=0.4\columnwidth]
(\nameprefix Name) {\color{blue}\scriptsize \fullname};
\begin{axis}[at={($(\nameprefix Name.south west) + (30, 0)$)},
  ymin=\distymin,xmin=0, ymax=\distymax,xmax=\distxmax,
  name=\nameprefix DistEpoch,
  width=1.0\columnwidth,
  height=\axisheight,
  ylabel=Distance from goal (m),
  legend pos=north east,
  \extrapropsaxisdistepoch,
  ]
  \addplot table [x=epoch, y=test/ag_g_dist, col sep=comma] {\pathprefix PR-\envversion-dqst/progress.csv};
  \addlegendentry{Ours};
  \addplot table [x=epoch, y=test/ag_g_dist, col sep=comma] {\pathprefix -\envversion-ddpg/progress.csv};
  \addlegendentry{HER};
  \addplot table [x=epoch, y=test/ag_g_dist, col sep=comma] {\pathprefix -\envversion-\fwrlsuffix/progress.csv};
  \addlegendentry{FWRL};
  \end{axis}
  \begin{axis}[at={(\nameprefix DistEpoch.south east)},
name=\nameprefix DistRew,
ymin=\distymin,xmin=0,ymax=\distymax,xmax=\succxmax,
  \extrapropsaxisdistrew,
  width=1.0\columnwidth,
  height=\axisheight,
  ytick=\empty,
  legend pos=north east]
  \addplot table [x expr={\thisrow{epoch}*\ncycles*\rolloutB*\T}, y=test/ag_g_dist, col sep=comma] {\pathprefix PR-\envversion-dqst/progress.csv};
  \addlegendentry{Ours};
  \addplot table [x expr={\thisrow{epoch}*\ncycles*\rolloutB*\T + \thisrow{epoch}*\ncycles*\batch}, y=test/ag_g_dist, col sep=comma] {\pathprefix -\envversion-ddpg/progress.csv};
  \addlegendentry{HER};
  \addplot table [x expr={\thisrow{epoch}*\ncycles*\rolloutB*\T + \thisrow{epoch}*\ncycles*\batch}, y=test/ag_g_dist, col sep=comma] {\pathprefix -\envversion-\fwrlsuffix/progress.csv};
  \addlegendentry{FWRL};
\end{axis}
  \begin{axis}[at={($(\nameprefix DistRew.south east) + (30,0)$)},
name=\nameprefix SuccEpoch,
ymin=0,xmin=0,ymax=\succymax,xmax=\distxmax,
  \extrapropsaxissuccepoch,
ylabel=Success Rate (test),
  width=1.0\columnwidth,
  height=\axisheight,
  legend pos=\rewlegendpos]
% these constants can be read from \pathprefix PR-\envversion-dqst/params.json
\def\xcol{epoch}
\def\ycol{test/success_rate}
  \addplot table [x=\xcol, y=\ycol, col sep=comma] {\pathprefix PR-\envversion-dqst/progress.csv};
  \addlegendentry{Ours};
  \addplot table [x=\xcol, y=\ycol, col sep=comma] {\pathprefix -\envversion-ddpg/progress.csv};
  \addlegendentry{HER};
  \addplot table [x=\xcol, y=\ycol, col sep=comma] {\pathprefix -\envversion-\fwrlsuffix/progress.csv};
  \addlegendentry{FWRL};
\end{axis}
  \begin{axis}[at={(\nameprefix SuccEpoch.south east)},
ymin=0,xmin=0,ymax=\succymax,xmax=\succxmax,
name=\nameprefix SuccRew,
  \extrapropsaxissuccrew,
  width=1.0\columnwidth,
  height=\axisheight,
  ytick=\empty,
  legend pos=\rewlegendpos]
\def\xexprPR{\thisrow{epoch}*\ncycles*\rolloutB*\T}
\def\xexprGoalRew{\thisrow{epoch}*\ncycles*\rolloutB*\T + \thisrow{epoch}*\ncycles*\batch}
\def\ycol{test/success_rate}

  \addplot table [x expr=\xexprPR, y=\ycol, col sep=comma] {\pathprefix PR-\envversion-dqst/progress.csv};
  \addlegendentry{Ours};
  \addplot table [x expr=\xexprGoalRew, y=\ycol, col sep=comma] {\pathprefix -\envversion-ddpg/progress.csv};
  \addlegendentry{HER};
  \addplot table [x expr=\xexprGoalRew, y=\ycol, col sep=comma] {\pathprefix -\envversion-\fwrlsuffix/progress.csv};
  \addlegendentry{FWRL};
\end{axis}
    \endgroup
    \node[fit=(HandReachName) (HandReachSuccRew)] (handreach) {}; 
  \end{scope}
  \begin{scope}[shift={($(handreach.south west)+(0.05\columnwidth, \scopeyshift)$)},
    local bounding box=handblock]
    \begingroup
    \def\succymax{0.6}
    \def\distymax{1.1}
    \def\distymin{0.4}
    \def\rewlegendpos{north west}
    \def\nameprefix{HandBlock}
    \def\fullname{ManipulateBlockRotateXYZ}
    \def\pathprefix{DhBaSiCoICLR2019/media/res/6efc1de-path_reward_low_thresh_chosen-HandManipulateBlockRotateXYZ}
    \def\axisheight{0.6\columnwidth}
\node [rotate=90,align=center,text width=0.4\columnwidth]
(\nameprefix Name) {\color{blue}\scriptsize \fullname};
\begin{axis}[at={($(\nameprefix Name.south west) + (30, 0)$)},
  ymin=\distymin,xmin=0, ymax=\distymax,xmax=\distxmax,
  name=\nameprefix DistEpoch,
  width=1.0\columnwidth,
  height=\axisheight,
  ylabel=Distance from goal (m),
  legend pos=north east,
  \extrapropsaxisdistepoch,
  ]
  \addplot table [x=epoch, y=test/ag_g_dist, col sep=comma] {\pathprefix PR-\envversion-dqst/progress.csv};
  \addlegendentry{Ours};
  \addplot table [x=epoch, y=test/ag_g_dist, col sep=comma] {\pathprefix -\envversion-ddpg/progress.csv};
  \addlegendentry{HER};
  \addplot table [x=epoch, y=test/ag_g_dist, col sep=comma] {\pathprefix -\envversion-\fwrlsuffix/progress.csv};
  \addlegendentry{FWRL};
  \end{axis}
  \begin{axis}[at={(\nameprefix DistEpoch.south east)},
name=\nameprefix DistRew,
ymin=\distymin,xmin=0,ymax=\distymax,xmax=\succxmax,
  \extrapropsaxisdistrew,
  width=1.0\columnwidth,
  height=\axisheight,
  ytick=\empty,
  legend pos=north east]
  \addplot table [x expr={\thisrow{epoch}*\ncycles*\rolloutB*\T}, y=test/ag_g_dist, col sep=comma] {\pathprefix PR-\envversion-dqst/progress.csv};
  \addlegendentry{Ours};
  \addplot table [x expr={\thisrow{epoch}*\ncycles*\rolloutB*\T + \thisrow{epoch}*\ncycles*\batch}, y=test/ag_g_dist, col sep=comma] {\pathprefix -\envversion-ddpg/progress.csv};
  \addlegendentry{HER};
  \addplot table [x expr={\thisrow{epoch}*\ncycles*\rolloutB*\T + \thisrow{epoch}*\ncycles*\batch}, y=test/ag_g_dist, col sep=comma] {\pathprefix -\envversion-\fwrlsuffix/progress.csv};
  \addlegendentry{FWRL};
\end{axis}
  \begin{axis}[at={($(\nameprefix DistRew.south east) + (30,0)$)},
name=\nameprefix SuccEpoch,
ymin=0,xmin=0,ymax=\succymax,xmax=\distxmax,
  \extrapropsaxissuccepoch,
ylabel=Success Rate (test),
  width=1.0\columnwidth,
  height=\axisheight,
  legend pos=\rewlegendpos]
% these constants can be read from \pathprefix PR-\envversion-dqst/params.json
\def\xcol{epoch}
\def\ycol{test/success_rate}
  \addplot table [x=\xcol, y=\ycol, col sep=comma] {\pathprefix PR-\envversion-dqst/progress.csv};
  \addlegendentry{Ours};
  \addplot table [x=\xcol, y=\ycol, col sep=comma] {\pathprefix -\envversion-ddpg/progress.csv};
  \addlegendentry{HER};
  \addplot table [x=\xcol, y=\ycol, col sep=comma] {\pathprefix -\envversion-\fwrlsuffix/progress.csv};
  \addlegendentry{FWRL};
\end{axis}
  \begin{axis}[at={(\nameprefix SuccEpoch.south east)},
ymin=0,xmin=0,ymax=\succymax,xmax=\succxmax,
name=\nameprefix SuccRew,
  \extrapropsaxissuccrew,
  width=1.0\columnwidth,
  height=\axisheight,
  ytick=\empty,
  legend pos=\rewlegendpos]
\def\xexprPR{\thisrow{epoch}*\ncycles*\rolloutB*\T}
\def\xexprGoalRew{\thisrow{epoch}*\ncycles*\rolloutB*\T + \thisrow{epoch}*\ncycles*\batch}
\def\ycol{test/success_rate}

  \addplot table [x expr=\xexprPR, y=\ycol, col sep=comma] {\pathprefix PR-\envversion-dqst/progress.csv};
  \addlegendentry{Ours};
  \addplot table [x expr=\xexprGoalRew, y=\ycol, col sep=comma] {\pathprefix -\envversion-ddpg/progress.csv};
  \addlegendentry{HER};
  \addplot table [x expr=\xexprGoalRew, y=\ycol, col sep=comma] {\pathprefix -\envversion-\fwrlsuffix/progress.csv};
  \addlegendentry{FWRL};
\end{axis}
    \endgroup
    \node[fit=(HandBlockName) (HandBlockSuccRew)] (handblock) {}; 
  \end{scope}
  \begin{scope}[shift={($(handblock.south west)+(0.05\columnwidth, \scopeyshift)$)},
    local bounding box=handegg]
    \begingroup
    \def\succymax{0.25}
    \def\distymax{1.1}
    \def\distymin{0.4}
    \def\rewlegendpos{north west}
    \def\nameprefix{HandEgg}
    \def\fullname{ManipulateEggFull}
    \def\pathprefix{DhBaSiCoICLR2019/media/res/6efc1de-path_reward_low_thresh_chosen-HandManipulateEggFull}
    \def\axisheight{0.6\columnwidth}
\node [rotate=90,align=center,text width=0.4\columnwidth]
(\nameprefix Name) {\color{blue}\scriptsize \fullname};
\begin{axis}[at={($(\nameprefix Name.south west) + (30, 0)$)},
  ymin=\distymin,xmin=0, ymax=\distymax,xmax=\distxmax,
  name=\nameprefix DistEpoch,
  width=1.0\columnwidth,
  height=\axisheight,
  ylabel=Distance from goal (m),
  legend pos=north east,
  \extrapropsaxisdistepoch,
  ]
  \addplot table [x=epoch, y=test/ag_g_dist, col sep=comma] {\pathprefix PR-\envversion-dqst/progress.csv};
  \addlegendentry{Ours};
  \addplot table [x=epoch, y=test/ag_g_dist, col sep=comma] {\pathprefix -\envversion-ddpg/progress.csv};
  \addlegendentry{HER};
  \addplot table [x=epoch, y=test/ag_g_dist, col sep=comma] {\pathprefix -\envversion-\fwrlsuffix/progress.csv};
  \addlegendentry{FWRL};
  \end{axis}
  \begin{axis}[at={(\nameprefix DistEpoch.south east)},
name=\nameprefix DistRew,
ymin=\distymin,xmin=0,ymax=\distymax,xmax=\succxmax,
  \extrapropsaxisdistrew,
  width=1.0\columnwidth,
  height=\axisheight,
  ytick=\empty,
  legend pos=north east]
  \addplot table [x expr={\thisrow{epoch}*\ncycles*\rolloutB*\T}, y=test/ag_g_dist, col sep=comma] {\pathprefix PR-\envversion-dqst/progress.csv};
  \addlegendentry{Ours};
  \addplot table [x expr={\thisrow{epoch}*\ncycles*\rolloutB*\T + \thisrow{epoch}*\ncycles*\batch}, y=test/ag_g_dist, col sep=comma] {\pathprefix -\envversion-ddpg/progress.csv};
  \addlegendentry{HER};
  \addplot table [x expr={\thisrow{epoch}*\ncycles*\rolloutB*\T + \thisrow{epoch}*\ncycles*\batch}, y=test/ag_g_dist, col sep=comma] {\pathprefix -\envversion-\fwrlsuffix/progress.csv};
  \addlegendentry{FWRL};
\end{axis}
  \begin{axis}[at={($(\nameprefix DistRew.south east) + (30,0)$)},
name=\nameprefix SuccEpoch,
ymin=0,xmin=0,ymax=\succymax,xmax=\distxmax,
  \extrapropsaxissuccepoch,
ylabel=Success Rate (test),
  width=1.0\columnwidth,
  height=\axisheight,
  legend pos=\rewlegendpos]
% these constants can be read from \pathprefix PR-\envversion-dqst/params.json
\def\xcol{epoch}
\def\ycol{test/success_rate}
  \addplot table [x=\xcol, y=\ycol, col sep=comma] {\pathprefix PR-\envversion-dqst/progress.csv};
  \addlegendentry{Ours};
  \addplot table [x=\xcol, y=\ycol, col sep=comma] {\pathprefix -\envversion-ddpg/progress.csv};
  \addlegendentry{HER};
  \addplot table [x=\xcol, y=\ycol, col sep=comma] {\pathprefix -\envversion-\fwrlsuffix/progress.csv};
  \addlegendentry{FWRL};
\end{axis}
  \begin{axis}[at={(\nameprefix SuccEpoch.south east)},
ymin=0,xmin=0,ymax=\succymax,xmax=\succxmax,
name=\nameprefix SuccRew,
  \extrapropsaxissuccrew,
  width=1.0\columnwidth,
  height=\axisheight,
  ytick=\empty,
  legend pos=\rewlegendpos]
\def\xexprPR{\thisrow{epoch}*\ncycles*\rolloutB*\T}
\def\xexprGoalRew{\thisrow{epoch}*\ncycles*\rolloutB*\T + \thisrow{epoch}*\ncycles*\batch}
\def\ycol{test/success_rate}

  \addplot table [x expr=\xexprPR, y=\ycol, col sep=comma] {\pathprefix PR-\envversion-dqst/progress.csv};
  \addlegendentry{Ours};
  \addplot table [x expr=\xexprGoalRew, y=\ycol, col sep=comma] {\pathprefix -\envversion-ddpg/progress.csv};
  \addlegendentry{HER};
  \addplot table [x expr=\xexprGoalRew, y=\ycol, col sep=comma] {\pathprefix -\envversion-\fwrlsuffix/progress.csv};
  \addlegendentry{FWRL};
\end{axis}
    \endgroup
    \node[fit=(HandEggName) (HandEggSuccRew)] (handegg) {}; 
  \end{scope}
  \begin{scope}[shift={($(handegg.south west)+(0.05\columnwidth, \scopeyshift)$)},
    local bounding box=handpen]
    \begingroup
    \def\extrapropsaxisdistepoch{xlabel=Epochs}
    \def\extrapropsaxissuccepoch{xlabel=Epochs},
    \def\extrapropsaxisdistrew{xlabel=Reward Computations},
    \def\extrapropsaxissuccrew{xlabel=Reward Computations},
    \def\succymax{0.25}
    \def\distymax{1.1}
    \def\distymin{0.6}
    \def\nameprefix{HandPen}
    \def\rewlegendpos{north west}
    \def\fullname{PenRotate}
    %\def\pathprefix{media/res/6efc1de-path_reward_low_thresh_chosen-HandManipulatePenRotate}
    \def\fwrlsuffix{qlst}
    \def\pathprefix{DhBaSiCoICLR2019/media/res/be467df-path_reward_low_thresh_alt-HandManipulatePenRotate}
    \def\axisheight{0.6\columnwidth}
\node [rotate=90,align=center,text width=0.4\columnwidth]
(\nameprefix Name) {\color{blue}\scriptsize \fullname};
\begin{axis}[at={($(\nameprefix Name.south west) + (30, 0)$)},
  ymin=\distymin,xmin=0, ymax=\distymax,xmax=\distxmax,
  name=\nameprefix DistEpoch,
  width=1.0\columnwidth,
  height=\axisheight,
  ylabel=Distance from goal (m),
  legend pos=north east,
  \extrapropsaxisdistepoch,
  ]
  \addplot table [x=epoch, y=test/ag_g_dist, col sep=comma] {\pathprefix PR-\envversion-dqst/progress.csv};
  \addlegendentry{Ours};
  \addplot table [x=epoch, y=test/ag_g_dist, col sep=comma] {\pathprefix -\envversion-ddpg/progress.csv};
  \addlegendentry{HER};
  \addplot table [x=epoch, y=test/ag_g_dist, col sep=comma] {\pathprefix -\envversion-\fwrlsuffix/progress.csv};
  \addlegendentry{FWRL};
  \end{axis}
  \begin{axis}[at={(\nameprefix DistEpoch.south east)},
name=\nameprefix DistRew,
ymin=\distymin,xmin=0,ymax=\distymax,xmax=\succxmax,
  \extrapropsaxisdistrew,
  width=1.0\columnwidth,
  height=\axisheight,
  ytick=\empty,
  legend pos=north east]
  \addplot table [x expr={\thisrow{epoch}*\ncycles*\rolloutB*\T}, y=test/ag_g_dist, col sep=comma] {\pathprefix PR-\envversion-dqst/progress.csv};
  \addlegendentry{Ours};
  \addplot table [x expr={\thisrow{epoch}*\ncycles*\rolloutB*\T + \thisrow{epoch}*\ncycles*\batch}, y=test/ag_g_dist, col sep=comma] {\pathprefix -\envversion-ddpg/progress.csv};
  \addlegendentry{HER};
  \addplot table [x expr={\thisrow{epoch}*\ncycles*\rolloutB*\T + \thisrow{epoch}*\ncycles*\batch}, y=test/ag_g_dist, col sep=comma] {\pathprefix -\envversion-\fwrlsuffix/progress.csv};
  \addlegendentry{FWRL};
\end{axis}
  \begin{axis}[at={($(\nameprefix DistRew.south east) + (30,0)$)},
name=\nameprefix SuccEpoch,
ymin=0,xmin=0,ymax=\succymax,xmax=\distxmax,
  \extrapropsaxissuccepoch,
ylabel=Success Rate (test),
  width=1.0\columnwidth,
  height=\axisheight,
  legend pos=\rewlegendpos]
% these constants can be read from \pathprefix PR-\envversion-dqst/params.json
\def\xcol{epoch}
\def\ycol{test/success_rate}
  \addplot table [x=\xcol, y=\ycol, col sep=comma] {\pathprefix PR-\envversion-dqst/progress.csv};
  \addlegendentry{Ours};
  \addplot table [x=\xcol, y=\ycol, col sep=comma] {\pathprefix -\envversion-ddpg/progress.csv};
  \addlegendentry{HER};
  \addplot table [x=\xcol, y=\ycol, col sep=comma] {\pathprefix -\envversion-\fwrlsuffix/progress.csv};
  \addlegendentry{FWRL};
\end{axis}
  \begin{axis}[at={(\nameprefix SuccEpoch.south east)},
ymin=0,xmin=0,ymax=\succymax,xmax=\succxmax,
name=\nameprefix SuccRew,
  \extrapropsaxissuccrew,
  width=1.0\columnwidth,
  height=\axisheight,
  ytick=\empty,
  legend pos=\rewlegendpos]
\def\xexprPR{\thisrow{epoch}*\ncycles*\rolloutB*\T}
\def\xexprGoalRew{\thisrow{epoch}*\ncycles*\rolloutB*\T + \thisrow{epoch}*\ncycles*\batch}
\def\ycol{test/success_rate}

  \addplot table [x expr=\xexprPR, y=\ycol, col sep=comma] {\pathprefix PR-\envversion-dqst/progress.csv};
  \addlegendentry{Ours};
  \addplot table [x expr=\xexprGoalRew, y=\ycol, col sep=comma] {\pathprefix -\envversion-ddpg/progress.csv};
  \addlegendentry{HER};
  \addplot table [x expr=\xexprGoalRew, y=\ycol, col sep=comma] {\pathprefix -\envversion-\fwrlsuffix/progress.csv};
  \addlegendentry{FWRL};
\end{axis}
    \endgroup
    \node[fit=(HandPenName) (HandPenSuccRew)] (handpen) {}; 
  \end{scope}
\end{tikzpicture}%
%\end{figure}%
%\end{wrapfigure}%
\end{minipage}%
}
  \caption{For the hand tasks, we compare our method (red) against HER (blue) ~\citep{andrychowicz2016learning}
    and FWRL (green) ~\citep{kaelbling1993learning} for the distance-from-goal
    and success rate metrics. Furthermore, both metrics are plotted
    against two progress measures, the number of training epochs and the number of reward
    computations. Measured by distance from the goal, our method performs comparable to or
    better than the baselines for both progress measurements. For the success rate,
    our method underperforms against the baselines. 
}%
  \label{fig:hand-results}%
\end{figure}%

