Reinforcement learning (RL) is an exciting field of research as it allows
for agents to learn complex, autonomous behaviors in a multitude of
environments while requiring minimal supervision only in the form of reward
signals. This is readily evidenced from RL's recent successes from goal-agnostic
activites such as playing Atari games~\cite{MnKaSiNATURE2015} from purely visual
input and defeating world GO~\cite{gibney2016google} and Starcraft champions
(\TODO{really? citation}), to
goal-driven ones such as recent applications in robotic navigation and
manipulation. In the realm of goal-conditioned tasks, this
work introduces Floyd-Warshall Reinforcement Learning (FWRL), a new algorithm
that allows transferring learned behaviours to environments in which the
underlying reward distribution is dynamic.

Algorithms that underly RL are often classified as being either
\emph{model-based} or \emph{model-free}, the distinction being whether
an environment state-transition function is learned explicitly or
implicitly.  In \emph{model-based} RL, the dynamics that govern an
environment's transitions is modelled as separate step for policy
computation. At any point in an episode, agents use this model to
predict future states and utilize this information to maximize possible
reward. This formulation is known to be sample-efficient while normally
not achieving high asymptomatic performance.  Due to the requirement of
planning steps to predict future states, policy computation can be an
expensive process. In contrast, in \emph{model-free} RL, algorithms such
as policy gradients and Q-learning learn the optimal action value
function by maximizing expected future reward for every state-action
pair that the agent perceives.  While highly sample-inefficient, agents
trained under this paradigm have been shown to achieve high asymptomatic
performance in a variety of different problem spheres.

Both paradigms of RL suffer different disadvantages in transferring
learned behaviors to \emph{reward-dynamic} settings i.e.  environments
in which the underlying reward distribution changes.  While model-based
RL allows for the separation of environment dynamics and reward, small
errors in the modelling function lead to significant drops in
performance. In \emph{model-free} RL, on the other hand, the conflated
representation of environment and reward makes any form of transfer
difficult. These problems are exacerbated in multi-goal settings. 

Inspired by the idea of combining advantages from both RL paradigms and
the Floyd-Warshall algorithm in graph-based path planning, this work
introduces Floyd-Warshall Reinforcement Learning, a new framework for
combining model-based and model-free RL in goal-driven reward-dynamic
settings. FWRL works by modeling a goal-conditioned action-value function
where every state in the state space can be a valid goal. This allows
FWRL to remember the paths even if they do not lead to the
goal location during a particular episode.
This motivation is similar to the Hindsight Experience Replay 
\cite{anderson2017vision}, however, we use the parameteric representation
for ``hindsight experience'' instead of the replay memory.

Ideas of combining both RL paradigms to improve agent performance are
not new. In Temporal-Difference modelling \cite{pong2018temporal} 
model a goal conditioned action-value function that is also conditioned
on the number of time-steps. In contrast our proposed value function
is independent of the time-steps.
In Hindsight Experience Replay \cite{andrychowicz2016learning},
previous episode experience is used augment and bootstrap learning. 

Experimentally, FWRL is shown to outperform both model-based, model-free
and above combintations thereof in both a tabular and neural-network
based setting. FWRL is found to outperform the next most significant
baselines by as much x\%.


In summary, this work introduces Floyd-Warshall Reinforcement Learning
outperforming several strong baselines on a varierty of multi-goal
reward-dynamic environments. The experimentation suite and all code is
made available.


%In similar veins and We define the Floyd-Warshall
%function $\fwcost$ for a goal state $\state_j$ starting from a
%state-action pair $\state_i$, $\act_i$ as the expected reward through
%all possible paths $p_{ij}$ between the starting and destination state
%while following policy $\pi_{j}$ to the destination $j$.



%Taking inspiration from the Floyd-Warshall algorithm in graph-based path
%planning domain, we propose Floyd-Warshall reinforcement learning (FWRL)
%algorithm that combines the benefits of both model-based and model-free
%methods.  We define the Floyd-Warshall function $\fwcost$ for a goal
%state $\state_j$ starting from a state-action pair $\state_i$, $\act_i$
%as the expected reward through all possible paths $p_{ij}$ between the
%starting and destination state while following policy $\pi_{j}$ to the
%destination $j$.



%In spite of these limitations, model-free approaches have been applied
%to multi-goal navigation scenarios with considerable success
%\cite{mirowski2018learning}.  However, all these methods depend upon
%exploring the entire space being a candidate goal space.  Not only
%model-free approaches are sample inefficient, multi-goal navigation
%problem exaggerates this problem.


%The algorithms that underly Reinforcement Learning (RL) are often
%broadly discriminated as either \emph{model-based} or \emph{model-free}.
%In model-based RL, the state-transition function that governs an
%environments dynamics is learned explicitly. In contrast, model-free RL
%learns a joint state-action value functions that maximizes expected
%future reward based on the state-action pairs percieved by the agent.
%In learning this function, agents implicity memorize a joint
%representation of the state-transition function and reward distribution
%that governs their environments. 
%
%The advantages of model-based RL are many. Algorithms are known to be
%sample-efficient as compared to model-free algorithms and the decoupled
%learning of the state-transtion functions allows the transfer of learned
%agent behaviors to alternate environment in which only the underlying
%reward distribution changes. These algorithms are disadvantaged by the
%requirement of additional expensive planning steps for online policy
%computation and how small errors in the dynamics can lead to larger
%errors in agent performance. Model-free RL, on the other hand, while
%requiring many more samples to learn, are more asymptotic in their
%performance. However, due to conflated representation of dynamics and
%reward, model-free algorithms struggle to transfer learned behaviors
%when only the reward distribution changes. 
%
%
%Based on this line of research, this work seeks to similarly combine
%advantages of RL, sepcifically in a multi goal navigation setting.
%Taking inspiration from the Floyd-Warshall algorithm in graph-based path
%planning domain, the Floyd-Warshall reinforcement learning (FWRL)
%algorithm is proposed that combines the benefits of both model-based and
%model-free methods. Specifically, ...
%
%This work seeks to combine advantages from both model-based and
%model-free approaches to scenarios involving multi-goal navigation in
%reinforcement learning domains. It i 
%
%specific consideration for multi-goal navigation domains
%
%This is not the first work to propose combinations of two algorithms.
%They are our baselines.
%
%In summary, 



%\emph{Model-based} RL
%
%, algorithms explicitly learn this function
%model allowing for easy separation of dynamics that governs environment
%transitions and the underlyding reward distribution.  Also model-based
%algorithms are known to be sample efficient than model-free algorithms
%especially in cases when the state-transition model is translation
%invariant in the environment.  However, model-based algorithms require
%an additional planning step on the model dynamics which make it
%expensive to compute the policy on the fly.  Although model-based
%algorithms fail to find an optimal policy, because small errors in
%models can add up and lead to wrong prediction over larger state spaces. 
%
%
%In contrast, in \emph{model-free} RL, algorithms such as Q-learning and
%policy gradients attempt to learn a joint optimal state-action value
%function that maximizes expected future reward for all state-action
%pairs percieved by an agent in an environment.  The reward distribution
%and state dynamics that governs the agent's environment are implicitly
%memorized in the learning of this function.  Due to the implicit nature
%of this process, \emph{model-free} RL struggles with transferring
%learned behaviours to environments in which only the underlying reward
%distribution changes.  The model-free approach fails to generalize to
%new rewards in these scenarios, because of the conflated representation
%of environment model and resulting reward.



%Reinforcement learning algorithms are classified on
%the basis of whether the state-transition model is learned explicitly
%into \emph{model-based} and \emph{model-free} algorithms.
%
%%
%The \emph{model-free} algorithms like Q-learning and policy gradients are easier to learn in cases when model is more complex than the policy.
%However, model-free algorithms are harder to transfer in cases when the reward function changes while the state-transition model remains the same.
%
%One example of such a problem is traveling salesman problem.
%Although classical traveling salesman problem is a posed as a planning problem where the space has already been explored, however it is not hard to recognize that someone has to create a map of the map through exploration and sometimes it has to be the agent them-self.
%In such a scenario, the traveling salesman has a dynamic reward with a static map that needs to be
%explored.
%
%The \emph{model-based} algorithms learn the state-transition model explicitly hence making it
%easier to separate the environment transition from the reward distribution. Also model-based
%algorithms are known to be sample efficient than model-free algorithms especially in cases when
%the state-transition model is translation invariant in the environment. 
%However, model-based algorithms require an additional planning step on the model dynamics
%which make it expensive to compute the policy on the fly.
%Although model-based algorithms fail to find an optimal policy, because small errors in models can add up and lead to wrong prediction over larger state spaces. 
%
%Taking inspiration from the Floyd-Warshall algorithm in graph-based path planning domain,
%we propose Floyd-Warshall reinforcement learning (FWRL) algorithm that combines the
%benefits of both model-based and model-free methods.
%We define the Floyd-Warshall function $\fwcost$ for a goal state $\state_j$ starting from a
%state-action pair $\state_i$, $\act_i$
%as the expected reward through all possible paths $p_{ij}$ between the starting and
%destination state while following policy $\pi_{j}$ to the destination $j$.
%%
%\begin{multline}
%\fwcost(\state_j | \state_i, \act_k) \\
%= \E_{p_{ij} \sim \pi_{j}}\left[
%\sum_{\state, \act \in p_{ij} } \discount^{t} r(\state, \act) \right| p_{ij} = (\state_i, \act_k, \dots, \state_J) \Biggr] \, .
%\end{multline}
%%
%
%Due to this formulation, the optimal Floyd-Warshall function should satisfy the transitive property
%\begin{align}
%\fwcost^*(\state_i, \act_i, \state_j) = \max_{\state_k} \left[
%\fwcost^*(\state_i, \act_i, \state_k)
%+ \max_{\act_k}\fwcost^*(\state_k, \act_k, \state_j) \right]
%\label{eq:transitive-fw}
%\end{align}
%
%The advantage of this formulation is that as soon as the goal state $\state_g$, the Q-function and
%hence the policy can be estimated from $Q(\state, \act) = F(\state, \act, \state_g) + \Rew(\state_g)$.
%Even if the path between two states has not been ever traversed, the value function
%can be computed using the transitive property in Eq~\ref{eq:transitive-fw}. 
%
%In our experiments, we show how our approach is better than both model-free Q-learning and model-based approaches. We also show how our approach can be switch between either approaches depending upon traversal experience in the state space.
