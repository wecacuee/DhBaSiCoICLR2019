\section{Background}

An RL problem is formalized as an Markov Decision Process (MDP)
\citep{sutton1998reinforcement}. A MDP governs
an evolving sequence of state-action-reward triples $[(\state_0, \act_0,
\rew_0), \dots, (\state_T, \act_T, \rew_T)]$, that is full governed
by the five tuple $(\State, \Action, \Trans, \Rew, \discount)$, where $\State$ is the
state space, $\Action$ is the action space, $\Trans : \State \times \Action
\rightarrow \State$ is the system dynamics, $\Rew : \State \times \Action
\rightarrow \R $ is the reward function and $\discount$ is the discount
factor.
In a typical RL problem the transition function $\Trans$ is not given but is
known to be static.
In RL the objective is to find a policy $\policy(\state): \State
\rightarrow \Action$ that
maximizes the expected cumulative
reward over time, $R_t = \sum_{k=t}^{\infty} \discount^{k-t}\rew_{k}$, called the
\emph{return}. $\discount < 1$, is the discount factor that forces
convergence of the return for long horizon problems. 
Reinforcement learning is typically formulated in single goal
contexts. More recently there has been interest in multi-goal
problems
\citep{andrychowicz2017hindsight,pong2018temporal,plappert2018multi}.


\subsection{Single-goal Reinforcement learning}

A number of reinforcement learning algorithms use parameteric function
approximators to estimate returns, either in the form of a value function
$\Value(\state; \param)$ or an action-value function $\Q(\state, \act; \param)$.
Here we use the action-value function which is the return for
state-action pairs.  
%
\begin{align}
\Q_\policy(\state, \act; \param_\Q) = \E_\policy\left[ \sum_{k=t}^T
  \discount^{k-t} \Rew(\state_k, \act_k)
  \middle| \state_t = \state, \act_t = \act \right] .
  \label{eq:q-def}
\end{align}%
%
When the policy, $\policy$, is optimal, the $\Q$-function satisfies the
\emph{Bellman equation}:
%
\begin{align}
    \Q_{\pi_*}(\state_t, \act_t; \param_\Q)
  &=
    \begin{cases}
        \Rew(\state_t, \act_t) + \discount \max_{\act \in \Action}
        \Q_{\pi_*}(\state_{t+1}, \act; \param_\Q)
      & \text{ if } t < T
      \\
      \Rew(\state_T, \act_T) & \text{ if } t = T
    \end{cases}.
  \label{eq:bellman}
\end{align}%
% 
Once $\Q_{\pi_*}$-function is approximated, the optimal policy is
computed greediliy, $\policy_*(\state_t) = \arg \max_{\act \in \Action} \Q_*(\state_t, \act)$.

In DQNs \cite{mnih2013playing}, the Bellman equation is used to
formulate a loss function over which gradient descent approximates the
$\Q_{\pi_*}$.
%
%Recent advances in deep reinforcement learning~\citep{MnKaSiNATURE2015} view the
%Bellman equation as the gradient of a loss function. For example, Deep
%Q-Networks minimize the loss function whose gradient is the Bellman equation:
%
\begin{align}
  \Loss(\param_{\Q_m}, \param_{\policy_m}) =
    \E_{a_t\sim\policy(\state_t; \param_{\policy_m})}\left[\left(
  \Q_m(\state_t, \act_t; \param_{\Q_m}) -
  y_t  \right)^2\right],
  \label{eq:bellman-dqn}
\end{align}
where 
\begin{equation}
    y_t = \Rew(\state_t, \act_t) + \max_{\act}\discount \Q_t(\state_{t+1}, \act;
    \param_{\Q_t}) \right)^2\right].
\end{equation}
%
$y_t$ is the \emph{target} value.
$\Q_m$ is the main network and $\Q_t$ is the target network. Target and main
networks are function approximators with same structure, but different
parameters. The target network parameters are changed slowly towards the main
network parameters either by periodically copying the main network parameters to
target network or by maintaining target network as exponential moving average of
the main network. This improves the stability of the learning.

% \citet{MnKaSiNATURE2015} also introduced the idea of \emph{replay buffer}.
% The transitions experienced are not used to update the neural network in order.
% Instead they are stored in a replay buffer which is then sampled out of order
% for independent transitions and updating the neural network.

In this paper, we use a variation of
Deep deterministic policy-gradients (DDPG)~\citep{lillicrap2015continuous} which
extends DQN~\citep{MnKaSiNATURE2015} to continuous action domains by introducing
a parameterized policy function $\policy(\state; \param_\policy)$, which is
learned through policy gradients.

% Assuming the policy to be optimal, the maxima of the second term in the Bellman
% equation the $\Q$-value at action chosen by the policy at that state:
% %
% \begin{align}
%   \Loss(\param_{\Q_m}, \param_{\policy_m}) = \E_{a_t\sim\policy(\state_t; \param_{\policy_m})}\left[
%   \Q^*_m(\state_t, \act_t; \param_{\Q_m}) -
%   \Rew(\state_t, \act_t) -
%   \discount \Q^*_t(\state_{t+1}, \policy_t(\state_{t+1}; \param_{\policy_t}); \param_{\Q_t}) \right]^2.
%   \label{eq:bellman-ddpg}
% \end{align}
% %
% The policy network is updated using policy gradients
% %
% \begin{align}
% \nabla_{\param_{\policy_m}} \Loss \propto \frac{1}{N} \sum_t \nabla_\act \Q^*(\state_t, \act; \param_{\Q_m})\nabla_{\param_{\policy_m}} \policy(\state_t; \param_{\policy_m}).
% \end{align}%
% % 


\subsection{Multi-goal Reinforcement learning}
Multi-goal tasks need the specification of goal state which can change for each
trial~\citep{plappert201802multigoalrl}. The examples include navigation to a
goal location, or moving a robotic arm so that the tip of the arm is at a
particular 3D location (Fetch-Reach task).

The recent state of the art MGRL algorithms learn a goal-conditioned value
function (GCVF), $\fwargs{\state}{\act}{\goal}{}{}$, which is defined similar to the
$\Q$-function in Eq~\ref{eq:q-def}, but with an additional dependence over the
desired goal specification $\goal \in \Goal$ :
%
\begin{align}
\fwargs\state\act\goal\policy{} = \E_\policy\left[ \sum_{k=t}^T
  \discount^{k-t} \Rew(\state_k, \act_k, \goal)
  \middle| \state_t = \state, \act_t = \act \right] .
  \label{eq:q-def}
\end{align}%
%
The goal specification $\goal \in \Goal$ can be arbitrary. To ease the learning
of correspondence between the state and the ``achieved goals'', they are assumed
to be an observable part of Goal-MDP $[(\state_0, \goal_0, \act_0, \rew_0), \dots,
(\state_T, \goal_T, \act_T, \rew_T)]$.

Also note the dependence of reward function $\Rew(\state, \act, \goal)$ on goal
specification $\goal$.

\subsubsection{Hindsight Experience Replay}
Hindsight Experience Replay (HER) builds upon this definition of GCVF
and and the insight that the goal locations can be very sparse in the
state-space and there is no valuable feedback from the environment unless the
agent reaches the goal.
HER solves this problem by learning through episodes when the agent fails to
reach the goal. This is implemented by re-sampling the trajectories from failed
experiences and recomputing rewards on trajectories, as if the achieved
goals were the desired pseudo-goals.

In our experiments, we employ the ``future'' strategy for pseudo-goal sampling
as described in the paper. More specifically two transitions $t$ and $t+f$ are
sampled from the same episode in the replay memory. The goal for the later
transition $\goal_{t+f}$ is assumed to the pseudo-goal for the first transition.
A new transition is generated for the time step $t$ with reward re-computed
with $\goal_{t+f}$ as the desired goal, $(\state_{t}, \act_t, \state_{t+1},
\Rew(\state_t, \act_t, \goal_{t+f})$.


\subsubsection{Goal conditioned value function as path rewards}
We define the goal-condition value function as expected path-rewards of reaching
the goal from a given start state:
%
\begin{align}
  \fwargs\state\act{\goal^*}\policy{P}
  &=
    \begin{cases}
      \E_{l \in [t+1, T], \policy}\left[ \sum_{k=t}^{l-1}
  \discount^{k-t} \Rew^P(\state_k, \act_k)
  \middle| \state, \act, \goal_l = \goal^* \right]
& \text{ if } \exists l \text{ such that } \goal_l = \goal^*
\\
-\infty & \text{ otherwise }
\end{cases}.
  \label{eq:qp-def}
\end{align}%
% 
The main difference between $\fw^P$~\eqref{eq:qp-def} and $\fw$~\eqref{eq:q-def}
is that $\fw^P$ explicitly depends upon reaching the goal within the episode,
thereby removing the dependence of reward on the goal. Surprisingly, the Bellman
Equation can still be applied to the path-value function $\fw^P$ with a slightly
different terminal condition:
%
\begin{align}
  \fwargs{\state_t}{\act_t}{\goal^*}*P
  &=
    \begin{cases}
      \Rew^P(\state_t, \act_t) + \discount \max_{\act \in \Action} \fwargs{\state_{t+1}}\act{\goal^*}*P
      & \text{ if } t < l-1 \text{ such that } \goal_l = \goal^*
      \\
      \Rew^P(\state_{l-1}, \act_{l-1}) & \text{ if } t = l-1 \text{ such that } \goal_l = \goal^*
    \end{cases}.
  \label{eq:bellman-path}
\end{align}%
% 
This formulation has several advantages. Firstly, it removes a design decision
about the choice of goal-reward. Secondly, it enables HER to work without reward
re-computation. This can be especially useful when reward computation is expensive.

Because we work with path rewards, the reward function design should
make sure that there are no positive cycles in the environment. In another
words, cycling back to the same state via another state should not yield
positive rewards, otherwise the agent can extract unlimited reward just by
going in circles. Hence a typical environment will have most of the reward
values as negative. In all our experiments with path rewards, we use a constant
value of -1 for all states, $\Rew^P(\state, \act) = -1 \, \forall \state, \act$.


% % Use less math
% At the start of the task, a state space $\State$ (example: joint angles of the
% arm), an action space $\Action$ (example: keypresses for discrete actions, joint
% torques for continuous actions), and a
% Goal space $\Goal$ (example: 3D coordinates of the destination location).
% It is known that the transition function $\TransFull$
% and the reward function $\Rew: \State \times \Action \times \Goal \rightarrow
% \R$ are static.
% At the start of each episode, a goal state $\goal^* \in \Goal$ is given. At each
% step $t$ the agent can chose an action to take $\act_t$ and can observe
% $(\state_t, \goal_t, \rew_t)$ where $\goal_t$ is the achieved goal which can be
% deterministically computed from state $\goal_t = f_\goal(\state_t)$.
% The reward function is formulated such that reward for reaching at the goal
% $\|\goal_t - \goal^*\|_2 < \epsilon$ once
% is higher than reward for visiting any other state $\infty$-times
% $\Rew(\state_g, \goal^*, \act) \ge
% \frac{1}{1-\discount}\max_{\state, \act'}\Rew(\state, \goal^*,
% \act') \forall \act$, where $\state_g$ is such that $\|f_\goal(\state_g) -
% \goal^*\| < \epsilon$.
