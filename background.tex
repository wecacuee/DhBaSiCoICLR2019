\section{Background}

\subsection{Floyd-Warshall Algorithm for Path-Planning}

Consider an arbitrary weighted graph, $G = (V, E)$, with both positive and
negative edges. Note that the algorithm does not work if negative cycles
are present. The Floyd-Warshall algorithm is a mechanism for finding the
shortest distances from every node to every other node in the graph. 


\subsection{Q-Learning}
Q-learning is an off-policy model-free method used to learn the optimal
behaviors in any environment. 

\begin{equation}
    R_t = \sum_{t^{'}=t}^{T} \gamma^{t^{'}-t} r_{t^{'}}
\end{equation}

\begin{equation}
    Q^*(s, a) = \max_{\pi} E[R_t \mid s_t=s, a_t=a, \pi]
\end{equation}


\noindent
Which derives from the Bellman equation.
\begin{equation}
    Q^*(s, a) = E[r + \gamma \max_{a^{'}} Q^*(s^{'}, a^{'}) \mid s, a]
\end{equation}

\subsection{Floyd-Warshall Algorithm}
\newcommand{\vertices}{\mathcal{V}}
\newcommand{\wts}{W}
% Floyd warshall data structure
\newcommand{\fwds}{D}

Floyd-Warshall algorithm~\citep{floydwarshall1962} is a shortest path finding
algorithm from any vertex in a graph to any other vertex in the graph.
For a weighted graph $G = (\vertices, \edges)$, Floyd-Warshall algorithm
finds the shortest path by keeping maintaining a shortest distance
data-structure $\fwds : \vertices \times \vertices \rightarrow \R$.
This data-structure represents the shortest known distance between any
two pair of vertices $i, j \in \vertices$.
The data-structure $\fwds$ is initialized with edges costs
$\fwds[i, j] \leftarrow \edges_{(i, j)} \, \forall (i, j) \in \edges$
and the uninitialized edges are assigned a high value
$\fwds[i, j] \leftarrow \infty \, \forall i, j \in \vertices$.
The algorithm works by sequentially observing all the nodes in the graph and
updating $\fwds$ as with the shortest path known so far:
%
\begin{align}
  \fwds[i, j] \leftarrow \min\{ \fwds[i, j], \fwds[i, k] + \fwds[k, j] \} \quad
  \forall i, j, k \in \vertices \, .
\end{align}%
%

\subsection{Goal conditioned value functions}

