\begin{figure}
\scalebox{0.5}{%
\begin{minipage}[t]{0.55\linewidth}
\begin{tikzpicture}
  % these constants can be read from \pathprefix */params.json
  \def\T{50}
  \def\ncycles{50}
  \def\rolloutB{2}
  \def\batch{256}
  \def\extrapropsaxisdistepoch{xlabel=Epochs}
  \def\extrapropsaxissuccepoch{xlabel=Epochs},
  \def\extrapropsaxisdistrew{xlabel=Reward Computations},
  \def\extrapropsaxissuccrew{xlabel=Reward Computations},
  \def\fwrlsuffix{fwrl}
  \def\envversion{v1}
  \def\distymin{0}
  \def\scopexshift{0.05\columnwidth}
  \def\scopeyshift{-0.3\columnwidth}
  \begin{scope}[local bounding box=fetchreach]
    %\def\pathprefix{media/res/6efc1de-path_reward_low_thresh_chosen-FetchReach}
\def\distymax{0.25}
  \begin{axis}[ymin=0,xmin=0, ymax=\distymax,xmax=16,
  name=FetchReachDistEpoch,
  xlabel=Epochs,
  width=1.0\columnwidth,
  height=0.70\columnwidth,
  ylabel=Distance from goal (m),
  legend pos=north east]
  \addplot table [x=epoch, y=test/ag_g_dist, col sep=comma] {\pathprefix PR-v1-dqst/progress.csv};
  \addlegendentry{Ours};
  \addplot table [x=epoch, y=test/ag_g_dist, col sep=comma] {\pathprefix -v1-ddpg/progress.csv};
  \addlegendentry{HER};
  \addplot table [x=epoch, y=test/ag_g_dist, col sep=comma] {\pathprefix -v1-fwrl/progress.csv};
  \addlegendentry{FWRL};
  \end{axis}
  \begin{axis}[at={(FetchReachDistEpoch.south east)},
name=FetchReachDistRew,
ymin=0,xmin=0,ymax=\distymax,xmax=5e4,
  xlabel=Reward Computations,
  width=1.0\columnwidth,
  height=0.70\columnwidth,
  ytick=\empty,
  legend pos=north east]
% these constants can be read from \pathprefix */params.json
\def\T{50}
\def\ncycles{10}
\def\rolloutB{2}
\def\batch{256}
  \addplot table [x expr={\thisrow{epoch}*\ncycles*\rolloutB*\T}, y=test/ag_g_dist, col sep=comma] {\pathprefix PR-v1-dqst/progress.csv};
  \addlegendentry{Ours};
  \addplot table [x expr={\thisrow{epoch}*\ncycles*\rolloutB*\T + \thisrow{epoch}*\ncycles*\batch}, y=test/ag_g_dist, col sep=comma] {\pathprefix -v1-ddpg/progress.csv};
  \addlegendentry{HER};
  \addplot table [x expr={\thisrow{epoch}*\ncycles*\rolloutB*\T + \thisrow{epoch}*\ncycles*\batch}, y=test/ag_g_dist, col sep=comma] {\pathprefix -v1-fwrl/progress.csv};
  \addlegendentry{FWRL};
\end{axis}
  \begin{axis}[at={($(FetchReachDistRew.south east) + (30,0)$)},
name=FetchReachSuccEpoch,
ymin=0,xmin=0,ymax=1,xmax=16,
  xlabel=Epoch,
ylabel=Success Rate (test),
  width=1.0\columnwidth,
  height=0.70\columnwidth,
  legend pos=south east]
% these constants can be read from \pathprefix PR-v1-dqst/params.json
\def\xcol{epoch}
\def\ycol{test/success_rate}
  \addplot table [x=\xcol, y=\ycol, col sep=comma] {\pathprefix PR-v1-dqst/progress.csv};
  \addlegendentry{Ours};
  \addplot table [x=\xcol, y=\ycol, col sep=comma] {\pathprefix -v1-ddpg/progress.csv};
  \addlegendentry{HER};
  \addplot table [x=\xcol, y=\ycol, col sep=comma] {\pathprefix -v1-fwrl/progress.csv};
  \addlegendentry{FWRL};
\end{axis}
  \begin{axis}[at={(FetchReachSuccEpoch.south east)},
    ymin=0,xmin=0,ymax=1,xmax=5e4,
    name=FetchReachSuccRew,
  xlabel=Reward Computations,
  width=1.0\columnwidth,
  height=0.70\columnwidth,
  ytick=\empty,
  legend pos=south east]
% these constants can be read from \pathprefix PR-v1-dqst/params.json
\def\T{50}
\def\ncycles{10}
\def\rolloutB{2}
\def\batch{256}
\def\xexprPR{\thisrow{epoch}*\ncycles*\rolloutB*\T}
\def\xexprGoalRew{\thisrow{epoch}*\ncycles*\rolloutB*\T + \thisrow{epoch}*\ncycles*\batch}
\def\ycol{test/success_rate}

  \addplot table [x expr=\xexprPR, y=\ycol, col sep=comma] {\pathprefix PR-v1-dqst/progress.csv};
  \addlegendentry{Ours};
  \addplot table [x expr=\xexprGoalRew, y=\ycol, col sep=comma] {\pathprefix -v1-ddpg/progress.csv};
  \addlegendentry{HER};
  \addplot table [x expr=\xexprGoalRew, y=\ycol, col sep=comma] {\pathprefix -v1-fwrl/progress.csv};
  \addlegendentry{FWRL};
\end{axis}
    \begingroup
    \def\distxmax{16}
    \def\succxmax{5e4}
    \def\succymax{1}
    \def\rewlegendpos{south east}
    \def\ncycles{10}
    \def\nameprefix{FetchReach}
    \def\fullname{Fetch Reach}
    \def\distymax{0.25}
    \def\pathprefix{media/res/6efc1de-path_reward_low_thresh_chosen-FetchReach}
    \def\axisheight{0.6\columnwidth}
\node [rotate=90,align=center,text width=0.4\columnwidth]
(\nameprefix Name) {\color{blue}\scriptsize \fullname};
\begin{axis}[at={($(\nameprefix Name.south west) + (30, 0)$)},
  ymin=\distymin,xmin=0, ymax=\distymax,xmax=\distxmax,
  name=\nameprefix DistEpoch,
  width=1.0\columnwidth,
  height=\axisheight,
  ylabel=Distance from goal (m),
  legend pos=north east,
  \extrapropsaxisdistepoch,
  ]
  \addplot table [x=epoch, y=test/ag_g_dist, col sep=comma] {\pathprefix PR-\envversion-dqst/progress.csv};
  \addlegendentry{Ours};
  \addplot table [x=epoch, y=test/ag_g_dist, col sep=comma] {\pathprefix -\envversion-ddpg/progress.csv};
  \addlegendentry{HER};
  \addplot table [x=epoch, y=test/ag_g_dist, col sep=comma] {\pathprefix -\envversion-\fwrlsuffix/progress.csv};
  \addlegendentry{FWRL};
  \end{axis}
  \begin{axis}[at={(\nameprefix DistEpoch.south east)},
name=\nameprefix DistRew,
ymin=\distymin,xmin=0,ymax=\distymax,xmax=\succxmax,
  \extrapropsaxisdistrew,
  width=1.0\columnwidth,
  height=\axisheight,
  ytick=\empty,
  legend pos=north east]
  \addplot table [x expr={\thisrow{epoch}*\ncycles*\rolloutB*\T}, y=test/ag_g_dist, col sep=comma] {\pathprefix PR-\envversion-dqst/progress.csv};
  \addlegendentry{Ours};
  \addplot table [x expr={\thisrow{epoch}*\ncycles*\rolloutB*\T + \thisrow{epoch}*\ncycles*\batch}, y=test/ag_g_dist, col sep=comma] {\pathprefix -\envversion-ddpg/progress.csv};
  \addlegendentry{HER};
  \addplot table [x expr={\thisrow{epoch}*\ncycles*\rolloutB*\T + \thisrow{epoch}*\ncycles*\batch}, y=test/ag_g_dist, col sep=comma] {\pathprefix -\envversion-\fwrlsuffix/progress.csv};
  \addlegendentry{FWRL};
\end{axis}
  \begin{axis}[at={($(\nameprefix DistRew.south east) + (30,0)$)},
name=\nameprefix SuccEpoch,
ymin=0,xmin=0,ymax=\succymax,xmax=\distxmax,
  \extrapropsaxissuccepoch,
ylabel=Success Rate (test),
  width=1.0\columnwidth,
  height=\axisheight,
  legend pos=\rewlegendpos]
% these constants can be read from \pathprefix PR-\envversion-dqst/params.json
\def\xcol{epoch}
\def\ycol{test/success_rate}
  \addplot table [x=\xcol, y=\ycol, col sep=comma] {\pathprefix PR-\envversion-dqst/progress.csv};
  \addlegendentry{Ours};
  \addplot table [x=\xcol, y=\ycol, col sep=comma] {\pathprefix -\envversion-ddpg/progress.csv};
  \addlegendentry{HER};
  \addplot table [x=\xcol, y=\ycol, col sep=comma] {\pathprefix -\envversion-\fwrlsuffix/progress.csv};
  \addlegendentry{FWRL};
\end{axis}
  \begin{axis}[at={(\nameprefix SuccEpoch.south east)},
ymin=0,xmin=0,ymax=\succymax,xmax=\succxmax,
name=\nameprefix SuccRew,
  \extrapropsaxissuccrew,
  width=1.0\columnwidth,
  height=\axisheight,
  ytick=\empty,
  legend pos=\rewlegendpos]
\def\xexprPR{\thisrow{epoch}*\ncycles*\rolloutB*\T}
\def\xexprGoalRew{\thisrow{epoch}*\ncycles*\rolloutB*\T + \thisrow{epoch}*\ncycles*\batch}
\def\ycol{test/success_rate}

  \addplot table [x expr=\xexprPR, y=\ycol, col sep=comma] {\pathprefix PR-\envversion-dqst/progress.csv};
  \addlegendentry{Ours};
  \addplot table [x expr=\xexprGoalRew, y=\ycol, col sep=comma] {\pathprefix -\envversion-ddpg/progress.csv};
  \addlegendentry{HER};
  \addplot table [x expr=\xexprGoalRew, y=\ycol, col sep=comma] {\pathprefix -\envversion-\fwrlsuffix/progress.csv};
  \addlegendentry{FWRL};
\end{axis}
    \endgroup
    \node[fit=(FetchReachName) (FetchReachSuccRew)] (fetchreach) {}; 
  \end{scope}
  \begin{scope}[shift={($(fetchreach.south west)+(0.05\columnwidth, \scopeyshift)$)},
    local bounding box=fetchpush]
    %\def\pathprefix{media/res/6efc1de-path_reward_low_thresh_chosen-FetchPush}
\def\distymax{0.25}
  \begin{axis}[ymin=0,xmin=0, ymax=\distymax,xmax=21,
  name=FetchPushDistEpoch,
  xlabel=Epochs,
  width=1.0\columnwidth,
  height=0.70\columnwidth,
  ylabel=Distance from goal (m),
  legend pos=north east]
  \addplot table [x=epoch, y=test/ag_g_dist, col sep=comma] {\pathprefix PR-v1-dqst/progress.csv};
  \addlegendentry{Ours};
  \addplot table [x=epoch, y=test/ag_g_dist, col sep=comma] {\pathprefix -v1-ddpg/progress.csv};
  \addlegendentry{HER};
  \addplot table [x=epoch, y=test/ag_g_dist, col sep=comma] {\pathprefix -v1-fwrl/progress.csv};
  \addlegendentry{FWRL};
  \end{axis}
  \begin{axis}[at={(FetchPushDistEpoch.south east)},
name=FetchPushDistRew,
ymin=0,xmin=0,ymax=\distymax,xmax=6e4,
  xlabel=Reward Computations,
  width=1.0\columnwidth,
  height=0.70\columnwidth,
  ytick=\empty,
  legend pos=north east]
% these constants can be read from \pathprefix */params.json
\def\T{50}
\def\ncycles{10}
\def\rolloutB{2}
\def\batch{256}
  \addplot table [x expr={\thisrow{epoch}*\ncycles*\rolloutB*\T}, y=test/ag_g_dist, col sep=comma] {\pathprefix PR-v1-dqst/progress.csv};
  \addlegendentry{Ours};
  \addplot table [x expr={\thisrow{epoch}*\ncycles*\rolloutB*\T + \thisrow{epoch}*\ncycles*\batch}, y=test/ag_g_dist, col sep=comma] {\pathprefix -v1-ddpg/progress.csv};
  \addlegendentry{HER};
  \addplot table [x expr={\thisrow{epoch}*\ncycles*\rolloutB*\T + \thisrow{epoch}*\ncycles*\batch}, y=test/ag_g_dist, col sep=comma] {\pathprefix -v1-fwrl/progress.csv};
  \addlegendentry{FWRL};
\end{axis}
  \begin{axis}[at={($(FetchPushDistRew.south east) + (30,0)$)},
name=FetchPushSuccEpoch,
ymin=0,xmin=0,ymax=1,xmax=21,
  xlabel=Epoch,
ylabel=Success Rate (test),
  width=1.0\columnwidth,
  height=0.70\columnwidth,
  legend pos=south east]
% these constants can be read from \pathprefix PR-v1-dqst/params.json
\def\xcol{epoch}
\def\ycol{test/success_rate}
  \addplot table [x=\xcol, y=\ycol, col sep=comma] {\pathprefix PR-v1-dqst/progress.csv};
  \addlegendentry{Ours};
  \addplot table [x=\xcol, y=\ycol, col sep=comma] {\pathprefix -v1-ddpg/progress.csv};
  \addlegendentry{HER};
  \addplot table [x=\xcol, y=\ycol, col sep=comma] {\pathprefix -v1-fwrl/progress.csv};
  \addlegendentry{FWRL};
\end{axis}
  \begin{axis}[at={(FetchPushSuccEpoch.south east)},
ymin=0,xmin=0,ymax=1,xmax=6e4,
name=FetchPushSuccRew,
  xlabel=Reward Computations,
  width=1.0\columnwidth,
  height=0.70\columnwidth,
  ytick=\empty,
  legend pos=south east]
% these constants can be read from \pathprefix PR-v1-dqst/params.json
\def\T{50}
\def\ncycles{10}
\def\rolloutB{2}
\def\batch{256}
\def\xexprPR{\thisrow{epoch}*\ncycles*\rolloutB*\T}
\def\xexprGoalRew{\thisrow{epoch}*\ncycles*\rolloutB*\T + \thisrow{epoch}*\ncycles*\batch}
\def\ycol{test/success_rate}

  \addplot table [x expr=\xexprPR, y=\ycol, col sep=comma] {\pathprefix PR-v1-dqst/progress.csv};
  \addlegendentry{Ours};
  \addplot table [x expr=\xexprGoalRew, y=\ycol, col sep=comma] {\pathprefix -v1-ddpg/progress.csv};
  \addlegendentry{HER};
  \addplot table [x expr=\xexprGoalRew, y=\ycol, col sep=comma] {\pathprefix -v1-fwrl/progress.csv};
  \addlegendentry{FWRL};
\end{axis}
LaTeX/MPS finished at Wed Nov  7 18:01:26

    \begingroup
    \def\rewlegendpos{south east}
    \def\distxmax{21}
    \def\succxmax{30e4}
    \def\succymax{1}
    \def\nameprefix{FetchPush}
    \def\fullname{Fetch Push}
    \def\distymax{0.25}
    \def\pathprefix{media/res/6efc1de-path_reward_low_thresh_chosen-FetchPush}
    \def\axisheight{0.6\columnwidth}
\node [rotate=90,align=center,text width=0.4\columnwidth]
(\nameprefix Name) {\color{blue}\scriptsize \fullname};
\begin{axis}[at={($(\nameprefix Name.south west) + (30, 0)$)},
  ymin=\distymin,xmin=0, ymax=\distymax,xmax=\distxmax,
  name=\nameprefix DistEpoch,
  width=1.0\columnwidth,
  height=\axisheight,
  ylabel=Distance from goal (m),
  legend pos=north east,
  \extrapropsaxisdistepoch,
  ]
  \addplot table [x=epoch, y=test/ag_g_dist, col sep=comma] {\pathprefix PR-\envversion-dqst/progress.csv};
  \addlegendentry{Ours};
  \addplot table [x=epoch, y=test/ag_g_dist, col sep=comma] {\pathprefix -\envversion-ddpg/progress.csv};
  \addlegendentry{HER};
  \addplot table [x=epoch, y=test/ag_g_dist, col sep=comma] {\pathprefix -\envversion-\fwrlsuffix/progress.csv};
  \addlegendentry{FWRL};
  \end{axis}
  \begin{axis}[at={(\nameprefix DistEpoch.south east)},
name=\nameprefix DistRew,
ymin=\distymin,xmin=0,ymax=\distymax,xmax=\succxmax,
  \extrapropsaxisdistrew,
  width=1.0\columnwidth,
  height=\axisheight,
  ytick=\empty,
  legend pos=north east]
  \addplot table [x expr={\thisrow{epoch}*\ncycles*\rolloutB*\T}, y=test/ag_g_dist, col sep=comma] {\pathprefix PR-\envversion-dqst/progress.csv};
  \addlegendentry{Ours};
  \addplot table [x expr={\thisrow{epoch}*\ncycles*\rolloutB*\T + \thisrow{epoch}*\ncycles*\batch}, y=test/ag_g_dist, col sep=comma] {\pathprefix -\envversion-ddpg/progress.csv};
  \addlegendentry{HER};
  \addplot table [x expr={\thisrow{epoch}*\ncycles*\rolloutB*\T + \thisrow{epoch}*\ncycles*\batch}, y=test/ag_g_dist, col sep=comma] {\pathprefix -\envversion-\fwrlsuffix/progress.csv};
  \addlegendentry{FWRL};
\end{axis}
  \begin{axis}[at={($(\nameprefix DistRew.south east) + (30,0)$)},
name=\nameprefix SuccEpoch,
ymin=0,xmin=0,ymax=\succymax,xmax=\distxmax,
  \extrapropsaxissuccepoch,
ylabel=Success Rate (test),
  width=1.0\columnwidth,
  height=\axisheight,
  legend pos=\rewlegendpos]
% these constants can be read from \pathprefix PR-\envversion-dqst/params.json
\def\xcol{epoch}
\def\ycol{test/success_rate}
  \addplot table [x=\xcol, y=\ycol, col sep=comma] {\pathprefix PR-\envversion-dqst/progress.csv};
  \addlegendentry{Ours};
  \addplot table [x=\xcol, y=\ycol, col sep=comma] {\pathprefix -\envversion-ddpg/progress.csv};
  \addlegendentry{HER};
  \addplot table [x=\xcol, y=\ycol, col sep=comma] {\pathprefix -\envversion-\fwrlsuffix/progress.csv};
  \addlegendentry{FWRL};
\end{axis}
  \begin{axis}[at={(\nameprefix SuccEpoch.south east)},
ymin=0,xmin=0,ymax=\succymax,xmax=\succxmax,
name=\nameprefix SuccRew,
  \extrapropsaxissuccrew,
  width=1.0\columnwidth,
  height=\axisheight,
  ytick=\empty,
  legend pos=\rewlegendpos]
\def\xexprPR{\thisrow{epoch}*\ncycles*\rolloutB*\T}
\def\xexprGoalRew{\thisrow{epoch}*\ncycles*\rolloutB*\T + \thisrow{epoch}*\ncycles*\batch}
\def\ycol{test/success_rate}

  \addplot table [x expr=\xexprPR, y=\ycol, col sep=comma] {\pathprefix PR-\envversion-dqst/progress.csv};
  \addlegendentry{Ours};
  \addplot table [x expr=\xexprGoalRew, y=\ycol, col sep=comma] {\pathprefix -\envversion-ddpg/progress.csv};
  \addlegendentry{HER};
  \addplot table [x expr=\xexprGoalRew, y=\ycol, col sep=comma] {\pathprefix -\envversion-\fwrlsuffix/progress.csv};
  \addlegendentry{FWRL};
\end{axis}
    \endgroup
    \node[fit=(FetchPushName) (FetchPushSuccRew)] (fetchpush) {}; 
  \end{scope}
  \begin{scope}[shift={($(fetchpush.south west)+(0.05\columnwidth, \scopeyshift)$)},
    local bounding box=fetchpickandplace]
    \begingroup
    \def\rewlegendpos{south east}
    \def\distxmax{33}
    \def\distymax{0.25}
    \def\succxmax{50e4}
    \def\succymax{1}
    \def\nameprefix{FetchPickAndPlace}
    \def\fullname{Fetch Pick And Place}
    \def\pathprefix{media/res/6efc1de-path_reward_low_thresh_chosen-FetchPickAndPlace}
    \def\axisheight{0.6\columnwidth}
\node [rotate=90,align=center,text width=0.4\columnwidth]
(\nameprefix Name) {\color{blue}\scriptsize \fullname};
\begin{axis}[at={($(\nameprefix Name.south west) + (30, 0)$)},
  ymin=\distymin,xmin=0, ymax=\distymax,xmax=\distxmax,
  name=\nameprefix DistEpoch,
  width=1.0\columnwidth,
  height=\axisheight,
  ylabel=Distance from goal (m),
  legend pos=north east,
  \extrapropsaxisdistepoch,
  ]
  \addplot table [x=epoch, y=test/ag_g_dist, col sep=comma] {\pathprefix PR-\envversion-dqst/progress.csv};
  \addlegendentry{Ours};
  \addplot table [x=epoch, y=test/ag_g_dist, col sep=comma] {\pathprefix -\envversion-ddpg/progress.csv};
  \addlegendentry{HER};
  \addplot table [x=epoch, y=test/ag_g_dist, col sep=comma] {\pathprefix -\envversion-\fwrlsuffix/progress.csv};
  \addlegendentry{FWRL};
  \end{axis}
  \begin{axis}[at={(\nameprefix DistEpoch.south east)},
name=\nameprefix DistRew,
ymin=\distymin,xmin=0,ymax=\distymax,xmax=\succxmax,
  \extrapropsaxisdistrew,
  width=1.0\columnwidth,
  height=\axisheight,
  ytick=\empty,
  legend pos=north east]
  \addplot table [x expr={\thisrow{epoch}*\ncycles*\rolloutB*\T}, y=test/ag_g_dist, col sep=comma] {\pathprefix PR-\envversion-dqst/progress.csv};
  \addlegendentry{Ours};
  \addplot table [x expr={\thisrow{epoch}*\ncycles*\rolloutB*\T + \thisrow{epoch}*\ncycles*\batch}, y=test/ag_g_dist, col sep=comma] {\pathprefix -\envversion-ddpg/progress.csv};
  \addlegendentry{HER};
  \addplot table [x expr={\thisrow{epoch}*\ncycles*\rolloutB*\T + \thisrow{epoch}*\ncycles*\batch}, y=test/ag_g_dist, col sep=comma] {\pathprefix -\envversion-\fwrlsuffix/progress.csv};
  \addlegendentry{FWRL};
\end{axis}
  \begin{axis}[at={($(\nameprefix DistRew.south east) + (30,0)$)},
name=\nameprefix SuccEpoch,
ymin=0,xmin=0,ymax=\succymax,xmax=\distxmax,
  \extrapropsaxissuccepoch,
ylabel=Success Rate (test),
  width=1.0\columnwidth,
  height=\axisheight,
  legend pos=\rewlegendpos]
% these constants can be read from \pathprefix PR-\envversion-dqst/params.json
\def\xcol{epoch}
\def\ycol{test/success_rate}
  \addplot table [x=\xcol, y=\ycol, col sep=comma] {\pathprefix PR-\envversion-dqst/progress.csv};
  \addlegendentry{Ours};
  \addplot table [x=\xcol, y=\ycol, col sep=comma] {\pathprefix -\envversion-ddpg/progress.csv};
  \addlegendentry{HER};
  \addplot table [x=\xcol, y=\ycol, col sep=comma] {\pathprefix -\envversion-\fwrlsuffix/progress.csv};
  \addlegendentry{FWRL};
\end{axis}
  \begin{axis}[at={(\nameprefix SuccEpoch.south east)},
ymin=0,xmin=0,ymax=\succymax,xmax=\succxmax,
name=\nameprefix SuccRew,
  \extrapropsaxissuccrew,
  width=1.0\columnwidth,
  height=\axisheight,
  ytick=\empty,
  legend pos=\rewlegendpos]
\def\xexprPR{\thisrow{epoch}*\ncycles*\rolloutB*\T}
\def\xexprGoalRew{\thisrow{epoch}*\ncycles*\rolloutB*\T + \thisrow{epoch}*\ncycles*\batch}
\def\ycol{test/success_rate}

  \addplot table [x expr=\xexprPR, y=\ycol, col sep=comma] {\pathprefix PR-\envversion-dqst/progress.csv};
  \addlegendentry{Ours};
  \addplot table [x expr=\xexprGoalRew, y=\ycol, col sep=comma] {\pathprefix -\envversion-ddpg/progress.csv};
  \addlegendentry{HER};
  \addplot table [x expr=\xexprGoalRew, y=\ycol, col sep=comma] {\pathprefix -\envversion-\fwrlsuffix/progress.csv};
  \addlegendentry{FWRL};
\end{axis}
    \endgroup
    \node[fit=(FetchPickAndPlaceName) (FetchPickAndPlaceSuccRew)] (fetchpickandplace) {}; 
  \end{scope}
  \begin{scope}[shift={($(fetchpickandplace.south west)+(0.05\columnwidth, \scopeyshift)$)},
    local bounding box=fetchslide]
    \begingroup
    \def\rewlegendpos{north west}
    \def\distxmax{200}
    \def\distymax{0.50}
    \def\succxmax{100e4}
    \def\succymax{0.15}
    \def\nameprefix{FetchSlide}
    \def\fullname{Fetch Slide}
    \def\pathprefix{media/res/6efc1de-path_reward_low_thresh_chosen-FetchSlide}
    \def\extrapropsaxisdistepoch{xlabel=Epochs}
    \def\extrapropsaxissuccepoch{xlabel=Epochs},
    \def\extrapropsaxisdistrew{xlabel=Reward Computations},
    \def\extrapropsaxissuccrew{xlabel=Reward Computations},
    \def\axisheight{0.6\columnwidth}
\node [rotate=90,align=center,text width=0.4\columnwidth]
(\nameprefix Name) {\color{blue}\scriptsize \fullname};
\begin{axis}[at={($(\nameprefix Name.south west) + (30, 0)$)},
  ymin=\distymin,xmin=0, ymax=\distymax,xmax=\distxmax,
  name=\nameprefix DistEpoch,
  width=1.0\columnwidth,
  height=\axisheight,
  ylabel=Distance from goal (m),
  legend pos=north east,
  \extrapropsaxisdistepoch,
  ]
  \addplot table [x=epoch, y=test/ag_g_dist, col sep=comma] {\pathprefix PR-\envversion-dqst/progress.csv};
  \addlegendentry{Ours};
  \addplot table [x=epoch, y=test/ag_g_dist, col sep=comma] {\pathprefix -\envversion-ddpg/progress.csv};
  \addlegendentry{HER};
  \addplot table [x=epoch, y=test/ag_g_dist, col sep=comma] {\pathprefix -\envversion-\fwrlsuffix/progress.csv};
  \addlegendentry{FWRL};
  \end{axis}
  \begin{axis}[at={(\nameprefix DistEpoch.south east)},
name=\nameprefix DistRew,
ymin=\distymin,xmin=0,ymax=\distymax,xmax=\succxmax,
  \extrapropsaxisdistrew,
  width=1.0\columnwidth,
  height=\axisheight,
  ytick=\empty,
  legend pos=north east]
  \addplot table [x expr={\thisrow{epoch}*\ncycles*\rolloutB*\T}, y=test/ag_g_dist, col sep=comma] {\pathprefix PR-\envversion-dqst/progress.csv};
  \addlegendentry{Ours};
  \addplot table [x expr={\thisrow{epoch}*\ncycles*\rolloutB*\T + \thisrow{epoch}*\ncycles*\batch}, y=test/ag_g_dist, col sep=comma] {\pathprefix -\envversion-ddpg/progress.csv};
  \addlegendentry{HER};
  \addplot table [x expr={\thisrow{epoch}*\ncycles*\rolloutB*\T + \thisrow{epoch}*\ncycles*\batch}, y=test/ag_g_dist, col sep=comma] {\pathprefix -\envversion-\fwrlsuffix/progress.csv};
  \addlegendentry{FWRL};
\end{axis}
  \begin{axis}[at={($(\nameprefix DistRew.south east) + (30,0)$)},
name=\nameprefix SuccEpoch,
ymin=0,xmin=0,ymax=\succymax,xmax=\distxmax,
  \extrapropsaxissuccepoch,
ylabel=Success Rate (test),
  width=1.0\columnwidth,
  height=\axisheight,
  legend pos=\rewlegendpos]
% these constants can be read from \pathprefix PR-\envversion-dqst/params.json
\def\xcol{epoch}
\def\ycol{test/success_rate}
  \addplot table [x=\xcol, y=\ycol, col sep=comma] {\pathprefix PR-\envversion-dqst/progress.csv};
  \addlegendentry{Ours};
  \addplot table [x=\xcol, y=\ycol, col sep=comma] {\pathprefix -\envversion-ddpg/progress.csv};
  \addlegendentry{HER};
  \addplot table [x=\xcol, y=\ycol, col sep=comma] {\pathprefix -\envversion-\fwrlsuffix/progress.csv};
  \addlegendentry{FWRL};
\end{axis}
  \begin{axis}[at={(\nameprefix SuccEpoch.south east)},
ymin=0,xmin=0,ymax=\succymax,xmax=\succxmax,
name=\nameprefix SuccRew,
  \extrapropsaxissuccrew,
  width=1.0\columnwidth,
  height=\axisheight,
  ytick=\empty,
  legend pos=\rewlegendpos]
\def\xexprPR{\thisrow{epoch}*\ncycles*\rolloutB*\T}
\def\xexprGoalRew{\thisrow{epoch}*\ncycles*\rolloutB*\T + \thisrow{epoch}*\ncycles*\batch}
\def\ycol{test/success_rate}

  \addplot table [x expr=\xexprPR, y=\ycol, col sep=comma] {\pathprefix PR-\envversion-dqst/progress.csv};
  \addlegendentry{Ours};
  \addplot table [x expr=\xexprGoalRew, y=\ycol, col sep=comma] {\pathprefix -\envversion-ddpg/progress.csv};
  \addlegendentry{HER};
  \addplot table [x expr=\xexprGoalRew, y=\ycol, col sep=comma] {\pathprefix -\envversion-\fwrlsuffix/progress.csv};
  \addlegendentry{FWRL};
\end{axis}
    \endgroup
    \node[fit=(FetchSlideName) (FetchSlideSuccRew)] (fetchpickandplace) {}; 
  \end{scope}
\end{tikzpicture}
%\end{figure}%
%\end{wrapfigure}%
\end{minipage}%
}
\caption{For the Fetch tasks, we compare our method (red) against HER (blue) ~\citep{andrychowicz2016learning}
  and FWRL (green) ~\citep{kaelbling1993learning} on the distance-from-goal
  and success rate metrics. Both metrics are plotted
  against two progress measures: the number of training epochs and the number of reward
  computations. Except for the Fetch Slide task, we achieve comparable or
  better performance across the metrics and progress measures. 
}%
\label{fig:fetch-results}%
\end{figure}