\section{Method}

%\section{Goal conditioned value function: Path Reward Formulation}
In our definition of the GCVF instead of making the reward function depend upon
the goal, we count accumulated rewards \emph{only if} the goal is reached. This makes
the dependence on the goal explicit instead  of implicit to the reward formulation.
Mathematically,
%
\begin{subnumcases}{
    \fwargs\state\act{\goal^*}\policy{P} = }
  \E_{\policy}\left[ \sum_{k=t}^{l-1} \discount^{k-t} \Rew^P(\state_k, \act_k)
    \middle| \state, \act, \goal_l = \goal^* \right]
  & $\text{ if } \exists \,l \text{ such that } \goal_l = \goal^*$
  \label{eq:qp-def-a}
  \\
  -\infty & \text{otherwise},
  \label{eq:qp-def-b}
\end{subnumcases}
% 
where $l$ is the time step when the agent reaches the goal. If the agent does
not reach the goal, the GCVF is defined to be negative infinity. This first term
\eqref{eq:qp-def-a} is the expected cumulative reward over
paths from a given start state to the goal.
This imposes the constraint that
cyclical paths in the state space must have negative cumulative reward, for the
\eqref{eq:qp-def-a} to yield finite values. For most practical physical
problems, this constraints naturally holds if reward is taken to be some measure of
negative energy expenditure. For example, in the robot arm experiment, moving the
arm must expend energy (negative reward). Achieving a positive reward cycle
would translate to generating infinite energy.
In all our experiments with this formulation, we use a constant
reward of -1 for all states, $\Rew^P(\state, \act) = -1 \, \forall \state, \act$.

For the cases when the agent does reach the goal \eqref{eq:qp-def-a},
the Bellman equation takes the following form:
%
\begin{subnumcases}{
  \fwargs{\state_t}{\act_t}{\goal^*}*P =}
      \Rew^P(\state_t, \act_t) + \discount \max_{\act \in \Action}
      \fwargs{\state_{t+1}}\act{\goal^*}*P 
      & $\text{ if } t < l-1 \text{ s. t. } \goal_l = \goal^*$
      \label{eq:bellman-path-a}
      \\
      \Rew^P(\state_{l-1}, \act_{l-1})
      & $\text{ if } t = l-1 \text{ s. t. } \goal_l = \goal^*$
      \label{eq:bellman-path-b}.
\end{subnumcases}
% 
This equation differs from Eq~\eqref{eq:bellman-dqn} in the
terminal step being the step to reach the goal in this, while
it being the end of episode in the latter.
Hence, this formulation is equivalent to the end of episode when the goal is reached.
This re-formulation does not require goal-rewards which in turn obviates the
requirement for pseudo-goals and reward re-computation.

\paragraph{One-Step Loss}
To enable algorithms like HER to work under this re-formulation we need to
recognize when the goal is reached~\eqref{eq:bellman-path-b}. This recognition
is usually done by reception of high goal reward. Instead for this formulation,
we propose one-step loss to serve this purpose:
%
\begin{align}
      \LossStep(\param_\fw) &= (\fwargs{\state_{l-1}}{\act_{l-1}}{\goal_{l};\param_\fw}{*}{P} - \Rew(\state_{l-1}, \act_{l-1}))^2.
\end{align}%
%
This loss is based on the assumption that one-step reward is the highest reward
between adjacent start-goal state and allows us to estimate one-step path reward
for all start and goal states. Once learned, the learned one step GCVF serves as
a proxy for the reward to the last step to the goal~\eqref{eq:bellman-path-b}.
The Bellman equation~\eqref{eq:bellman-path-a}, serves as a one-step rollout
to combine rewards over a shortest path to the goal.


We modify an implementation of HER to include the step-loss term and disable goal
rewards for our experiments.
As in HER, we use the DDPG loss $\LossDDPG$ while
using ``future'' goal sampling strategy described in the paper.
The resulting algorithm in shown in Alg~\ref{alg:floyd-warshall-deep}.
\begin{algorithm}
  \tcc{By default all states are unreachable}
  Initialize networks
  $\fwargs{\state_i}{\act_i }{\goal_j; \param_{\fwcost}}{*}{}$ and
  $\policy(\state_i, \state_g; \param_{\policy})$ \;
  Copy the main network to target network
  $\fwargs{\state_i}{\act_i ;\param_{\fwcost}}{\state_j}{t}{} \leftarrow
  \fwargs{\state_i}{\act_i ; \param_{\fwcost}}{\state_j}{*}{} $ \;

  % We do not know the goal location
  Initialize replay memory $M$ \;
  \tcc{Collect experience}
  \For{$e \leftarrow 1$ \KwTo $M$}{
    Sample $\goal_e \in \Goal$ \;
    Set $t \leftarrow 0$\;
    Observe state $\state_t$ and achieved goal $\goal_t$ \;
    \For{$t \leftarrow 1$ \KwTo $\epiT$}{
      Take action $\act_{t} \leftarrow \epsilon\text{-greedy}(\policy(\state_{t}, \goal, \fw))$ \;
      Observe $\state_{t+1}, \goal_{t+1}, \rew_t$ \;
      Store $(\state_{t}, \goal_t, \act_t, \state_{t+1}, \goal_{t+1}, \rew_t; \goal_e)$ in memory $M[e]$ \;
    }
    
    \tcc{Train}
    \For{$t \leftarrow 1$ \KwTo $\epiT$}{
      % Update the transition reward
      
      HER sample batch $B = [
      (\state_{i}, \goal_i, \act_i, \state_{i+1}, \goal_{i+1}, \rew_i;
      \goal_{i+f_i}),
      \dots ,
      (\state_{b}, \goal_b, \act_b, \state_{b+1}, \goal_{b+1}, \rew_b; \goal_{b+f_b})]$ from $M$ \;
      $\Loss(\dots) = 0$ \;
      \For{$b \in B$}{
        $(\state_{b}, \goal_b, \act_b, \state_{b+1}, \goal_{b+1}, \rew_b, \goal_{b+f_b}) = B[b]$ \;
        $\Loss(\dots) += (\fwargs{\state_{b}}{\act_{b}}{\goal_{b+1}}{*}{} - \rew_b)^2$ 
        \tcc*{Step loss}
        $\Loss(\dots) += (\fwargs{\state_{b}}{\act_{b}}{\goal_{b+f_b}}{*}{} -
        \rew_b - \discount\fwargs{\state_{b+1}}{\policy_t(\state_{b+1}, \goal_{b+f_b};\param_\policy)}{\goal_{b+f_b}}{t}{})^2$
        \tcc{DDPG loss}
      }
      Update gradients for $\fw_*$ and $\policy$ using loss $\Loss(\dots)$\;
    }
  }
  \KwResult{$\policy^*(\state_k, \state_g, \fwcost)$}
  \caption{\small Path-reward reinforcement learning}
  \label{alg:floyd-warshall-deep}
\end{algorithm}


%\begin{function}
%\eIf{$\state_g = \phi$ or $\text{all}(\fwcost(\state_t, :, \state_g) = -\infty)$ }{
%  \tcc{Exploration mode}
%  $\act^* = \arg\max_{\act} Q(\state_t, \act)$\;
%}{
%  \tcc{Exploitation mode}
%  $\act^* = \arg\max_{\act} F(\state_t, \act, \state_g)$\;
%}
%
%\caption{Policy()}%$\policy^*(\state_t, \state_g, Q(., .), \fwcost(.,.,.))$}
%\label{func:policy}
%\KwRet{$\act^*$}
%\end{function}

\subsection{Deep Floyd-Warshall Reinforcement Learning}

The GCVF redefinition and step-loss introduced in this paper are inspired by
the tabular formulation of Floyd-Warshall
Reinforcement Learning (FWRL)~\citep{dhiman2018floydwarshall}.
We extend this algorithm to use deep neural networks.
Unfortunately, the algorithm itself does not show significant improvement over
the baselines. However, the intuitions gained in its implementation led to
the contributions of this paper. 

The core contribution of FWRL is a compositionality constraint in the space
of GCVFs. 
This constraint states that the optimal $\Q_*$ value 
from any state $\state_t$ to any goal $\goal_{t+f}$ is greater than or equal to
the sum of optimal $\Q_*$ values via any intermediate state-goal pair $(\state_{w}, \goal_{w})$:
%
\begin{align}
      \fwargs{\state_{t}}{\act_t}{\goal_{w}}{*}{}
      + \fwargs{\state_{w}}{\policy_*(\state_{w}, \goal_{t+f};\param_\policy)}{\goal_{t+f}}{*}{}
      \ge \fwargs{\state_{t}}{\act_t}{\goal_{t+f}}{*}{}.
\end{align}%
% 

We translate these constraints into loss terms and they are added to the
ddpg loss $\LossDDPG$ and one-step loss $\LossStep$.
Taking cue from \citet{MnKaSiNATURE2015}, we do
not repeat the main online network $\Q_{m}$ in the loss term. We use target
network $\Qtgt$ and split the constraint into two loss terms. One loss term acts
as a lower bound $\LossLo$ and the other acts as the upper bound $\LossUp$:
%
\begin{align}
  \LossLo &= \text{ReLU}[
      \fwargs{\state_{t}}{\act_t}{\goal_{w}}{\tgt}{}
      + \fwargs{\state_{w}}{\policy_t(\state_{w}, \goal_{t+f};\param_\policy)}{\goal_{t+f}}{\tgt}{}
      - \fwargs{\state_{t}}{\act_t}{\goal_{t+f}}{m}{}
      ]^2
                         \\
  \LossUp &= \text{ReLU}[
      \fwargs{\state_{t}}{\act_t}{\goal_{w}}{m}{}
      + \fwargs{\state_{w}}{\policy_t(\state_{w}, \goal_{t+f};\param_\policy)}{\goal_{t+f}}{\tgt}{}
      - \fwargs{\state_{t}}{\act_t}{\goal_{t+f}}{\tgt}{}
      ]^2.
\end{align}%
% 
Note that the above terms differ only by choice of the target and main network. 

\paragraph{FWRL Sampling}
We augment HER sampling to additionally get the intermediate state-goal pair
$(\state_{w}, \goal_{w})$.
Once, a transition $(\state_t, \act_t, \rew_t, \state_{t+1})$ and a
future goal $\goal_{t+f}$ has been sampled from the same episode, we sample
another intermediate state and goal pair $(\state_{w}, \goal_{w})$ such that
$t \le im \le t + f$.
% 
% \begin{align}
%   \LossTrieq &= |
%       \fwargs{\state_{b}}{\act_b}{\goal_s}{t}{}
%       + \fwargs{\state_{s}}{\policy_t(\state_s, \goal_{b};\param_\policy)}{\goal_b}{t}{}
%       - \fwargs{\state_{b}}{\act_b}{\goal_b}{m}{}
%       |^2
% \end{align}

% An
%ablation of loss functions is shown in Figure~\ref{fig:fig:fwrl-stepfwrl-noop-FetchPush}.

