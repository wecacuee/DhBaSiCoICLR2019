\section{Method}
\subsection{Goal conditioned tasks: Alternative formulation}
Unlike \citet{andrychowicz2016learning}, we do not believe that reward
information is meaningless without achieving the desired goal state.

We make three assumptions:
\begin{enumerate}
\item We assume that there is no special reward
  on reaching a particular state even the goal state. The reward formulation is
  static even with change in goal states.
  In a graph interpretation, this equivalent to assuming that there are no
  vertex rewards but only edge rewards.
  $\Rew(\state, \goal, \act) = \Rew(\state, \act)$.
\item Like the Floyd-Warshall algorithm we assume no positive reward cycles
  exist i.e. $  \sum_{k=t}^{T-1} \Rew(\state_k, \act_k) \le 0 \text{ if }\state_T = \state_t$.
\end{enumerate}

The benefit of these assumptions is that we do not need to resample the reward
function with simulated goals as HER~\cite{andrychowicz2016learning} need to do.
Resampling the reward function can be expensive.

\subsection{Proposed algorithm: Floyd-Warshall RL}

We present a model-free reinforcement learning method that transfers
learned behavior when goal locations are dynamic. We call this algorithm
Floyd-Warshall Reinforcement Learning, because of its similarity to
Floyd-Warshall algorithm~\cite{floydwarshall1962}:
a shortest-path planning algorithm on graphs. Similar
to universal value function~\cite{schaul2015universal}, we define the Floyd-Warshall
(FW) function as the expected cumulative reward on going from a start
state to an end state:
%
\begin{align}
\fwargs{\state_t = \state}{\act_t = \act}{\goal}{\policy}{} =
\E_{\act_k \sim \policy(\state_t),T}\left[ \sum_{k=t}^{T-1} \Rew(\state_k, \act_k) \middle\vert \state_t = \state, \act_t = \act, \|f_\goal(\state_T) - \goal\|_2 < \epsilon \right] ,
\end{align}%
%

Note that objective is different from reinforcement learning formulated in HER,
as it is conditioned on the goal now. In \citet{andrychowicz2016learning}, the
reward formulation has to reflect that the desired goal to reach. In our formulation
we make it a precondition to the objective function, hence we do not need to
make the reward function dependent upon goal.
We want to find the policy that maximizes the expected reward from any
state $\state_t$ to a desired goal $\goal$:
%
\begin{align}
  \policy^* &=
\arg \max_{\policy} \E_{\state_t, \act_k \sim \policy(\state_t), T}\left[ \sum_{k=t}^{T-1} \Rew(\state_k, \act_k) \middle\vert \state_t = \state, \act_t = \act, \|f_\goal(\state_T) - \goal\|_2 < \epsilon \right],
  \\
  \policy^*(\state, \goal) &= \arg \max_{\act} \fwargs{\state_t = \state}{\act_t = \act}{\goal}{}{*}\\
\text{where } &\fwargs{\state_t = \state}{\act_t = \act}{\goal}{\policy}{*} = 
\max_{\policy} \fwargs{\state_t = \state}{\act_t = \act}{\goal}{\policy}{},
\end{align}%
%

Now all the properties of the Floyd-Warshall function, that we know can be used
as the an objective function.
What are the properties of the Floyd-Warshall
function

\subsubsection{Step loss}
One step reward path is the highest reward path:
\begin{align}
      \LossStep &= (\fwargs{\state_{b}}{\act_{b}}{\goal_{b+1}}{m}{} - \rew_b)^2
\end{align}
%

\subsubsection{DDPG loss}
This is the one step rollout for a long term goal and stays same if we sample a
long term achieved goal $\goal_{b\text{her}}$ using HER sampling:
\begin{align}
  \LossDDPG &= (\fwargs{\state_{b}}{\act_{b}}{\goal_{b\text{her}}}{m}{} -
      \rew_b + \discount\fwargs{\state_{b}}{\policy_t(\state_b, \goal_{b\text{her}};\param_\policy)}{\goal_{b\text{her}}}{t}{})^2
\end{align}
%

\subsubsection{Bounds}
Floyd-Warshall inspires us to come up with these triangular inequality like
bounds that force the highest reward path to be higher that any other path
through an intermediate state.
%
\begin{align}
  \LossUp &= |
      \fwargs{\state_{b}}{\act_b}{\goal_s}{t}{}
      + \fwargs{\state_{s}}{\policy_t(\state_s, \goal_{b};\param_\policy)}{\goal_b}{t}{}
      - \fwargs{\state_{b}}{\act_b}{\goal_b}{m}{}
      |_+^2
                         \\
  \LossLo &= |
      \fwargs{\state_{b}}{\act_b}{\goal_s}{m}{}
      + \fwargs{\state_{s}}{\policy_t(\state_s, \goal_{b};\param_\policy)}{\goal_b}{t}{}
      - \fwargs{\state_{b}}{\act_b}{\goal_b}{t}{}
      |_+^2
\end{align}%
% 

\begin{align}
  \LossTrieq &= |
      \fwargs{\state_{b}}{\act_b}{\goal_s}{t}{}
      + \fwargs{\state_{s}}{\policy_t(\state_s, \goal_{b};\param_\policy)}{\goal_b}{t}{}
      - \fwargs{\state_{b}}{\act_b}{\goal_b}{m}{}
      |^2
\end{align}

The resulting algorithm in shown in Alg~\ref{alg:floyd-warshall-deep}. An
ablation of loss functions is shown in Figure~\ref{fig:fig:fwrl-stepfwrl-noop-FetchPush}.

\begin{algorithm}
  Let $\rew_g \leftarrow 10$\;
  \tcc{By default all states are unreachable}
  Initialize $\fwcost(\state_i, \act_i, \state_j; \param_{\fwcost}) \leftarrow -\infty$ \;
  Initialize $Q(\state_i, \act_i) \leftarrow 1$ \;
  Initialize $\state_g = \phi$ \;
  Set $t \leftarrow 0$\;
  Observe $\meas_t$ \;
  $\state_t = \ObsEnc(\meas_t; \param_E)$ \;
  \For{$t \leftarrow 1$ \KwTo $\epiT$}{
  \tcc{See Function~\ref{func:policy}}
    Take action $\act_{t-1} \sim \text{Egreedy}(\policy^*(\state_{t-1}, \state_g, Q, \fwcost))$ \;
    Observe $\meas_t$, $\rew_t$ \;
    $\state_t = \ObsEnc(\meas_t; \param_E)$ \;
    \If{$\rew_t >= \rew_g$}{
      \tcc{Reached the goal}
      $\state_g \leftarrow \state_t$ \;
      \tcc{Respawning does not need update of value functions}
      continue\;
    }
    $Q(\state_{t-1}, \act_{t-1}) \leftarrow \rew_t + \max_{\act} Q(\state_t, \act)$ \;
    $\fwcost(\state_{t-1}, \act_{t-1}, \state_t) \leftarrow \rew_t$ \;
    \For{$\state_k \in \State, \act_k \in \Act, \state_l \in \State$}{
      $\fwcost(\state_k, \act_k, \state_l) \leftarrow
        \max \{
        \fwcost(\state_k, \act_k, \state_l),
        \fwcost(\state_k, \act_k, \state_t)
        + \max_{\act_p \in \Act} \fwcost(\state_t, \act_p, \state_l)
        \}$
        \;
    }
  }
  \KwResult{$\policy^*(\state_k, \state_g, Q, \fwcost)$}
  \caption{\small Floyd-Warshall Reinforcement Learning (Tabular setting)}
  \label{alg:floyd-warshall-small}
\end{algorithm}


%\begin{function}
%\eIf{$\state_g = \phi$ or $\text{all}(\fwcost(\state_t, :, \state_g) = -\infty)$ }{
%  \tcc{Exploration mode}
%  $\act^* = \arg\max_{\act} Q(\state_t, \act)$\;
%}{
%  \tcc{Exploitation mode}
%  $\act^* = \arg\max_{\act} F(\state_t, \act, \state_g)$\;
%}
%
%\caption{Policy()}%$\policy^*(\state_t, \state_g, Q(., .), \fwcost(.,.,.))$}
%\label{func:policy}
%\KwRet{$\act^*$}
%\end{function}

