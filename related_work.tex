\section{Related Work}

Goal-conditioned tasks in reinforcement learning have been approached in two
ways, depending upon whether the algorithm explicitly separates state and goal
representations.
The first approach is to use vanilla reinforcement
learning algorithms that do not explicitly make this separation
\citep{mirowski2016learning,dosovitskiy2016learning,gupta2017cognitive,parisotto2017neural,mirowski2018learning}.
These algorithms depend upon neural network architectures to
carry the burden of learning the separated representations.

The second approach makes this separation explicit via the use of goal-conditioned value
functions \citep{foster2002structure,sutton2011horde}. \emph{Universal
Value Function Appoximators} \citep{schaul2015universal} propose a
network architecture and a factorization technique that separately
encodes states and goals, taking advantage of correlations in their representations.
\emph{Temporal Difference Models} combine model-free
and model-based RL to gain advantages from both realms by defining and learning
a horizon-dependent GCVF. All these works require the use of
goal-dependent reward functions and define GCVFs as future-rewards instead of
path-rewards, contrasting them from our contribution. 

Unlike our approach, \citet{andrychowicz2017hindsight} propose \emph{Hindsight Experience Replay},
a technique for resampling state-goal pairs from failed experiences; which leads
to faster learning in the presence of sparse rewards. In addition to depending
on goal rewards, HER also requires the repeated recomputation of the reward
function. In contrast, we show how removing goal-rewards removes the need for
such recomputations. We utilize HER as a baseline in our work.

\citet{dhiman2018floydwarshall} also use the structure of the space of GCVFs to
learn.
This work employs compositionality constraints
in the space of these functions to accelerate learning in a tabular domain. While
their definition of GCVFs is similar to ours, they still require
goal-rewards and do not employ one-step loss. We extend their
work to deep neural networks.


%\subsection{Experience Replay and Target Networks}
%The use of \emph{experience replay} to learn from off-policy experiences
%and and slower-changing \emph{target
%networks} are fundamental to the succesful operation of DFWRL. Both
%methods were popularized for use in the DRL literature in
%\citet{mnih2015human}'s original DQN work. 
%
%
%\subsection{Goal conditioned value functions}
%Several variants of goal-conditioned value functions have been explored
%extensively in the literature
%\citep{sutton2011horde,schaul2015universal}.  Driven by the idea that
%simulataneously learning several tasks leads to better generalizability
%than learning a single tasks \citep{pong2018temporal}, goal-conditioned
%value functions have been shown to possess value in manipulation
%\citep{plappert2018multi,peng2018sim} and navigation domains
%\citep{zhang2017deep,mirowski2018learning}. Aside from Dhiman et. al.,
%while much work has utilized these specialized value functions to learn
%how to solve complicated tasks, we were unable to find any other work
%that leveraged the structure in the space of these functions to
%accelerate and improve learning. 
%
%
%\subsection{Multi-goal Reinforcement Learning Methods (HER/TDM)}
%Learning goal-conditioned value functions have also taked a variety of
%forms. In \emph{model-free} RL, \emph{Hindsight Experience Replay} (HER)
%\citep{andrychowicz2017hindsight} represents a form of \emph{implicit
%curriculum} for training these functions. In HER, previous traversals
%are re-used for training with the goal state relabelled to be a part of
%these state traversals. HER sampling is shown to be highlight effective
%in training goal-conditioned value functions quickly and efficiently and
%is highly influential on our methods. In
%\emph{model-based} RL, \emph{Temporal Difference Models}
%\citep{pong2018temporal} derive a
%connection between model-based and model-free algorithms for
%finite-horizon problems. TDM's define and learn goal and
%horizon-conditioned value functions and showcase improved performance
%over scores and sample efficiencies over model-based baselines while
%retaining the asymptotic performance of model-free RL. Being exclusively
%\emph{model-free} and horizon-independent, DFWRL represents an
%orthogonal direction of research as compared to TDMs. 


