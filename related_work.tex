\section{Related Work}


The recent surge in popularity of reinforcement learning occured with
the introduction of \emph{Deep Q Networks} which
learned to play atari video games from raw pixel values while beating
human players \citep{mnih2013playing}. Subsequently,
DQNs saw

Goal-conditioned value functions have been explored extensively in the
literature. Our work builds of the formulation of \emph{Universal Value Function
Approximators}\cite{schaul2015universal}.  


\subsection{Experience Replay and Target Networks}
The use of \emph{experience replay} to learn from off-policy experiences
and and slower-changing \emph{target
networks} are fundamental to the succesful operation of DFWRL. Both
methods were popularized for use in the DRL literature in
\citet{mnih2015human}'s original DQN work. 

\subsection{Curriculum Learning}
Curriculum learning, first introduced by \citet{bengio2009curriculum},
operates on the simple assumption that learning can in complex domains
can be achieved by first iterating over simpler variants of the problem.
This work builds of \emph{Hindsight Experience Replay}
\citep{andrychowicz2017hindsight}, a form of \emph{implicit curriculum}
wherein past failed experiences are relabelled with goal states that in
turn significantly accelerates learning. 


\subsection{Goal conditioned value functions}
Several variants of goal-conditioned value functions have been explored
extensively in the literature
\citep{sutton2011horde,schaul2015universal}.  Driven by the idea that
simulataneously learning several tasks leads to better generalizability
than learning a single tasks \citep{pong2018temporal}, goal-conditioned
value functions have been shown to possess value in manipulation
\citep{plappert2018multi,peng2018sim} and navigation domains
\citep{zhang2017deep,mirowski2018learning}. Aside from Dhiman et. al.,
while much work has utilized these specialized value functions to learn
how to solve complicated tasks, we were unable to find any other work
that leveraged the structure in the space of these functions to
accelerate and improve learning. 


\subsection{Multi-goal Reinforcement Learning Methods (HER/TDM)}
Learning goal-conditioned value functions have also taked a variety of
forms. In \emph{model-free} RL, \emph{Hindsight Experience Replay} (HER)
\citep{andrychowicz2017hindsight} represents a form of \emph{implicit
curriculum} for training these functions. In HER, previous traversals
are re-used for training with the goal state relabelled to be a part of
these state traversals. HER sampling is shown to be highlight effective
in training goal-conditioned value functions quickly and efficiently and
is highly influential on our methods. In
\emph{model-based} RL, \emph{Temporal Difference Models}
\citep{pong2018temporal} derive a
connection between model-based and model-free algorithms for
finite-horizon problems. TDM's define and learn goal and
horizon-conditioned value functions and showcase improved performance
over scores and sample efficiencies over model-based baselines while
retaining the asymptotic performance of model-free RL. Being exclusively
\emph{model-free} and horizon-independent, DFWRL represents an
orthogonal direction of research as compared to TDMs. 


