\def\year{2019}\relax
%File: formatting-instruction.tex
\documentclass[letterpaper]{article} %DO NOT CHANGE THIS
\usepackage{aaai18}  %Required
\usepackage{times}  %Required
\usepackage{helvet}  %Required
\usepackage{courier}  %Required
\usepackage{url}  %Required
\usepackage{graphicx}  %Required
\usepackage{hyperref}
\usepackage{amsmath}
\usepackage{amsfonts}
\usepackage{natbib}
\usepackage{xcolor}
\frenchspacing  %Required
\setlength{\pdfpagewidth}{8.5in}  %Required
\setlength{\pdfpageheight}{11in}  %Required
%PDF Info Is Required:
\setcounter{secnumdepth}{0}  

\usepackage[ruled,vlined]{algorithm2e}
%\newcommand{\TODO}[1]{{\color{red}TODO: {#1}}}
\newcommand{\TODO}[1]{}

\def\state{s}
\def\statet{\state_t}
\def\statetp{\state_{t-1}}
\def\statehist{\state_{1:t-1}}
\def\statetn{\state_{t+1}}
\def\obs{\meas}
\def\obst{\obs_t}
\def\act{a}
\def\actt{\act_t}
\def\acttp{\act_{t-1}}
\def\acttn{\act_{t+1}}
\def\Obs{\mathcal{O}}
\def\ObsEnc{\Phi_o}
\def\ObsProb{P_o}
\def\ObsFunc{C}
\def\ObsFuncFull{\ObsFunc(\statet, \actt) \rightarrow \obst}
\def\ObsFuncInv{\ObsFunc^{-1}}
\def\ObsFuncInvFull{\ObsFuncInv(\obst, \statetp, \actt) \rightarrow \statet}
\def\State{\mathcal{S}}
\def\Action{\mathcal{A}}
\def\TransP{P_{T}}
\def\Trans{T}
\def\TransFull{\Trans(\statet, \actt) \rightarrow \statetn}
\def\TransObs{T_c}
\def\Rew{R}
\def\rew{r}
\newcommand{\vect}[1]{\mathbf{#1}}
\def\rewards{\vect{r}_{1:t}}
\def\rewt{\rew_t}
\def\rewtp{\rew_{t-1}}
\def\rewtn{\rew_{t+1}}
\def\RewFull{\Rew(\statet, \actt) \rightarrow \rewtn}
\def\TransObsFull{\TransObs(\statet, \obst, \actt, \rewt; \theta_T) \rightarrow \statetn}
\def\Value{V}
\def\pit{\pi_t}
\def\piDef{\pi(\acttn|\statet, \obst, \actt, \rewt; \theta_\pi) \rightarrow \pit(\acttn ; \theta_\pi)}
\def\Valuet{\Value_t}
\def\ValueDef{\Value(\statet, \obst, \actt, \rewt; \theta_\Value) \rightarrow \Valuet(\theta_\Value)}
\def\R{\mathbb{R}}
\def\E{\mathbb{E}}
\newcommand{\meas}{z}
\newcommand{\measurements}{\vect{\meas}_{1:t}}
\newcommand{\meast}[1][t]{\meas_{#1}}
\newcommand{\param}{\theta}
\newcommand{\policy}{\chi}
\newcommand{\graph}{G}
\newcommand{\vtces}{V}
\newcommand{\edges}{E}
\newcommand{\st}{\state}
\newcommand{\stn}{\st_{t+1}}
\newcommand{\stt}{\st_t}
\newcommand{\stk}{\st_k}
\newcommand{\stj}{\st_j}
\newcommand{\sti}{\st_i}
\newcommand{\St}{\mathcal{S}}
\newcommand{\Act}{\mathcal{A}}
\newcommand{\acti}{\act_i}
\newcommand{\lpt}{\delta}
\newcommand{\trans}{P_T}
\newcommand{\Q}{\qValue}
\newcommand{\V}{V}
\newcommand{\fw}{\fwcost}

\newcommand{\fwcost}{F}
\newcommand{\qValue}{Q}
\newcommand{\prew}{\Upsilon}
\newcommand{\epiT}{T}
\newcommand{\vma}{\alpha_\Value}
\newcommand{\qma}{\alpha_\qValue}
\newcommand{\prewma}{\alpha_\prew}
\newcommand{\fwma}{\alpha_\fwcost}
\newcommand{\maxValueBeam}{\vect{\state}_{\Value:\text{max}(m)}}
\newcommand{\nil}{\emptyset}
\newcommand{\discount}{\gamma}
\newcommand{\minedgecost}{\fwcost_0}
\newcommand{\goal}{g}
% The file aaai.sty is the style file for AAAI Press 
% proceedings, working notes, and technical reports.
%
\title{Floyd-Warshall Reinforcement Learning: \\Combining Advantages from
Model-Free and Model-Based RL \\ Improving Performance in Multi-Goal
Environments}
\author{Vikas Dhiman, Shurjo Banerjee, Jeffrey M. Siskind and Jason J.
Corso}
\pdfinfo{
/Title ()
/Author ()}
\begin{document}

\maketitle
\begin{abstract}
Reinforcement learning algorithms are oftentimes classified as either
\emph{model-free} or \emph{model-based} based on whether a state-dynamics
is learned. In the \emph{model-free} case, algorithms such as
Q-learning and policy gradients learn the optimal action value function that
maximizes expected future reward for all state-action pairs perceived by an
agent in an environment.
% VD: Not sure about that
% The reward distribution and state dynamics that governs the agent's
% environment are implicitly memorized in the learning of this function.
The action value function depends upon both reward distribution and state
dynamics but neither can be recovered from the learned action value function.
That is why, \emph{model-free} RL struggles with transferring learned behaviours
to tasks where only the reward distribution can change, for example,
goal-conditioned tasks.
In contrast, \emph{model-based} RL explicitly learns the state-dynamics
function allowing for the transfer of an agent's learned behaviours to
environments in which only the reward distribution changes. Oftentimes,
this explicit modelling requires  an additional planning step to predict
state-dynamics making policy computation in such algorithms
an expensive process.  Inspired by both these paradigms and the
Floyd-Warshall algorithm for path-planning on graphs,  this work
introduced Floyd-Warshall Reinforcement Learning (FWRL), a novel
algorithm that combines advantages from both \emph{model-based} and
\emph{model-free} approaches.  The algorithm works by learning a
goal-conditioned value function that transitions from model-based
behavior to model-free behavior in well explored regions of an agent's
state space.  FWRL is shown to transfer knowledge about an environment
when the reward location is dynamic compared to a model-based Q-learning
baseline.  FWRL is shown to meet the ground between model-free and
model-based algorithms by being model-free in the more frequently
visited regions while being model-based on less visited paths.

%The Reinforcement learning algorithms are classified on the basis of whether
%the state-transition model is learned explicitly into \emph{model-based}
%and \emph{model-free} algorithms.
%%
%The \emph{model-free} algorithms like Q-learning and policy gradients are easier to learn in cases when model is more complex than the policy.
%However, model-free algorithms are harder to transfer in cases when the reward function changes while the state-transition model remains the same.
%The \emph{model-based} algorithms learn the state-transition model explicitly hence making it
%easier to separate the environment transition from the reward distribution.
%However, model-based algorithms require an additional planning step on the model dynamics
%which make it expensive to compute the policy.
%%
%We propose a novel algorithm that is based on a combination of model-free and model-based approaches.
%Due to it's similarity to Floyd-Warshall algorithm in path-planning on graphs, we call
%it Floyd-Warshall reinforcement learning (FWRL).
%We show that FWRL transfers the knowledge about the environment when the reward location is dynamic as compared to the model-based Q-learning algorithm.
%We show (\TODO{fill in the results when exp done}) that FWRL meets the ground between model-free and model-based algorithms by being model-free in the more frequently visited regions while being model-based in less visited paths.
\end{abstract}


\section{ Introduction}

\section{Introduction}

% Reinforcement learning and it's uses
% Why we care about Multi-Goal Reinforcement Learning
Many tasks in robotics require the specification of a \emph{goal} for every
trial. For example, a robotic arm may need to move an object to an arbitrary
goal position on a table \citep{gu2017deep}; a mobile robot may need to with
navigating to a arbitrary goal landmark on a map \citep{zhu2017target}. The
adaptation of reinforcement learning to such goal-conditioned tasks where goal
locations can change is called Multi-Goal Reinforcement Learning
(MGRL)~\citep{plappert2018multi}.

% A common way to solve MGRL problems is the use of Goal-Conditioned
% Value functions. Solving several tasks instead of individual tasks
State-of-the-art MGRL algorithms
\citep{andrychowicz2017hindsight, pong2018temporal}
work by estimating \emph{goal-conditioned value functions} (GCVF). A
GCVF is defined as the expected cumulative reward from a
start state with a specified the goal. It is used to compute the
actions to be take at any state, also known as the \emph{policy}.

% Goal conditioned value functions make a big assumption however. It is
% assumed that to learn to achieve goals, goal locations must themselves
% must be associated with reward. This however is not the case. We
% introduce ....
These algorithms require the use of \emph{goal-rewards}, which is the
dependence of reward functions on the desired goal, $r(s,a,g)$.
For example, in the Fetch-Push task \citep{plappert2018multi} of moving
a block to a given location on a table, every movement incurs a ``-1''
reward while reaching the desired goal induces ``0'' reward. Although
this dependence allows for the agent to associate reaching the goal with a high
reward, it may incur additional sampling costs. For example, in Hindsight Experience
Replay (HER) \citep{andrychowicz2017hindsight}, failed experiences are
re-evaluated as succesful ones by assuming traversed states to be
desired pseudo-goals. Due to the dependence of reward function on the goal,
the association of every pseudo-goal with high reward requires an independent
\emph{reward-recomputation} step. 

%This leads to the
%requirement of \emph{reward-recomputation}, which is that these reward functions can be
%computed indendent of \emph{environment} interaction.  The second assumption enables a technique called \emph{Hindsight
%Experience Replay} (HER) that accelerates learning 

% Do goal-conditioned value functions require goal rewards to learn?
% Describe some form of the intuition
The goal-rewards requirement is avoidable. Let us consider an example to
motivate why goal-rewards may be avoided in goal-conditioned tasks. Consider a
student who has moved to a new university. To learn about the campus,
the student explores it randomly with no specific goal in mind. When
tasked with finding a goal classroom, the student can then use the
learned connectivity of the campus to make their way to class along the
shortest explored path. The key intuition here is that the
student is not incentivized to explore specific goal locations (i.e. no
goal rewards). 
%
Based on this intuition of goal-less learning, we redefine GCVFs to be
the expected \emph{path-reward} that is learned for all possible
start-goal pairs. Instead of goal rewards, we  
introduce a \emph{one-step loss} that assumes single steps paths 
to be the maximum reward paths between adjacent pairs.
Under this interpretation, the \emph{Bellman equation} chooses and chains
together one-step paths to find longer maximum reward paths. 

%
Experimentally, we show how this simple reinterpration, which does not use goal
rewards, performs as well as a representative state-of-the-art method (HER) and
even outperforms it in terms of reward sample complexity.

% Detail the Floyd-Warshall contribution
In this work, we also extend a closely related algorithm, Floyd-Warshall Reinforcement
Learning (FWRL) \citep{dhiman2018floydwarshall} to use parametric
function approximators instead of tabular functions. Similar to our
re-definition of GCVFs, FWRL learn a goal-conditioned Floyd-Warshall function
that represents path-rewards instead of future-rewards.
We translate FWRL's compositionality constraints in the space of GCVFs to introduce
additional loss terms to the objective. However, these additional loss
terms do not show improvement over the baseline. We conjecture that the
compositionality constraints are already captured by other loss terms. 

% Summarize 
In summary, the  contributions of this work are twofold. Firstly, we
reinterpret goal-conditioned value functions as expected path-rewards
and introduce one-step loss thereby removing their dependency on
goal-rewards and reward-resampling. We showcase how the
application of this interpreation leads to improved sample efficiency in  terms
of reward samples.
Secondly, we extend the tabular Floyd-Warshal Reinforcement Learning to
use deep neural networks.


%% There is inherent structure in the formulation of these functions
%% thathas been ignored. Dhiman et al introduced structure in the space
%% of these functions 
%Floyd-Warshall Reinforcement Learning was first introduced in Dhiman et.
%al. for tabular domains. 
%The FW functions is defined 
%%
%\begin{align}
%\fwargs{\state}{\act}{\state'}{\policy}{} =
%\E_{\policy}\left[ \sum_{t=0}^{t=k} \rew_t \middle\vert \state_0 = \state, \act_0 = \state, \state_k = \state' \right] .
%\end{align}%
%%
%which is the expected sum of rewards when the state is s and the
%end state is the goal. Compositionality constraints are then used to
%learn the FW function.
%%
%\begin{align}
%\fwargs{\state_i}{\act_i}{\state_j}{\policy^*_{\state_j}}{*}
% \ge 
%  \fwargs{\state_i}{\act_i}{\state_k}{\policy^*_{\state_k}}{*}
%  + \max_{\act_k}\fwargs{\state_k}{\act_k}{\state_j}{\policy^*_{\state_j}}{*}
%  ,\forall \state_k \in \State.
%  \label{eq:fwconstraint}
%\end{align}%
%%
%The constraint is inspired by the Floyd-Warshall algorithm for path
%planning on graphs and states that the expected reward in traversing
%from one state to another should be greater than or equal to that of the
%paths through intermediary states.  While Dhiman et al. was restritcted
%to toy examples in the gridworld doman and tabular functions, they are
%not naturally scalabale to real life problems. This work extends their
%formulation to paramteric functions like neural networks by addition
%terms to the RL objective based on these compositionality constraints.
%Since we extend the algorithm to the Deep Learning domain, we call it
%Deep Floyd-Warshall Reinforcement Learning. 









\subsection{Claims}
\begin{itemize} \item
Using FWRL is better than model-free learning like Q-learning in terms of generalizing to dynamic goals, static maps.
\item 
Using FWRL is better than model-based learning because of less accumulation over long term, hence more accurate value functions.
\item FWRL is better than Hindsight Experience Replay (HER) \cite{andrychowicz2016learning} because it is a parametric representation of hindsight.
\item FWRL is better than TDM~\cite{pong2018temporal} because it does not require the distance reward function.
\end{itemize}

\section{Related work}
\subsection{Goal-conditioned value functions}
The idea of goal-conditioned value function is not new but has got attention because of revival of reinforcement learning based on deep neural networks.
We build upon the goal-conditioned value functions of \citet{schaul2015universal}.
\citet{schaul2015universal} proposed an architecture and a matrix factorization based algorithm for faster learning of UVFA (Universal value function approximators).
UVFA focused on fast estimation of goal-conditioned value functions using sparse
matrix factorization but not on bridging the gap between model-based and model-free
algorithms.


\citet{andrychowicz2016learning} introduced the idea of Hindsight experience replay (HER) two learn about the model from previous episodes even when the goal location has changed or not been achieved.
Our method can be seen as parametric approximation of ``Hindsight memory'',
that can help compress information from previous episodes instead of maintaining the
entire history of replay memory.

\cite{pong2018temporal} propose temporal difference models that estimate goal directed Q function for a specific kind of reward function, in particular the distance from the goal and in contrast with limited temporal horizon.

\subsection{Combining model-based and model-free methods}



\subsection{Navigation with mapping}
 (1) CMP from Saurabh Gupta: is metric, might not working in continuous spaces.
 (2) Semi-parameteric Topological mapping: is not end to end.
 (3) Neural Map: Is actually not mapping

\subsection{Model free DRL }
does not generalize to multi-goal environments.

\subsection{Model based DRL}
Needs more exploration.
Find the paper that shows that Model based DRL can actually compete with Model free DRL as long as it models uncertainty.

\subsection{Multi-goal navigation based papers}
Mirowski 2017, 2018: No one shot map learning, does not generalizes to new maps.

\section{Problem definition}
\newcommand{\Rgoal}{R_{\text{goal}}}
Let the agent and its environment be represented by a state space $\State$ and the
agent can traverse through state space by taking actions $\act \in \Act$ in a
fixed action space $\Act$. At every time step $t$, the agent takes action $\act_t$
and observes state $\state_t \in \State$ and reward
$\rew_t \in [-\Rgoal, \Rgoal]$.
Consider an episode of $T$ time steps.
For every episode the environment chooses one of the states in the state space is
the goal state $\goal \in \State$.
We want to find the sequence of actions to take that
maximizes the total reward over $T$ time steps.
Once the agent reaches the goal $\|\state_t - \goal\| < \delta$, the agent receives the highest reward possible $\Rgoal$ in the game and gets respawned in the environment at a randomg \emph{spawn} state.
This cycle continues enabling the agent to reach the goal multiple times,
incentivizing rapid exploration and finding the shortest path to the goal.
For the new episode a new goal location is chosen, but rest of the environment stays the same.
The agent is allowed to remember about the environment from previous episode and use
it for this episode.

%
\begin{align}
\policy^*(\state_t ; \goal) = \act^*_t = \arg \max_{\act_t} \E_{\policy}\left[ \sum_{t=0}^T \rew_t \right]
\end{align}%
%

\subsection{Why is this problem important?}
Many real world problems can be formulated in this context.
Consider a traveling postman problem who has moved into a new city. The postman has
to explore the city and find the buildings that match the given address.
The next time the postman gets the same address, they can use their experience to find out the building.
Even when a new address is provided (in the next episode), the postman can use
experience to find the new episode more quickly.

In robotics, tasks like picking and placing the object at a desired location can be
formulated as goal-directed navigation.

\subsection{Why is the problem hard?}
Model-free Reinforcement learning methods assume that the rewards are being sampled
from the a static reward function.
In a problem where the goal location changes, hence the reward function also changes,
it becomes hard to transfer the learned value-function or action-value function to
the changed location.
One alternative is to concatenate the goal location the state, making the new state space
$[\state_t, \goal]^\top \in \State^2$ larger.
This method is wasteful in computation and more importantly in sample complexity.

\section{Method}
We present a model-free reinforcement learning method that easily transfers when goal
location is dynamic.
We accomplish this by maintaining a path based expected reward function from any state to any goal state.
We call this algorithm Floyd-Warshall Reinforcement Learning, because of its
similarity to Floyd-Warshall algorithm : a shortest-path planning algorithm on graphs.
We define Floyd-Warshall value function as
%
\begin{align}
\fwcost_{\policy}(i, l,  j) =
\E_{\policy}\left[ \sum_{t=0}^{t=k} \rew_t \middle\vert \state_0 = i, \act_0 = l, \state_k = j \right] .
\end{align}%
%
When the policy is optimal, the Floyd-Warshall function should satisfy the constraint
%
\begin{align}
\fwcost^*(\state_i, \act_i, \state_j) = \max_{\state_k} \left[
\fwcost^*(\state_i, \act_i, \state_k)
+ \max_{\act_k}\fwcost^*(\state_k, \act_k, \state_j) \right] .
\end{align}%
%

We summarize the algorithm in Alg~\ref{alg:floyd-warshall-small}.

\section{Experiments}
We setup two environments in the grid world and windy grid world.
\subsection{Four room grid world}

Four room grid world is a grid world with four rooms connected to each other as shown in Figure~\ref{fig:four-room-grid-world}.

\subsection{Four room windy world}

Four room windy world is a grid world with four rooms connected to each other as shown in Figure~\ref{fig:four-room-grid-world}.
Some of the grid cells in have wind shown by arrow and the agent gets pushed around by
the wind with 0.25 probability irrespective of the action taken.


%
\begin{figure}%
\includegraphics[width=0.48\columnwidth]{media/4-room-grid-world.pdf}
\hfill
\includegraphics[width=0.48\columnwidth]{media/4-room-windy-world.pdf}%
\caption{Left: Four room grid world. Right: Four room windy grid world with wind direction shown by arrows. The windy pushes the agent in the direction of wind with 0.25 probability irrespective of the action taken.}
\label{fig:four-room-grid-world}%
\end{figure}%
%

\section{Results}

We evaluate Q-learning and Floyd-Warshall Reinforcement Learning (FWRL)  on two
metrics in two different environments. The two metrics we use are Latency Ratio
and average reward per episode. The Latency ratio metric was introduced in
\citet{MiPaViICLR2017}, which is defined as the ratio of time taken to reach the
goal for the first to time to the average time taken to hit the goal thereafter.
The Latency ratio thus measures the ratio of exploration time for first time
finding the goal to the average exploitation time to reach the goal. Hence,
higher latency ratio is better.
Fig~\ref{fig:ql-fw-grid-world-results} and
Fig~\ref{fig:ql-fw-windy-world-results} show the results.

\begin{figure}%
\includegraphics[width=\columnwidth]{./media/ql-fw-grid-world.pdf}\\
\caption{Results on grid world. FWRL beats Q-Learning consistently. Higher is
  better in both the metrics (higher is better).}
\label{fig:ql-fw-grid-world-results}%
\end{figure}
\begin{figure}
\includegraphics[width=\columnwidth]{./media/ql-fw-windy-world.pdf}%
\caption{Results on windy world. FWRL beats Q-Learning consistently. Higher is
  better in both the metrics (higher is better).}
\label{fig:ql-fw-windy-world-results}%
\end{figure}

\subsection{Conclusion}
Floyd-Warshall Reinforcement Learning (FWRL) allows us to learn a goal
conditioned action-value function which is invariant to the change in goal
location as compared to the action-value function used in typical Q-learning.
This allows FWRL to transfer learned behaviors about the environment when the
goal location changes. Many tasks like navigation, robotic pick and place are
examples of goal-conditioned tasks that can benefit from this framework.

%\subsection{Future work}
%Items to improve the algorithm:
%\begin{itemize} \item
%\TODO{Justify the computational cost of constraint} The cost of going through the entire state space.
%How do you extend to a network? and large state spaces.
%\begin{enumerate}\item
%Observation 1: If there is only one goal, then the computation should not be any more than Q-learning.
%This can be accomplished by assuming that transitivity is satisfied till
%$\state_{t-1}$ and needs to be extend to only the next step. This sounds similar to the
% Floyd-Warshall dynamic programming update.
%However, this assumes that $\state_t$ is being visited for the first time.
%If the state $\state_t$ is being visited for the second time, the earlier
%value may be the shorter path for it.
%\end{enumerate}
%\item
%We need Q-value for exploration.
%\end{itemize}

\begin{function}
\eIf{$\state_g = \phi$ or $\text{all}(\fwcost(\state_t, :, \state_g) = -\infty)$ }{
  \tcc{Exploration mode}
  $\act^* = \arg\max_{\act} Q(\state_t, \act)$\;
}{
  \tcc{Exploitation mode}
  $\act^* = \arg\max_{\act} F(\state_t, \act, \state_g)$\;
}

\caption{Policy()}%$\policy^*(\state_t, \state_g, Q(., .), \fwcost(.,.,.))$}
\label{func:policy}
\KwRet{$\act^*$}
\end{function}

\begin{algorithm}
  Let $\rew_g \leftarrow 10$\;
  \tcc{By default all states are unreachable}
  Initialize $\fwcost(\state_i, \act_i, \state_j; \param_{\fwcost}) \leftarrow -\infty$ \;
  Initialize $Q(\state_i, \act_i) \leftarrow 1$ \;
  Initialize $\state_g = \phi$ \;
  Set $t \leftarrow 0$\;
  Observe $\meas_t$ \;
  $\state_t = \ObsEnc(\meas_t; \param_E)$ \;
  \For{$t \leftarrow 1$ \KwTo $\epiT$}{
  \tcc{See Function~\ref{func:policy}}
    Take action $\act_{t-1} \sim \text{Egreedy}(\policy^*(\state_{t-1}, \state_g, Q, \fwcost))$ \;
    Observe $\meas_t$, $\rew_t$ \;
    $\state_t = \ObsEnc(\meas_t; \param_E)$ \;
    \If{$\rew_t >= \rew_g$}{
      \tcc{Reached the goal}
      $\state_g \leftarrow \state_t$ \;
      \tcc{Respawning does not need update of value functions}
      continue\;
    }
    $Q(\state_{t-1}, \act_{t-1}) \leftarrow \rew_t + \max_{\act} Q(\state_t, \act)$ \;
    $\fwcost(\state_{t-1}, \act_{t-1}, \state_t) \leftarrow \rew_t$ \;
    \For{$\state_k \in \State, \act_k \in \Act, \state_l \in \State$}{
      $\fwcost(\state_k, \act_k, \state_l) \leftarrow
        \max \{
        \fwcost(\state_k, \act_k, \state_l),
        \fwcost(\state_k, \act_k, \state_t)
        + \max_{\act_p \in \Act} \fwcost(\state_t, \act_p, \state_l)
        \}$
        \;
    }
  }
  \KwResult{$\policy^*(\state_k, \state_g, Q, \fwcost)$}
  \caption{\small Floyd-Warshall Reinforcement Learning (Tabular setting)}
  \label{alg:floyd-warshall-small}
\end{algorithm}

\begin{itemize}\item
Grid world: Set up a random goal static maze scenario, compare with normal Q-learning.
\item
Lava world: Set up a Lava world like \cite{schaul2015universal} and test on it.
\item
\TODO{Atari games}: Compare performance with normal Q-learning.
\item
\TODO{Deepmind Lab}: Set up a random goal static maze scenario, compare with normal Q-learning. 
\end{itemize}



\def\localbib{/home/dhiman/wrk/group-bib/shared}
\IfFileExists{\localbib.bib}{
\bibliography{\localbib,main,main_filtered}}{
\bibliography{main,main_filtered}}
\bibliographystyle{aaai}
\end{document}
