\def\year{2019}\relax
%File: formatting-instruction.tex
\documentclass[letterpaper]{article} %DO NOT CHANGE THIS
\usepackage{aaai18}  %Required
\usepackage{times}  %Required
\usepackage{helvet}  %Required
\usepackage{courier}  %Required
\usepackage{url}  %Required
\usepackage{graphicx}  %Required
\frenchspacing  %Required
\setlength{\pdfpagewidth}{8.5in}  %Required
\setlength{\pdfpageheight}{11in}  %Required
%PDF Info Is Required:
  \pdfinfo{
/Title (2018 Formatting Instructions for Authors Using LaTeX)
/Author (AAAI Press Staff)}
\setcounter{secnumdepth}{0}  

\usepackage[ruled,vlined]{algorithm2e}

\def\state{s}
\def\statet{\state_t}
\def\statetp{\state_{t-1}}
\def\statehist{\state_{1:t-1}}
\def\statetn{\state_{t+1}}
\def\obs{\meas}
\def\obst{\obs_t}
\def\act{a}
\def\actt{\act_t}
\def\acttp{\act_{t-1}}
\def\acttn{\act_{t+1}}
\def\Obs{\mathcal{O}}
\def\ObsEnc{\Phi_o}
\def\ObsProb{P_o}
\def\ObsFunc{C}
\def\ObsFuncFull{\ObsFunc(\statet, \actt) \rightarrow \obst}
\def\ObsFuncInv{\ObsFunc^{-1}}
\def\ObsFuncInvFull{\ObsFuncInv(\obst, \statetp, \actt) \rightarrow \statet}
\def\State{\mathcal{S}}
\def\Action{\mathcal{A}}
\def\TransP{P_{T}}
\def\Trans{T}
\def\TransFull{\Trans(\statet, \actt) \rightarrow \statetn}
\def\TransObs{T_c}
\def\Rew{R}
\def\rew{r}
\def\rewards{\vect{r}_{1:t}}
\def\rewt{\rew_t}
\def\rewtp{\rew_{t-1}}
\def\rewtn{\rew_{t+1}}
\def\RewFull{\Rew(\statet, \actt) \rightarrow \rewtn}
\def\TransObsFull{\TransObs(\statet, \obst, \actt, \rewt; \theta_T) \rightarrow \statetn}
\def\Value{V}
\def\pit{\pi_t}
\def\piDef{\pi(\acttn|\statet, \obst, \actt, \rewt; \theta_\pi) \rightarrow \pit(\acttn ; \theta_\pi)}
\def\Valuet{\Value_t}
\def\ValueDef{\Value(\statet, \obst, \actt, \rewt; \theta_\Value) \rightarrow \Valuet(\theta_\Value)}
\def\R{\mathbb{R}}
\def\E{\mathbb{E}}
\newcommand{\meas}{z}
\newcommand{\measurements}{\vect{\meas}_{1:t}}
\newcommand{\meast}[1][t]{\meas_{#1}}
\newcommand{\param}{\theta}
\newcommand{\policy}{\chi}
\newcommand{\graph}{G}
\newcommand{\vtces}{V}
\newcommand{\edges}{E}
\newcommand{\state}[2]{\mathbf{s}^{#1}(#2)}
\newcommand{\egos}[1][t]{\state{c}{#1}}
\newcommand{\st}{\state}
\newcommand{\stn}{\st_{t+1}}
\newcommand{\stt}{\st_t}
\newcommand{\stk}{\st_k}
\newcommand{\stj}{\st_j}
\newcommand{\sti}{\st_i}
\newcommand{\St}{\mathcal{S}}
\newcommand{\Act}{\mathcal{A}}
\newcommand{\acti}{\act_i}
\newcommand{\actt}{\act_t}
\newcommand{\lpt}{\delta}
\newcommand{\trans}{P_T}
\newcommand{\Q}{\qValue}
\newcommand{\V}{V}
\newcommand{\fw}{\fwcost}

\newcommand{\fwcost}{F}
\newcommand{\qValue}{Q}
\newcommand{\vma}{\alpha_\Value}
\newcommand{\qma}{\alpha_\qValue}
\newcommand{\prewma}{\alpha_\prew}
\newcommand{\fwma}{\alpha_\fwcost}
\newcommand{\maxValueBeam}{\vect{\state}_{\Value:\text{max}(m)}}
\newcommand{\nil}{\emptyset}
\newcommand{\discount}{\gamma}
\newcommand{\minedgecost}{\fwcost_0}
\begin{document}
% The file aaai.sty is the style file for AAAI Press 
% proceedings, working notes, and technical reports.
%
\title{Floyd warshall deep reinforcement learning}
\author{}
\maketitle
\begin{abstract}
Problem: Multi-goal navigation without mapping
\\
Model free deep reinforcement learning is to learn $\policy(\act | \st)$ or $\Q(\st, \act)$
\\
Model based deep reinforcement learning is to learn $\trans(\stn| \stt, \actt)$ and $\V(\st)$.
and planning on a graph to get the shortest path.
Note that $\trans(.)$ is highly sparse and keeping a list of non-zero $\stn$ is much better than
keeping all the value of $\trans(.)$ for all $\stn \in \St$.
\\
Floyd-Warshall deep reinforcement learning is to generalize model based DRL to
directly learn the $\fw(\stj|\sti, \acti)$ which is the cost of reaching state $\stj$
starting from $\sti$ when the first action taken is $\acti$.
If we directly try to learn $\fw(.)$ we are likely to get conflicting results
that do not obey FW identity
$\fw(\stj|\sti, \acti) = \min_{\stk} \min_\act \fw(\stj|\stk, \act) + \fw(\stk | \sti, \acti)$
It is expected that since we will be visiting nearby states more often, so the
$\fw(.)$ will be consistent over small distances but will grow inconsistent over
large distances.
We can draw few samples from $\fw(.)$ to check for it's inconsistencies and then 
plan over graph over higher ranges.
\end{abstract}


%\section{ Introduction}
%Navigation is the problem of finding shortest path from one point to another.
%Multi-goal navigation is the problem of reaching a sequence of goals in a specified order.
%Multi-goal navigation requires one to remember the map to accomplish the goal faster.
%If we include exploration as a part of navigation, then we need to explore what paths are
%available from one point to another and we also need to find the shortest path among those
%paths and the paths formed by joining those paths.
%The space can be thought of an unexplored graph with nodes as the states $\st \in \St$ and
%transition probabilities $P(\st_{t+1} | \st_t, a_t)$ as weighted edges.
%The problem of constructing a representation of the map is building a representation that
%explores the connectivity between different states by taking different actions.
%How much does change the shortest path from any point to any other point?
%Do we need to capture the details that we are within visible region from a point?
%Assuming that we have a closed loop system at the time of execution, we can execute shortest
%path planning at a higher detail in the visible area.
%We only need to capture the shortest path on a resolution coarser than visible area.
%How much area is visible are when there are no walls?
%Dependent upon the size of the robot, we need to know the
%minimum size of obstacle that will obstruct the robot.
%Depending upon the resolution of obstacle detection, what is the maximum distance at which we
%can detect the minimum sized obstacle that will obstruct the robot.
%Let us call this distance as local planning threshold $\lpt$.
%In an open space there should be a node at least in a radius of $\lpt$.

\subsection{Claims}
\begin{itemize} \item
Using Floyd Warshall value function leads to better generalization in case of static maps and
random goals.
\item
Hypothesis: Multi-goal navigation is more common than we think. Does FW algo improves performance in attari games.
\end{itemize}

\section{Related work}
\subsection{Navigation with mapping}
 (1) CMP from Saurabh Gupta: is metric, might not working in continuous spaces.
 (2) Semi-parameteric Topological mapping: is not end to end.
 (3) Neural Map: Is actually not mapping

\subsection{Model free DRL }
does not generalize to multi-goal environments.

\subsection{Model based DRL}
Needs more exploration.
Find the paper that shows that Model based DRL can actually compete with Model free DRL as long as it models uncertainty.

\subsection{Multi-goal navigation based papers}
Mirowski 2017, 2018: No one shot map learning, does not generalizes to new maps.


\section{Method}

See Alg~\ref{alg:floyd-warshall-small}
Over simplified. Ignoring the cost of going through the entire state space.
    \begin{algorithm}
      \KwData{Graph $\graph_0 = (\vtces, \edges)$\;}
      Initialize $\fwcost(\state_i, \act_i, \state_j; \param_{\fwcost}) = 100$ \;
      Initialize $\qValue(\state_i,\act_i; \param_{\qValue}) = 1$ \;
      Initialize $\vma = 0.1$, $\prewma = 0.9$ \;
      Let minimum path cost $\minedgecost = 0.05$ \;
      Observe $\meas_0$ from environment \;
      $\state_0 = \ObsEnc(\meas_0; \param_E)$ \;
      \For{$t \leftarrow 1$ \KwTo $\epiT$}{
        Take action $\act_{t-1}$\;
        Observe $\meas_t$, $\rew_t$\;
        Encode state $\state_t = \ObsEnc(\measurements; \param_E)$\;
        \tcc{Initialize new FW values}
        $\fwcost(\state, \act, \state_{t})
        = \min \{
               \fwcost(\state, \act, \state_t),
                \fwcost(\state, \act, \state_{t-1}) + \minedgecost
            \}
        \qquad \forall \state \in \State, \act \in \Action$ \;
        \tcc{Q-Value update}
        $\qValue(\state_{t-1}, \act_{t-1}) = (1-\qma) (\rew_t + \discount \max_{\act_k}\qValue(\state_t, \act_k)) + \qma \qValue(\state_{t-1}, \act_{t-1})$\;
        \If{$\state_t$ is visited the first time}{
            \For{$(\state_i, \state_k, \act_k) \in (\State \times \State \times \Action)$}{
                \tcc{Run the Floyd Warshall update}
                $\fwcost(\state_k, \act_k, \state_i) =
                \min \{
                    \fwcost(\state_k, \act_k, \state_i),
                    \fwcost(\state_k, \act_k, \state_t) + \min_{\act \in \Action}\fwcost(\state_t, \act, \state_i)
                \}$
                \;

                $\qValue(\state_k, \act_k) = \max \{
                        \qValue(\state_k, \act_k),
                        \max_{\act} \qValue(\state_i, \act) - \fwcost(\state_k, \act_k, \state_i)
                        \}$
                    \;

            }
        }
      }
      \KwResult{To follow the shortest path $\state_i$ to $\state_j$, follow the
        neighbors with highest $\qValue$\;
        $\policy(\state_k) = \arg \max_{\act_k \in \Action} \qValue(\state_k, \act_k)$\;
      }
      \caption{\small How to solve small windy grid world with randomized goals?}
      \label{alg:floyd-warshall-small}
  \end{algorithm}

\section{Experiments}
\begin{itemize}\item
Grid world: Set up a random goal static maze scenario, compare with normal Q-learning.
\item
Deepmind Lab: Set up a random goal static maze scenario, compare with normal Q-learning. 
\item
Atari games: Compare performance with normal Q-learning.
Analyze games in which FW does better.
Show that those games have dynamic goals rather than static.

\bibliography{main_filtered,main}
\bibliographystyle{aaai}
\end{document}
