\def\year{2019}\relax
%File: formatting-instruction.tex
\documentclass[letterpaper]{article} %DO NOT CHANGE THIS
\usepackage{aaai18}  %Required
\usepackage{times}  %Required
\usepackage{helvet}  %Required
\usepackage{courier}  %Required
\usepackage{url}  %Required
\usepackage{graphicx}  %Required
\usepackage{amsmath}
\usepackage{amsfonts}
\frenchspacing  %Required
\setlength{\pdfpagewidth}{8.5in}  %Required
\setlength{\pdfpageheight}{11in}  %Required
%PDF Info Is Required:
  \pdfinfo{
/Title (2018 Formatting Instructions for Authors Using LaTeX)
/Author (AAAI Press Staff)}
\setcounter{secnumdepth}{0}  

\usepackage[ruled,vlined]{algorithm2e}

\def\state{s}
\def\statet{\state_t}
\def\statetp{\state_{t-1}}
\def\statehist{\state_{1:t-1}}
\def\statetn{\state_{t+1}}
\def\obs{\meas}
\def\obst{\obs_t}
\def\act{a}
\def\actt{\act_t}
\def\acttp{\act_{t-1}}
\def\acttn{\act_{t+1}}
\def\Obs{\mathcal{O}}
\def\ObsEnc{\Phi_o}
\def\ObsProb{P_o}
\def\ObsFunc{C}
\def\ObsFuncFull{\ObsFunc(\statet, \actt) \rightarrow \obst}
\def\ObsFuncInv{\ObsFunc^{-1}}
\def\ObsFuncInvFull{\ObsFuncInv(\obst, \statetp, \actt) \rightarrow \statet}
\def\State{\mathcal{S}}
\def\Action{\mathcal{A}}
\def\TransP{P_{T}}
\def\Trans{T}
\def\TransFull{\Trans(\statet, \actt) \rightarrow \statetn}
\def\TransObs{T_c}
\def\Rew{R}
\def\rew{r}
\newcommand{\vect}[1]{\mathbf{#1}}
\def\rewards{\vect{r}_{1:t}}
\def\rewt{\rew_t}
\def\rewtp{\rew_{t-1}}
\def\rewtn{\rew_{t+1}}
\def\RewFull{\Rew(\statet, \actt) \rightarrow \rewtn}
\def\TransObsFull{\TransObs(\statet, \obst, \actt, \rewt; \theta_T) \rightarrow \statetn}
\def\Value{V}
\def\pit{\pi_t}
\def\piDef{\pi(\acttn|\statet, \obst, \actt, \rewt; \theta_\pi) \rightarrow \pit(\acttn ; \theta_\pi)}
\def\Valuet{\Value_t}
\def\ValueDef{\Value(\statet, \obst, \actt, \rewt; \theta_\Value) \rightarrow \Valuet(\theta_\Value)}
\def\R{\mathbb{R}}
\def\E{\mathbb{E}}
\newcommand{\meas}{z}
\newcommand{\measurements}{\vect{\meas}_{1:t}}
\newcommand{\meast}[1][t]{\meas_{#1}}
\newcommand{\param}{\theta}
\newcommand{\policy}{\chi}
\newcommand{\graph}{G}
\newcommand{\vtces}{V}
\newcommand{\edges}{E}
\newcommand{\state}[2]{\mathbf{s}^{#1}(#2)}
\newcommand{\egos}[1][t]{\state{c}{#1}}
\newcommand{\st}{\state}
\newcommand{\stn}{\st_{t+1}}
\newcommand{\stt}{\st_t}
\newcommand{\stk}{\st_k}
\newcommand{\stj}{\st_j}
\newcommand{\sti}{\st_i}
\newcommand{\St}{\mathcal{S}}
\newcommand{\Act}{\mathcal{A}}
\newcommand{\acti}{\act_i}
\newcommand{\actt}{\act_t}
\newcommand{\lpt}{\delta}
\newcommand{\trans}{P_T}
\newcommand{\Q}{\qValue}
\newcommand{\V}{V}
\newcommand{\fw}{\fwcost}

\newcommand{\fwcost}{F}
\newcommand{\qValue}{Q}
\newcommand{\prew}{\Upsilon}
\newcommand{\epiT}{T}
\newcommand{\vma}{\alpha_\Value}
\newcommand{\qma}{\alpha_\qValue}
\newcommand{\prewma}{\alpha_\prew}
\newcommand{\fwma}{\alpha_\fwcost}
\newcommand{\maxValueBeam}{\vect{\state}_{\Value:\text{max}(m)}}
\newcommand{\nil}{\emptyset}
\newcommand{\discount}{\gamma}
\newcommand{\minedgecost}{\fwcost_0}
\begin{document}
% The file aaai.sty is the style file for AAAI Press 
% proceedings, working notes, and technical reports.
%
\title{Floyd-Warshall RL: Combining model-free and model-based algorithms}
\author{Vikas Dhiman}
\maketitle
\begin{abstract}
The Reinforcement learning algorithms are classified on the basis of whether
the state-transition model is learned explicitly into \emph{model-based}
and \emph{model-free} algorithms.
%
The \emph{model-free} algorithms like Q-learning and policy gradients are easier to learn in cases when model is more complex than the policy.
However, model-free algorithms are harder to transfer in cases when the reward function changes while the state-transition model remains the same.
The \emph{model-based} algorithms learn the state-transition model explicitly hence making it
easier to separate the environment transition from the reward distribution.
However, model-based algorithms require an additional planning step on the model dynamics
which make it expensive to compute the policy.
%
We propose a novel algorithm that is based on a combination of model-free and model-based approaches.
Due to it's similarity to Floyd-Warshall algorithm in path-planning on graphs, we call
it Floyd-Warshall reinforcement learning (FWRL).
We show (DONE) that FWRL transfers the knowledge about the environment when the reward location is dynamic as compared to the model-based Q-learning algorithm.
We hypothesize (TODO: show with our experiments) that FWRL meets the ground between model-free and model-based algorithms by being model-free in the more frequently visited regions while being model-based in less visited paths.
\end{abstract}


\section{ Introduction}

The Reinforcement learning algorithms are classified on the basis of whether
the state-transition model is learned explicitly into \emph{model-based}
and \emph{model-free} algorithms.
%
The \emph{model-free} algorithms like Q-learning and policy gradients are easier to learn in cases when model is more complex than the policy.
However, model-free algorithms are harder to transfer in cases when the reward function changes while the state-transition model remains the same.
One example of such a problem is traveling salesman problem.
Although classical traveling salesman problem is a posed as a planning problem where the space has already been explored, however it is not hard to recognize that someone has to create a map of the map through exploration and sometimes it has to be the agent them-self.
In such a scenario, the traveling salesman has a dynamic reward with a static map that needs to be
explored.
The model-free approach fails to generalize to new rewards in these scenarios, because of the conflated representation of environment model and resulting reward.
In spite of these limitations, model-free approaches have been applied to multi-goal navigation
scenarios with considerable success \cite{mirowski2018learning,schaul2015universal}.
However, all these methods depend upon exploring the entire space being a candidate goal space.
Not only model-free approaches are sample inefficient, multi-goal navigation problem exaggerates this problem.

The \emph{model-based} algorithms learn the state-transition model explicitly hence making it
easier to separate the environment transition from the reward distribution. Also model-based
algorithms are known to be sample efficient than model-free algorithms especially in cases when
the state-transition model is translation invariant in the environment. 
However, model-based algorithms require an additional planning step on the model dynamics
which make it expensive to compute the policy on the fly.

Taking inspiration from the Floyd-Warshall algorithm in graph-based path planning domain, we propose Floyd-Warshall reinforcement learning (FWRL) algorithm that combines the benefits of both
model-based and model-free methods.
We define the Floyd-Warshall function $F$ between two states $\state_i$, $\state_j$ as the maximum reward that can be gained by following a between a two states.
\begin{align}
F(\state_i, \act_i, \state_j) = \E_{\pi}\left[
\sum_{k \in p_{ij} } \discount^{t} r_{k} \right| \state_i, \act_i, \state_j\Biggr],
\end{align}
where $p_{ij}$ is a possible path from $\state_i, \act_i$ to $\state_j$ while following policy $\pi$. For an optimal policy $
F^*(\state_i, \act_i, \state_j) = \max_{p_{ij}} \sum_{k \in p_{ij} } \discount^{t} r_{k}$.

In terms of Q-function and Value function the Floyd-Warshall function has the following relationships
\begin{align}
V(\state_i) &= \E_{\act}[Q(\state_i, \act)|\state_i] \\
F(\state_i, \act_i, \state_j) &= V(\state_j) - Q(\state_i, \act_i).
\end{align}
For the optimal policy the expectation is replaced by the max operator, that is $F^*(\state_i, \act_i, \state_j) = \max_{\act} Q^*(\state_j, \act) - Q^*(\state_i, \act_i)$.

Due to this formulation, the optimal Floyd-Warshall function should satisfy the transitive property
\begin{align}
F^*(\state_i, \act_i, \state_j) = \max_{\state_k} F^*(\state_i, \act_i, \state_k)
+ \max_{\act_k}F^*(\state_k, \act_k, \state_j)
\label{eq:transitive-fw}
\end{align}

The advantage of this formulation is that as soon as the goal state $\state_g$, the Q-function and
hence the policy can be estimated from $Q(\state, \act) = F(\state, \act, \state_g) + \Rew(\state_g)$.

In our experiments, we show how our approach is better than both model-free Q-learning and model-based approaches. We also show how our approach can be switch between either approaches depending upon traversal experience in the state space.

It is clear that another equivalent way of transferring knowledge from Q-function is to keep it's derivative and transfer that instead of transferring the Q-function.

\section{Related work}

\subsection{Universal value function approximators and derivatives}
\cite{schaul2015universal} proposed an architecture and a matrix factorization based algorithm for faster learning of UVFA (Universal value function approximators).

\cite{pong2018temporal} propose temporal difference models that estimate goal directed Q function for a specific kind of reward function, in particular the distance from the goal and in contrast with limited temporal horizon.
They do not use the transitive property of goal directed value functions.

\subsection{Claims}
\begin{itemize} \item
Using FW is better than Q function in terms of generalizing to dynamic goals, static maps.
\item 
Using Floyd Warshall value function leads to better generalization in case of static maps and
random goals.
\item
Hypothesis: Multi-goal navigation is more common than we think. Does FW algo improves performance in attari games.
\end{itemize}

\section{Related work}
\subsection{Navigation with mapping}
 (1) CMP from Saurabh Gupta: is metric, might not working in continuous spaces.
 (2) Semi-parameteric Topological mapping: is not end to end.
 (3) Neural Map: Is actually not mapping

\subsection{Model free DRL }
does not generalize to multi-goal environments.

\subsection{Model based DRL}
Needs more exploration.
Find the paper that shows that Model based DRL can actually compete with Model free DRL as long as it models uncertainty.

\subsection{Multi-goal navigation based papers}
Mirowski 2017, 2018: No one shot map learning, does not generalizes to new maps.


\section{Method}

See Alg~\ref{alg:floyd-warshall-small}
Over simplified. Ignoring the cost of going through the entire state space.
    \begin{algorithm}
      \KwData{Graph $\graph_0 = (\vtces, \edges)$\;}
      Initialize $\fwcost(\state_i, \act_i, \state_j; \param_{\fwcost}) = 100$ \;
      Initialize $\qValue(\state_i,\act_i; \param_{\qValue}) = 1$ \;
      Initialize $\vma = 0.1$, $\prewma = 0.9$ \;
      Let minimum path cost $\minedgecost = 0.05$ \;
      Observe $\meas_0$ from environment \;
      $\state_0 = \ObsEnc(\meas_0; \param_E)$ \;
      \For{$t \leftarrow 1$ \KwTo $\epiT$}{
        Take action $\act_{t-1}$\;
        Observe $\meas_t$, $\rew_t$\;
        Encode state $\state_t = \ObsEnc(\measurements; \param_E)$\;
        \tcc{Initialize new FW values}
        $\fwcost(\state, \act, \state_{t})
        = \min \{
               \fwcost(\state, \act, \state_t),
                \fwcost(\state, \act, \state_{t-1}) + \minedgecost
            \}
        \qquad \forall \state \in \State, \act \in \Action$ \;
        \tcc{Q-Value update}
        $\qValue(\state_{t-1}, \act_{t-1}) = (1-\qma) (\rew_t + \discount \max_{\act_k}\qValue(\state_t, \act_k)) + \qma \qValue(\state_{t-1}, \act_{t-1})$\;
        \If{$\state_t$ is visited the first time}{
            \For{$(\state_i, \state_k, \act_k) \in (\State \times \State \times \Action)$}{
                \tcc{Run the Floyd Warshall update}
                $\fwcost(\state_k, \act_k, \state_i) =
                \min \{
                    \fwcost(\state_k, \act_k, \state_i),
                    \fwcost(\state_k, \act_k, \state_t) + \min_{\act \in \Action}\fwcost(\state_t, \act, \state_i)
                \}$
                \;

                $\qValue(\state_k, \act_k) = \max \{
                        \qValue(\state_k, \act_k),
                        \max_{\act} \qValue(\state_i, \act) - \fwcost(\state_k, \act_k, \state_i)
                        \}$
                    \;

            }
        }
      }
      \KwResult{To follow the shortest path $\state_i$ to $\state_j$, follow the
        neighbors with highest $\qValue$\;
        $\policy(\state_k) = \arg \max_{\act_k \in \Action} \qValue(\state_k, \act_k)$\;
      }
      \caption{\small How to solve small windy grid world with randomized goals?}
      \label{alg:floyd-warshall-small}
  \end{algorithm}

\section{Experiments}
\begin{itemize}\item
Grid world: Set up a random goal static maze scenario, compare with normal Q-learning.
\item
Deepmind Lab: Set up a random goal static maze scenario, compare with normal Q-learning. 
\item
Atari games: Compare performance with normal Q-learning.
Analyze games in which FW does better.
Show that those games have dynamic goals rather than static.
\end{itemize}
\def\localbib{/home/dhiman/wrk/group-bib/shared.bib}
\IfFileExists{\localbib}{\bibliography{\localbib}}{}
\bibliography{main_filtered,main}
\bibliographystyle{aaai}
\end{document}
