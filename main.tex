%File: formatting-instruction.tex
\documentclass[letterpaper]{article} %DO NOT CHANGE THIS
\usepackage{iclr2019_conference}  %Required
\usepackage{times}  %Required
\usepackage{helvet}  %Required
\usepackage{courier}  %Required
\usepackage{url}  %Required
\usepackage{graphicx}  %Required
\usepackage{hyperref}
\usepackage{amsmath}
\usepackage{amsfonts}
\usepackage{amssymb}
\usepackage{natbib}
\usepackage{xcolor}
\usepackage{subfig}
\usepackage{booktabs}
\frenchspacing  %Required
\setlength{\pdfpagewidth}{8.5in}  %Required
\setlength{\pdfpageheight}{11in}  %Required
%PDF Info Is Required:
\setcounter{secnumdepth}{2}  

\usepackage[ruled,vlined]{algorithm2e}

\input{preamble}

% The file aaai.sty is the style file for AAAI Press 
% proceedings, working notes, and technical reports.
%
%\title{Do Goal-Conditioned Value Functions need Goal Rewards to Learn?}
\title{Learning Goal-Conditioned Value Functions without Goal Rewards}

\author{Anonymous}
%\author{Vikas Dhiman$^1$, Shurjo Banerjee$^1$, Jeffrey M. Siskind$^2$ and Jason J.
%Corso$^1$\\
%The University of Michigan$^1$\\
%Purdue University$^2$}

\pdfinfo{
/Title ()
/Author ()}
\begin{document}

\maketitle
\begin{abstract}
    Multi-goal reinforcement learning addresses the tasks where goal
    specifications are required. State-of-the-art methods in this field,
    utilize goal-conditioned value functions in their operation.
    Estimating these functions operate on the assumption that the
    achievement of goal states are tied with large reward. Our first
    contribution is the redefinition of these functions as expected
    cumulative path rewards that allows for equally efficient learning
    in the absence of these goal rewards. This formulation of the
    goal-less learning of goal-conditioned value functions obviates the
    requirement of reward-recomputation that is needed by state-of-the-art MGRL
    algorithms. This futher leads to substantially improved reward
    sample complexity which is our second contribution. Our third
    contribution is the extension of the Floyd-Warshall Reinforcement
    Learning algorithm from tabular domains to deep neural networks. 
\end{abstract}


\section{Introduction}

% Reinforcement learning and it's uses
% Why we care about Multi-Goal Reinforcement Learning
Many tasks in robotics require the specification of a \emph{goal} for every
trial. For example, a robotic arm may need to move an object to an arbitrary
goal position on a table \citep{gu2017deep}; a mobile robot may need to with
navigating to a arbitrary goal landmark on a map \citep{zhu2017target}. The
adaptation of reinforcement learning to such goal-conditioned tasks where goal
locations can change is called Multi-Goal Reinforcement Learning
(MGRL)~\citep{plappert2018multi}.

% A common way to solve MGRL problems is the use of Goal-Conditioned
% Value functions. Solving several tasks instead of individual tasks
State-of-the-art MGRL algorithms
\citep{andrychowicz2017hindsight, pong2018temporal}
work by estimating \emph{goal-conditioned value functions} (GCVF). A
GCVF is defined as the expected cumulative reward from a
start state with a specified the goal. It is used to compute the
actions to be take at any state, also known as the \emph{policy}.

% Goal conditioned value functions make a big assumption however. It is
% assumed that to learn to achieve goals, goal locations must themselves
% must be associated with reward. This however is not the case. We
% introduce ....
These algorithms require the use of \emph{goal-rewards}, which is the
dependence of reward functions on the desired goal, $r(s,a,g)$.
For example, in the Fetch-Push task \citep{plappert2018multi} of moving
a block to a given location on a table, every movement incurs a ``-1''
reward while reaching the desired goal induces ``0'' reward. Although
this dependence allows for the agent to associate reaching the goal with a high
reward, it may incur additional sampling costs. For example, in Hindsight Experience
Replay (HER) \citep{andrychowicz2017hindsight}, failed experiences are
re-evaluated as succesful ones by assuming traversed states to be
desired pseudo-goals. Due to the dependence of reward function on the goal,
the association of every pseudo-goal with high reward requires an independent
\emph{reward-recomputation} step. 

%This leads to the
%requirement of \emph{reward-recomputation}, which is that these reward functions can be
%computed indendent of \emph{environment} interaction.  The second assumption enables a technique called \emph{Hindsight
%Experience Replay} (HER) that accelerates learning 

% Do goal-conditioned value functions require goal rewards to learn?
% Describe some form of the intuition
The goal-rewards requirement is avoidable. Let us consider an example to
motivate why goal-rewards may be avoided in goal-conditioned tasks. Consider a
student who has moved to a new university. To learn about the campus,
the student explores it randomly with no specific goal in mind. When
tasked with finding a goal classroom, the student can then use the
learned connectivity of the campus to make their way to class along the
shortest explored path. The key intuition here is that the
student is not incentivized to explore specific goal locations (i.e. no
goal rewards). 
%
Based on this intuition of goal-less learning, we redefine GCVFs to be
the expected \emph{path-reward} that is learned for all possible
start-goal pairs. Instead of goal rewards, we  
introduce a \emph{one-step loss} that assumes single steps paths 
to be the maximum reward paths between adjacent pairs.
Under this interpretation, the \emph{Bellman equation} chooses and chains
together one-step paths to find longer maximum reward paths. 

%
Experimentally, we show how this simple reinterpration, which does not use goal
rewards, performs as well as a representative state-of-the-art method (HER) and
even outperforms it in terms of reward sample complexity.

% Detail the Floyd-Warshall contribution
In this work, we also extend a closely related algorithm, Floyd-Warshall Reinforcement
Learning (FWRL) \citep{dhiman2018floydwarshall} to use parametric
function approximators instead of tabular functions. Similar to our
re-definition of GCVFs, FWRL learn a goal-conditioned Floyd-Warshall function
that represents path-rewards instead of future-rewards.
We translate FWRL's compositionality constraints in the space of GCVFs to introduce
additional loss terms to the objective. However, these additional loss
terms do not show improvement over the baseline. We conjecture that the
compositionality constraints are already captured by other loss terms. 

% Summarize 
In summary, the  contributions of this work are twofold. Firstly, we
reinterpret goal-conditioned value functions as expected path-rewards
and introduce one-step loss thereby removing their dependency on
goal-rewards and reward-resampling. We showcase how the
application of this interpreation leads to improved sample efficiency in  terms
of reward samples.
Secondly, we extend the tabular Floyd-Warshal Reinforcement Learning to
use deep neural networks.


%% There is inherent structure in the formulation of these functions
%% thathas been ignored. Dhiman et al introduced structure in the space
%% of these functions 
%Floyd-Warshall Reinforcement Learning was first introduced in Dhiman et.
%al. for tabular domains. 
%The FW functions is defined 
%%
%\begin{align}
%\fwargs{\state}{\act}{\state'}{\policy}{} =
%\E_{\policy}\left[ \sum_{t=0}^{t=k} \rew_t \middle\vert \state_0 = \state, \act_0 = \state, \state_k = \state' \right] .
%\end{align}%
%%
%which is the expected sum of rewards when the state is s and the
%end state is the goal. Compositionality constraints are then used to
%learn the FW function.
%%
%\begin{align}
%\fwargs{\state_i}{\act_i}{\state_j}{\policy^*_{\state_j}}{*}
% \ge 
%  \fwargs{\state_i}{\act_i}{\state_k}{\policy^*_{\state_k}}{*}
%  + \max_{\act_k}\fwargs{\state_k}{\act_k}{\state_j}{\policy^*_{\state_j}}{*}
%  ,\forall \state_k \in \State.
%  \label{eq:fwconstraint}
%\end{align}%
%%
%The constraint is inspired by the Floyd-Warshall algorithm for path
%planning on graphs and states that the expected reward in traversing
%from one state to another should be greater than or equal to that of the
%paths through intermediary states.  While Dhiman et al. was restritcted
%to toy examples in the gridworld doman and tabular functions, they are
%not naturally scalabale to real life problems. This work extends their
%formulation to paramteric functions like neural networks by addition
%terms to the RL objective based on these compositionality constraints.
%Since we extend the algorithm to the Deep Learning domain, we call it
%Deep Floyd-Warshall Reinforcement Learning. 








\section{Background}

We present a short review of the background material that our work depends upon.

\subsection{Dijkstra}
\newcommand{\vertices}{\State}
\newcommand{\edge}{\rew}
% Floyd warshall data structure
\newcommand{\fwds}{D}
% Dijkstra data structure
\newcommand{\dds}{D}
% Q-function
Dijkstra~\citep{dijkstra1959note} is a shortest path finding algorithm from a
given vertex in the graph. Consider a weighted graph $G = (\vertices,
\edges)$, with $\vertices$ as the vertices and $\edges$ as the edges. Dijkstra
algorithms works by maintaining a data-structure $\dds : \vertices \rightarrow
\R$, that represents the shortest path length from the source. The data
structure $\dds$ is initialized with zero at start location $\dds[\state_0]
\leftarrow 0$ and a high value everywhere else $\dds[i] \leftarrow \infty \, \forall
i \in \vertices$. The algorithm then sequentially updates $\dds$ by
%
\begin{align}
  \dds[j] \leftarrow \min\{\dds[j], \edge_{(i, j)} + \dds[i] \} \, \forall (i, j) \in \edges ,
  \label{eq:dijkstra}
\end{align}%
%
where $\edge_{(i, j)}$ is the edge-weight for directed edge $(i, j) \in \edges$.
The shortest path $(\state_0, \state_{1}, \dots)$ starting from vertex
$\state_0$ can be read from $\dds$ via $\state_{t+1} = \arg \min_{i \in
\text{Nbr}(\state_t)} \dds[i]$ where $\text{Nbr}(\state_t) = \{ i | (i,
\state_t) \in \edges \} $ denotes the neighborhood of $\state_t$. With a
carefully chosen data-structure and traversal order, the Dijkstra Algorithm can
be made to run in $O(|\vertices|\log|\vertices|)$.

\subsection{Q-Learning}
Q-learning~\citep{watkins1992qlearning} is a reinforcement learning (RL)
algorithm that allows agent to explore environment and simultaneously
compute maximum reward paths.

An RL problem is formalized as an Markov Decision Process (MDP). A MDP is
defined by a four tuple $(\State, \Action, \Trans, \Rew)$, where $\State$ is the
state space, $\Action$ is the action space, $\Trans : \State \times \Action
\rightarrow \State$ is the system dynamics and $\Rew : \State
\rightarrow \R $ is the reward yielded on a execution of an action.
The objective of a typical RL problem is to maximize the expected cumulative
reward over time, called the returns  $ R_t = \sum_{t^{'}=t}^{T} \rew_{t^{'}}$.

Q-learning works by maintaining an action-value function $\Q : \State \times
\Action \rightarrow \R$ which is defined as the expected return
$\Q_\policy(\state_t, \act_t) = \E_\policy[R_t]$ from a given state-action pair.
The Q-learning algorithm works by updating the $\Q$-function using the Bellman
equation for every transition from $\state$ to $\state'$ on action $\act$
yielding reward $\rew$, 
%
\begin{align}
  \Q^*(\state, \act) &= \E_{\policy}\left[
                       \rew + \max_{\act^{'}} \Q^*(\state^{'}, \act^{'})
                       \middle| \state, \act \right] \, .
    \label{eq:q-learn-bellman}
\end{align}%
%

\subsection{Floyd-Warshall}

The Floyd-Warshall algorithm~\citep{floydwarshall1962} is a shortest path finding
algorithm from any vertex to any other vertex in a graph.
Similar to Dijkstra's algorithm, the Floyd-Warshall algorithm
finds the shortest path by keeping maintaining a shortest distance
data-structure $\fwds : \vertices \times \vertices \rightarrow \R$. between any
two pair of vertices $i, j \in \vertices$.
The data-structure $\fwds$ is initialized with edges weights
$\fwds[i, j] \leftarrow \edge_{(i, j)} \, \forall (i, j) \in \edges$
and the uninitialized edges are assigned a high value,
$\fwds[i, j] \leftarrow \infty \, \forall i, j \in \vertices$.
The algorithm works by sequentially observing all the nodes in the graph and
updating $\fwds$ with the shortest explored path known so far:
%
\begin{align}
  \fwds[i, j] \leftarrow \min\{ \fwds[i, j], \fwds[i, k] + \fwds[k, j] \} \quad
  \forall i, j, k \in \vertices \, .
\end{align}%
%


The update equation in the algorithm depends upon the triangular inequality for
shortest paths distances ($\fwds[i, j] \le \fwds[i, k] + \fwds[k, j]$) and hence
works only in the absence of negative cycles in the graph. Fortunately, many
practical problems can be formulated such that negative cycles
do not occur. The Floyd-Warshall algorithm runs in $O(|\vertices|^3)$
and is suitable for dense graphs. There also exists extensions of the algorithm
like Johnson's algorithm~\citep{johnson1977efficient} that run in
$O(|\vertices|^2\log|\vertices| + |\vertices||\edges|)$ while working on
the same principle.

% DONE: Add more math (?) that draws these parallels.
From the parallels between Eq.~\eqref{eq:dijkstra} and
Eq.~\eqref{eq:q-learn-bellman}, Q-learning can be seen as a generalization
Dijkstra's algorithm. Both the algorithms work by taking one step minimum (or
maximum) over the neighboring state. Unlike Dijkstra, in Q-learning one has to
compute an additional maximum over actions. This is because in an MDP, the agent
cannot directly choose the next state to be in. Instead, it chooses an action
that leads it to the next state based on transition probabilities. Moreover,
Q-learning has to explore the state space before it can exploit the learned
information to find most-rewarding path. With these parallels in mind, we
generalize the Floyd-Warshall algorithm to work on an MDP and call it
Floyd-Warshall Reinforcement Learning.



\section{Problem definition}

%\subsection{Environment Setup}
Consider an agent interacting with an environment, $\varepsilon$. At
every time step, $t$, the agent observes a state, $\state_t \in \State$,
where $\State$ is the observation state space. The agent can traverse
the state space by taking actions, $\act \in \Act$, where $\Act$ is a
fixed action space. A goal state, $\goal \in \State$, is specified to
the agent where the goal state is a specific observation in the state
space.  $\Rgoal$ is the reward recieved by the agent for finding the
goal state and constitutes the largest reward in the environment.  For
every time step $t$, the agent takes an action $\act_t$, observes a
state $\state_t \in \State$ and receives a reward $\rew_t \in [-\Rgoal,
\Rgoal]$.  Episodes are of a fixed number of time steps, $T$. For every
episode, a randomized goal state is provided to the agent as input. If
the goal state is observed by the agent during the course of an episode,
the agent is randomly reinitialized within the environment while the
goal state remains unchanged.

As is typical in RL domains, the agent's objective is to find the
sequence of actions to take that maximizes the total reward from episode
to episode. Since the environment itself is static, this is best
achieved via the agent first discovering the goal location and then
traversing the shortest path to it from every subsequent \emph{spawn}
state during the course of an episode. The agent is best suited by
algorithms that emphasize the  transfer \emph{environment structure}
from episode to episode. 


%Once the agent reaches the goal $\|\state_t - \goal\| <
%\delta$

%
%\begin{align}
%\policy^*(\state_t ; \goal) = \act^*_t = \arg \max_{\act_t} \E_{\policy}\left[ \sum_{t=0}^T \rew_t \right]
%\end{align}%
%

%\subsection{Why is this problem important?}
Many real world problems can be formulated in this context. Consider a robot
who has moved into a new city.
The salesman has to explore the city and find the buildings that match the given
address. The next time the postman gets the same address, they can use their
experience to find out the building. Even when a new address is provided in the
next episode, the postman can use experience to find the new address in shorter
time.

In robotics, tasks like picking and placing the object at a desired
location can be formulated as goal-directed navigation.

%\subsection{Why is the problem hard?}
Model-free Reinforcement learning methods assume that the rewards are
being sampled from the a static reward function.  In a problem where the
goal location changes, hence the reward function also changes, it
becomes hard to transfer the learned value-function or action-value
function to the changed location.  One alternative is to concatenate the
goal location the state, making the new state space $[\state_t,
\goal]^\top \in \State^2$ larger.  This method is wasteful in
computation and more importantly in sample complexity.

\section{Method}
We present a model-free reinforcement learning method that easily transfers the
learned behavior when goal location is dynamic. We accomplish this by
maintaining a path based expected reward function from any state to any goal
state. We call this algorithm Floyd-Warshall Reinforcement Learning, because of
its similarity to Floyd-Warshall algorithm : a shortest-path planning algorithm
on graphs. We define Floyd-Warshall value function (FW) as
%
\begin{align}
\fwargs{\state}{\act}{\state'}{\policy}{} =
\E_{\policy}\left[ \sum_{t=0}^{t=k} \rew_t \middle\vert \state_0 = \state, \act_0 = \state, \state_k = \state' \right] ,
\end{align}%
%
where $\policy: \State \times \Action \rightarrow \Delta$ is the
stochastic policy being followed.

What is the relationship between Q-function and FW-function?
\begin{align}
  \Q_\policy(\state, \act) = \sum_{\state'} P_\policy(\state' | \state, \act) \fwargs{\state}{\act}{\state'}{\policy}{},
\end{align}%
%
where $P_\policy(\state' | \state, \act)$ is the probability of the agent
arriving at $\state'$ within the episode.

The optimal FW-function is defined as
\begin{align}
\fwargs{\state}{\act}{\state'}{\policy^*_{\state'}}{*} =
\max_{\policy_{\state'}}\E_{\policy_{\state'}}\left[ \sum_{t=0}^{t=k} \rew_t \middle\vert \state_0 = \state, \act_0 = \state, \state_k = \state' \right] ,
\end{align}%
where $\policy^*_{\state'}: \State \times \Action \rightarrow \Delta$ is the
optimal policy towards the goal state $\state'$. The
optimal Q-function and FW-function are equal same goal
  $\Q^*_{\policy^*_{\state_g}}(\state, \act) =
  \fwargs{\state}{\act}{\state_g}{\policy^*_{\state_g}}{*}$, as long as they are
  following the policy towards the same goal.

When the policy is optimal, the Floyd-Warshall function must satisfy the constraint
%
\begin{align}
\fwargs{\state_i}{\act_i}{\state_j}{\policy^*_{\state_j}}{*}
 = \sup_{\state_k} \left[
  \fwargs{\state_i}{\act_i}{\state_k}{\policy^*_{\state_k}}{*}
  + \max_{\act_k}\fwargs{\state_k}{\act_k}{\state_j}{\policy^*_{\state_j}}{*} \right] ,
\end{align}%
%
which states that the optimal FW function from a given start state to a
given goal state should be greater than or equal to the summation of FW
function via any intermediate state. 

We summarize the algorithm in Alg~\ref{alg:floyd-warshall-small}.

\TODO{Should we use Q-function as backdrop?}


\begin{algorithm}
  Let $\rew_g \leftarrow 10$\;
  \tcc{By default all states are unreachable}
  Initialize $\fwcost(\state_i, \act_i, \state_j; \param_{\fwcost}) \leftarrow -\infty$ \;
  Initialize $Q(\state_i, \act_i) \leftarrow 1$ \;
  Initialize $\state_g = \phi$ \;
  Set $t \leftarrow 0$\;
  Observe $\meas_t$ \;
  $\state_t = \ObsEnc(\meas_t; \param_E)$ \;
  \For{$t \leftarrow 1$ \KwTo $\epiT$}{
  \tcc{See Function~\ref{func:policy}}
    Take action $\act_{t-1} \sim \text{Egreedy}(\policy^*(\state_{t-1}, \state_g, Q, \fwcost))$ \;
    Observe $\meas_t$, $\rew_t$ \;
    $\state_t = \ObsEnc(\meas_t; \param_E)$ \;
    \If{$\rew_t >= \rew_g$}{
      \tcc{Reached the goal}
      $\state_g \leftarrow \state_t$ \;
      \tcc{Respawning does not need update of value functions}
      continue\;
    }
    $Q(\state_{t-1}, \act_{t-1}) \leftarrow \rew_t + \max_{\act} Q(\state_t, \act)$ \;
    $\fwcost(\state_{t-1}, \act_{t-1}, \state_t) \leftarrow \rew_t$ \;
    \For{$\state_k \in \State, \act_k \in \Act, \state_l \in \State$}{
      $\fwcost(\state_k, \act_k, \state_l) \leftarrow
        \max \{
        \fwcost(\state_k, \act_k, \state_l),
        \fwcost(\state_k, \act_k, \state_t)
        + \max_{\act_p \in \Act} \fwcost(\state_t, \act_p, \state_l)
        \}$
        \;
    }
  }
  \KwResult{$\policy^*(\state_k, \state_g, Q, \fwcost)$}
  \caption{\small Floyd-Warshall Reinforcement Learning (Tabular setting)}
  \label{alg:floyd-warshall-small}
\end{algorithm}


%\begin{function}
%\eIf{$\state_g = \phi$ or $\text{all}(\fwcost(\state_t, :, \state_g) = -\infty)$ }{
%  \tcc{Exploration mode}
%  $\act^* = \arg\max_{\act} Q(\state_t, \act)$\;
%}{
%  \tcc{Exploitation mode}
%  $\act^* = \arg\max_{\act} F(\state_t, \act, \state_g)$\;
%}
%
%\caption{Policy()}%$\policy^*(\state_t, \state_g, Q(., .), \fwcost(.,.,.))$}
%\label{func:policy}
%\KwRet{$\act^*$}
%\end{function}




\section{Experiments}
We setup two environments in the grid world and windy grid world.
\subsection{Four room grid world}

Four room grid world is a grid world with four rooms connected to each other as shown in Figure~\ref{fig:four-room-grid-world}.

\subsection{Four room windy world}

Four room windy world is a grid world with four rooms connected to each other as shown in Figure~\ref{fig:four-room-grid-world}.
Some of the grid cells in have wind shown by arrow and the agent gets pushed around by
the wind with 0.25 probability irrespective of the action taken.

\subsection{Metrics}
We describe our metrics.

\subsubsection{Latency-1:1}

\subsubsection{Distance Inefficiency}


%
\begin{figure}%
\includegraphics[width=0.48\columnwidth]{media/4-room-grid-world.pdf}
\hfill
\includegraphics[width=0.48\columnwidth]{media/4-room-windy-world.pdf}%
\caption{Left: Four room grid world. Right: Four room windy grid world with wind direction shown by arrows. The windy pushes the agent in the direction of wind with 0.25 probability irrespective of the action taken.}
\label{fig:four-room-grid-world}%
\end{figure}%



\section{Results}

\subsection{Quantitative Results}
We evaluate Q-learning (QL), model-based RL (MBRL) and Floyd-Warshall
Reinforcement Learning (FWRL) on two metrics in two different environments. The
two metrics we use are Latency Ratio and average reward per episode. The Latency
ratio metric was introduced in \citet{MiPaViICLR2017}, which is defined as the
ratio of time taken to reach the goal for the first to time to the average time
taken to hit the goal thereafter. The Latency ratio thus measures the ratio of
exploration time for first time finding the goal to the average exploitation
time to reach the goal. Hence, higher latency ratio is better.
Fig~\ref{fig:ql-fw-grid-world-results} and
Fig~\ref{fig:ql-fw-windy-world-results} show the results.

\begin{figure}%
    \begin{subfigure}
        \includegraphics[width=\columnwidth]{./media/metrics-grid-world.pdf}{a}
        \caption{Results on grid world. FWRL beats Q-Learning
        consistently. Lower is better for Distance-Inefficiency. Higher
        is better for reward per episode. }
    \end{subfigure}
    \begin{subfigure}
        \includegraphics[width=\columnwidth]{./media/rewards-metrics-grid-world.pdf}{b}
        \caption{Reward curves on grid world. FWRL reward climbs much
        faster than all other baselines showcasing the improved \emph{sample
        efficiency} of the algorithm.}
    \end{subfigure}
    \label{fig:ql-fw-grid-world-results}%
\end{figure}


\begin{figure}
    \begin{subfigure}
        \includegraphics[width=\columnwidth]{./media/metrics-windy-world.pdf}
        \caption{Results on windy world. FWRL beats Q-Learning
        consistently. Lower is better for Distance-Inefficiency. Higher
        is better for reward per episode. }
    \end{subfigure}
    \begin{subfigure}
        \includegraphics[width=\columnwidth]{./media/rewards-metrics-windy-world.pdf}
        \caption{Reward curves on windy world. FWRL reward climbs much
        faster than all other baselines showcasing the improved \emph{sample
        efficiency} of the algorithm.}
    \end{subfigure}
    \label{fig:ql-fw-windy-world-results}%
\end{figure}


\subsection{Qualitative Results}


\section{Analysis}

\paragraph{Is the step-loss really needed?}
%
%
\begin{figure}%
  \def\frac{0.24}
  % Ours with and without step loss
  \includegraphics[width=\frac\columnwidth]{media/res/ablate-ddpg-with-without-step-loss/FetchPush-6efc1de-ddpgepoch-test/ag_g_dist.pdf}%
  \includegraphics[width=\frac\columnwidth]{media/res/ablate-ddpg-with-without-step-loss/FetchPush-6efc1de-ddpgepoch-test/success_rate.pdf}%
  % Ours with and without goal rewards
  \includegraphics[width=\frac\columnwidth]{media/res/ablate-ours-with-goal-reward/FetchPickAndPlace-dqstepoch-test/success_rate.pdf}%
  \includegraphics[width=\frac\columnwidth]{media/res/ablate-ours-with-goal-reward/FetchPickAndPlace-dqstepoch-test/ag_g_dist.pdf}%
  \label{fig:with-and-without-step-loss}%
  \caption{What if we remove goal reward from DDPG
  What if we provide goal-rewards to our algorithm? Nothing much}%
\end{figure}%
% 



\paragraph{Effects of different distance thresholds}
%
\begin{figure}%
  \def\frac{0.24}
  \includegraphics[width=\frac\columnwidth]{media/res/ablate-ddpg-dqst-low_tresh_chosen-low_thresh_alt-ddpg/0.05-be0910creward_computes-test/success_rate.pdf}%
  \includegraphics[width=\frac\columnwidth]{media/res/ablate-ddpg-dqst-low_tresh_chosen-low_thresh_alt-ddpg/0.05-be0910creward_computes-test/ag_g_dist.pdf}%
  \includegraphics[width=\frac\columnwidth]{media/res/ablate-ddpg-dqst-low_tresh_chosen-low_thresh_alt-dqst/0.001-FetchPushPR-be467dfepoch-test/success_rate.pdf}%
  \includegraphics[width=\frac\columnwidth]{media/res/ablate-ddpg-dqst-low_tresh_chosen-low_thresh_alt-dqst/0.001-FetchPushPR-be467dfepoch-test/ag_g_dist.pdf}%
  \label{fig:with-different-distance-thresholds}%
  \caption{Effects of distance threshold on DDPG and our experiment.}%
\end{figure}%
% 





\section{Related Work}


\subsection{DDPG}


\subsection{Hindsight Experience Replay}


\subsection{Temporal Difference Models}


\subsection{Conclusion}
Floyd-Warshall Reinforcement Learning (FWRL) allows us to learn a goal
conditioned action-value function which is invariant to the change in goal
location as compared to the action-value function used in typical Q-learning.
This allows FWRL to transfer learned behaviors about the environment when the
goal location changes. Many tasks like navigation, robotic pick and place are
examples of goal-conditioned tasks that can benefit from this framework.


%\begin{algorithm}
  Let $\rew_g \leftarrow 10$\;
  \tcc{By default all states are unreachable}
  Initialize $\fwcost(\state_i, \act_i, \state_j; \param_{\fwcost}) \leftarrow -\infty$ \;
  Initialize $Q(\state_i, \act_i) \leftarrow 1$ \;
  Initialize $\state_g = \phi$ \;
  Set $t \leftarrow 0$\;
  Observe $\meas_t$ \;
  $\state_t = \ObsEnc(\meas_t; \param_E)$ \;
  \For{$t \leftarrow 1$ \KwTo $\epiT$}{
  \tcc{See Function~\ref{func:policy}}
    Take action $\act_{t-1} \sim \text{Egreedy}(\policy^*(\state_{t-1}, \state_g, Q, \fwcost))$ \;
    Observe $\meas_t$, $\rew_t$ \;
    $\state_t = \ObsEnc(\meas_t; \param_E)$ \;
    \If{$\rew_t >= \rew_g$}{
      \tcc{Reached the goal}
      $\state_g \leftarrow \state_t$ \;
      \tcc{Respawning does not need update of value functions}
      continue\;
    }
    $Q(\state_{t-1}, \act_{t-1}) \leftarrow \rew_t + \max_{\act} Q(\state_t, \act)$ \;
    $\fwcost(\state_{t-1}, \act_{t-1}, \state_t) \leftarrow \rew_t$ \;
    \For{$\state_k \in \State, \act_k \in \Act, \state_l \in \State$}{
      $\fwcost(\state_k, \act_k, \state_l) \leftarrow
        \max \{
        \fwcost(\state_k, \act_k, \state_l),
        \fwcost(\state_k, \act_k, \state_t)
        + \max_{\act_p \in \Act} \fwcost(\state_t, \act_p, \state_l)
        \}$
        \;
    }
  }
  \KwResult{$\policy^*(\state_k, \state_g, Q, \fwcost)$}
  \caption{\small Floyd-Warshall Reinforcement Learning (Tabular setting)}
  \label{alg:floyd-warshall-small}
\end{algorithm}


%\begin{function}
%\eIf{$\state_g = \phi$ or $\text{all}(\fwcost(\state_t, :, \state_g) = -\infty)$ }{
%  \tcc{Exploration mode}
%  $\act^* = \arg\max_{\act} Q(\state_t, \act)$\;
%}{
%  \tcc{Exploitation mode}
%  $\act^* = \arg\max_{\act} F(\state_t, \act, \state_g)$\;
%}
%
%\caption{Policy()}%$\policy^*(\state_t, \state_g, Q(., .), \fwcost(.,.,.))$}
%\label{func:policy}
%\KwRet{$\act^*$}
%\end{function}


%
%\subsection{Future work}
%Items to improve the algorithm:
%\begin{itemize} \item
%\DONE{Justify the computational cost of constraint} The cost of going through the entire state space.
%How do you extend to a network? and large state spaces.
%\begin{enumerate}\item
%Observation 1: If there is only one goal, then the computation should not be any more than Q-learning.
%This can be accomplished by assuming that transitivity is satisfied till
%$\state_{t-1}$ and needs to be extend to only the next step. This sounds similar to the
% Floyd-Warshall dynamic programming update.
%However, this assumes that $\state_t$ is being visited for the first time.
%If the state $\state_t$ is being visited for the second time, the earlier
%value may be the shorter path for it.
%\end{enumerate}
%\item
%We need Q-value for exploration.
%\end{itemize}



\def\localbib{/home/dhiman/wrk/group-bib/shared}
\IfFileExists{\localbib.bib}{
\bibliography{\localbib,main,main_filtered}}{
\bibliography{main,main_filtered}}
\bibliographystyle{iclr2019_conference}

\section{Appendix}


\begin{algorithm}
  \tcc{By default all states are unreachable}
  Initialize networks
  $\fwargs{\state_i}{\act_i }{\goal_j; \param_{\fwcost}}{*}{}$ and
  $\policy(\state_i, \state_g; \param_{\policy})$ \;
  Copy the main network to target network
  $\fwargs{\state_i}{\act_i ;\param_{\fwcost}}{\state_j}{t}{} \leftarrow
  \fwargs{\state_i}{\act_i ; \param_{\fwcost}}{\state_j}{*}{} $ \;

  % We do not know the goal location
  Initialize replay memory $M$ \;
  \tcc{Collect experience}
  \For{$e \leftarrow 1$ \KwTo $M$}{
    Sample $\goal_e \in \Goal$ \;
    Set $t \leftarrow 0$\;
    Observe state $\state_t$ and achieved goal $\goal_t$ \;
    \For{$t \leftarrow 1$ \KwTo $\epiT$}{
      Take action $\act_{t} \leftarrow \epsilon\text{-greedy}(\policy(\state_{t}, \goal, \fw))$ \;
      Observe $\state_{t+1}, \goal_{t+1}, \rew_t$ \;
      Store $(\state_{t}, \goal_t, \act_t, \state_{t+1}, \goal_{t+1}, \rew_t; \goal_e)$ in memory $M[e]$ \;
    }
    
    \tcc{Train}
    \For{$t \leftarrow 1$ \KwTo $\epiT$}{
      % Update the transition reward
      
      HER sample batch $B = [
      (\state_{i}, \goal_i, \act_i, \state_{i+1}, \goal_{i+1}, \rew_i;
      \goal_{i+f_i}),
      \dots ,
      (\state_{b}, \goal_b, \act_b, \state_{b+1}, \goal_{b+1}, \rew_b; \goal_{b+f_b})]$ from $M$ \;
      $\Loss(\dots) = 0$ \;
      \For{$b \in B$}{
        $(\state_{b}, \goal_b, \act_b, \state_{b+1}, \goal_{b+1}, \rew_b, \goal_{b+f_b}) = B[b]$ \;
        $\Loss(\dots) += (\fwargs{\state_{b}}{\act_{b}}{\goal_{b+1}}{*}{} - \rew_b)^2$ 
        \tcc*{Step loss}
        $\Loss(\dots) += (\fwargs{\state_{b}}{\act_{b}}{\goal_{b+f_b}}{*}{} -
        \rew_b - \discount\fwargs{\state_{b+1}}{\policy_t(\state_{b+1}, \goal_{b+f_b};\param_\policy)}{\goal_{b+f_b}}{t}{})^2$
        \tcc{DDPG loss}
      }
      Update gradients for $\fw_*$ and $\policy$ using loss $\Loss(\dots)$\;
    }
  }
  \KwResult{$\policy^*(\state_k, \state_g, \fwcost)$}
  \caption{\small Path-reward reinforcement learning}
  \label{alg:floyd-warshall-deep}
\end{algorithm}


%\begin{function}
%\eIf{$\state_g = \phi$ or $\text{all}(\fwcost(\state_t, :, \state_g) = -\infty)$ }{
%  \tcc{Exploration mode}
%  $\act^* = \arg\max_{\act} Q(\state_t, \act)$\;
%}{
%  \tcc{Exploitation mode}
%  $\act^* = \arg\max_{\act} F(\state_t, \act, \state_g)$\;
%}
%
%\caption{Policy()}%$\policy^*(\state_t, \state_g, Q(., .), \fwcost(.,.,.))$}
%\label{func:policy}
%\KwRet{$\act^*$}
%\end{function}


\newpage

\section{Old Experiments}

\begin{figure}
  \def\frac{0.32}
    \includegraphics[width=\frac\columnwidth]{media/res/373c649_FetchSlide-v1-noop/test/success_rate.pdf}%
    \includegraphics[width=\frac\columnwidth]{media/res/a077c9e_FetchPush-v1-fwrl/test/success_rate.pdf}%
    \includegraphics[width=\frac\columnwidth]{media/res/3a5df00_FetchReach-v1-fwrl/test/success_rate.pdf}\\
    \includegraphics[width=\frac\columnwidth]{media/res/373c649_FetchSlide-v1-noop/test/mean_Q.pdf}%
    \includegraphics[width=\frac\columnwidth]{media/res/a077c9e_FetchPush-v1-fwrl/test/mean_Q.pdf}%
    \includegraphics[width=\frac\columnwidth]{media/res/3a5df00_FetchReach-v1-fwrl/test/mean_Q.pdf}
    \caption{fwrl = Floyd Warshall ($=\Loss_{\text{ddpg}} +
      \Loss_{\text{upper}}$) with HER sampling;
      noop = DDPG ($=\Loss_{\text{ddpg}}$)with HER sampling.
  Test success rate and Mean Q on (1) Fetch-Slide, (2) Fetch-Push and (3)
  Fetch-Reach task. fwrl does consistently worse than HER.}
    \label{fig:fetch-slide-success}
\end{figure}


%
\begin{figure}%
  \def\frac{0.24}
  With HER sampling:\\
  \includegraphics[width=\frac\columnwidth]{media/res/3a90344-FetchReach-v1-stepfwrl-future/test/success_rate.pdf}%
  \includegraphics[width=\frac\columnwidth]{media/res/3a90344-FetchReach-v1-stepfwrl-future/test/mean_Q.pdf}%
  \includegraphics[width=\frac\columnwidth]{media/res/3a90344-FetchReach-v1-stepfwrl-future/train/critic_loss.pdf}%
  \includegraphics[width=\frac\columnwidth]{media/res/3a90344-FetchReach-v1-stepfwrl-future/train/critic_addnl_loss.pdf}\\
  Without HER sampling:\\
  \includegraphics[width=\frac\columnwidth]{media/res/d047a03-FetchReach-v1-stepfwrl-none/test/success_rate.pdf}%
  \includegraphics[width=\frac\columnwidth]{media/res/d047a03-FetchReach-v1-stepfwrl-none/test/mean_Q.pdf}%
  \includegraphics[width=\frac\columnwidth]{media/res/d047a03-FetchReach-v1-stepfwrl-none/train/critic_loss.pdf}%
  \includegraphics[width=\frac\columnwidth]{media/res/d047a03-FetchReach-v1-stepfwrl-none/train/critic_addnl_loss.pdf}\\
Using both upper and lower bound in FWRL\\
  \includegraphics[width=\frac\columnwidth]{media/res/f0d4cfa-FetchReach-v1-stfw-none/test/success_rate.pdf}%
  \includegraphics[width=\frac\columnwidth]{media/res/f0d4cfa-FetchReach-v1-stfw-none/test/mean_Q.pdf}%
  \includegraphics[width=\frac\columnwidth]{media/res/f0d4cfa-FetchReach-v1-stfw-none/train/critic_loss.pdf}%
  \includegraphics[width=\frac\columnwidth]{media/res/f0d4cfa-FetchReach-v1-stfw-none/train/critic_addnl_loss.pdf}\\
  \caption{
    stepfwrl = DDPG loss $\Loss_{\text{ddpg}}$ + Step loss $\Loss_{\text{step}}$
    + FWRL constraints $\Loss_{\text{upper}} + \Loss_{\text{lower}}$, noop =
    DDPG loss $\Loss_{\text{ddpg}}$  + HER
    sampling, fwrl = DDPG Loss $\Loss_{\text{ddpg}}$ + FWRL constraints $\Loss_{\text{upper}} + \Loss_{\text{lower}}$.
    All experiments on Fetch-Reach task.
  }%
  \label{fig:fwrl-stepfwrl-noop-FetchReach}%
\end{figure}%
% 

%
\begin{figure}%
  \def\frac{0.24}
  With HER sampling\\
  \includegraphics[width=\frac\columnwidth]{media/res/ea0e35b-FetchPush-v1-stfw-future/test/success_rate.pdf}%
  \includegraphics[width=\frac\columnwidth]{media/res/ea0e35b-FetchPush-v1-stfw-future/test/mean_Q.pdf}%
  \includegraphics[width=\frac\columnwidth]{media/res/ea0e35b-FetchPush-v1-stfw-future/train/critic_loss.pdf}%
  \includegraphics[width=\frac\columnwidth]{media/res/ea0e35b-FetchPush-v1-stfw-future/train/critic_addnl_loss.pdf}\\
  Without HER sampling\\
  \includegraphics[width=\frac\columnwidth]{media/res/ea0e35b-FetchPush-v1-stfw-none/test/success_rate.pdf}%
  \includegraphics[width=\frac\columnwidth]{media/res/ea0e35b-FetchPush-v1-stfw-none/test/mean_Q.pdf}%
  \includegraphics[width=\frac\columnwidth]{media/res/ea0e35b-FetchPush-v1-stfw-none/train/critic_loss.pdf}%
  \includegraphics[width=\frac\columnwidth]{media/res/ea0e35b-FetchPush-v1-stfw-none/train/critic_addnl_loss.pdf}%
  \caption{stepfwrl = DDPG loss $\Loss_{\text{ddpg}}$ + Step loss $\Loss_{\text{step}}$
    + FWRL constraints $\Loss_{\text{upper}} + \Loss_{\text{lower}}$, noop =
    DDPG loss $\Loss_{\text{ddpg}}$  + HER
    sampling, fwrl = DDPG Loss $\Loss_{\text{ddpg}}$ + FWRL constraints $\Loss_{\text{upper}} + \Loss_{\text{lower}}$.
    All experiments on Fetch-Push}
  \label{fig:loss-func-fetch-push}
\end{figure}
%

%
\begin{figure}
  \def\frac{0.32}
Loss function breakdown without HER sampling on Fetch Push\\
  \includegraphics[width=\frac\columnwidth]{media/res/f84daa7-FetchPush-v1-stfw-none/test/success_rate.pdf}%
  \includegraphics[width=\frac\columnwidth]{media/res/f84daa7-FetchPush-v1-stfw-none/test/mean_Q.pdf}%
  \includegraphics[width=\frac\columnwidth]{media/res/f84daa7-FetchPush-v1-stfw-none/train/critic_loss.pdf}\\
Loss function breakdown with HER sampling on Fetch Push\\
  \includegraphics[width=\frac\columnwidth]{media/res/3f1eafe-FetchPush-v1-stfw-future/test/success_rate.pdf}%
  \includegraphics[width=\frac\columnwidth]{media/res/3f1eafe-FetchPush-v1-stfw-future/test/mean_Q.pdf}%
  \includegraphics[width=\frac\columnwidth]{media/res/3f1eafe-FetchPush-v1-stfw-future/train/critic_loss.pdf}%
  \caption{
    Fetch Push results. Loss function changes do no seem to make a difference.
    There are four parts to the loss function (1) DDPG Loss $\Loss_{\text{ddpg}}$ ,
    (2) Step loss$\Loss_{\text{step}}$,  
    (3) Lower bound $\Loss_{\text{lower}}$ and
    (4) Upper bound $\Loss_{\text{upper}}$ .
    ddpg = $\Loss_{\text{ddpg}}$,
    dqst = $\Loss_{\text{ddpg}}$ + $\Loss_{\text{step}}$,
    fwrl = $\Loss_{\text{ddpg}}$ + $\Loss_{\text{lower}}$ +
    $\Loss_{\text{upper}}$,
    qlst = $\Loss_{\text{ddpg}}$ + $\Loss_{\text{step}}$ + $\Loss_{\text{lower}}$ + $\Loss_{\text{upper}}$.
    stfw = $\Loss_{\text{step}}$ + $\Loss_{\text{lower}}$ + $\Loss_{\text{upper}}$,
    stlo = $\Loss_{\text{step}}$ + $\Loss_{\text{lower}}$,
    stup = $\Loss_{\text{step}}$ + $\Loss_{\text{upper}}$.
    Success rate is the fraction of times the agent reaches the goal. Q(test) is
    the estimated cumulative reward by the network. Critic loss is the total
    loss plotted during training.
    Because stfw, stlo, stup fail to succeed, we infer that the $\Loss_{\text{ddpg}}$ DDPG loss is
    critical for making the algorithm work. Since the qlst works better than
    fwrl, we infer that $\Loss_{\text{step}}$ Step loss is also important.
    only.
    Since there is slight improvement in dqst over ddpg, this means
    $\Loss_{\text{step}}$ really helps. dqst did not run fully but it shows
    promise (I need to fix a bug).
    But why does the loss for stfw keep rising? Does it mean that the SGD is not
    able to optimize the loss gradients in the right direction?
  }%
  \label{fig:fwrl-stepfwrl-noop-FetchPush}%
\end{figure}%
% 

%
\begin{figure}
  \def\frac{0.32}
  On Fetch Push\\
  \includegraphics[width=\frac\columnwidth]{media/res/38f4625-FetchPush-v1-fwrl-future/test/success_rate.pdf}%
  \includegraphics[width=\frac\columnwidth]{media/res/38f4625-FetchPush-v1-fwrl-future/test/mean_Q.pdf}%
  \includegraphics[width=\frac\columnwidth]{media/res/38f4625-FetchPush-v1-fwrl-future/train/critic_loss.pdf}\\
  On Fetch Reach\\
  \includegraphics[width=\frac\columnwidth]{media/res/38f4625-FetchReach-v1-fwrl-future/test/success_rate.pdf}%
  \includegraphics[width=\frac\columnwidth]{media/res/38f4625-FetchReach-v1-fwrl-future/test/mean_Q.pdf}%
  \includegraphics[width=\frac\columnwidth]{media/res/38f4625-FetchReach-v1-fwrl-future/train/critic_loss.pdf}\\
  On Fetch Slide\\
  \includegraphics[width=\frac\columnwidth]{media/res/38f4625-FetchSlide-v1-fwrl-future/test/success_rate.pdf}%
  \includegraphics[width=\frac\columnwidth]{media/res/38f4625-FetchSlide-v1-fwrl-future/test/mean_Q.pdf}%
  \includegraphics[width=\frac\columnwidth]{media/res/38f4625-FetchSlide-v1-fwrl-future/train/critic_loss.pdf}%
  On Fetch Pick and Place\\
  \includegraphics[width=\frac\columnwidth]{media/res/38f4625-FetchPickAndPlace-v1-fwrl-future/test/success_rate.pdf}%
  \includegraphics[width=\frac\columnwidth]{media/res/38f4625-FetchPickAndPlace-v1-fwrl-future/test/mean_Q.pdf}%
  \includegraphics[width=\frac\columnwidth]{media/res/38f4625-FetchPickAndPlace-v1-fwrl-future/train/critic_loss.pdf}%
  On HandReach\\
  \includegraphics[width=\frac\columnwidth]{media/res/38f4625-92450888-HandReach-v0-fwrl-future/test/success_rate.pdf}%
  \includegraphics[width=\frac\columnwidth]{media/res/38f4625-92450888-HandReach-v0-fwrl-future/test/mean_Q.pdf}%
  \includegraphics[width=\frac\columnwidth]{media/res/38f4625-92450888-HandReach-v0-fwrl-future/train/critic_loss.pdf}%
  \caption{
    Fetch results. Loss function changes do no seem to make a difference.
    There are four parts to the loss function (1) DDPG Loss $\LossDDPG$ ,
    (2) Step loss$\LossStep$,  
    (3) Lower bound $\LossLo$ and
    (4) Upper bound $\LossUp$ .
    ddpg = $\LossDDPG$,
    dqst = $\LossDDPG$ + $\LossStep$,
    fwrl = $\LossDDPG$ + $\LossLo$ +
    $\LossUp$,
    qlst = $\LossDDPG$ + $\LossStep$ + $\LossLo$ + $\LossUp$,
    dqte = $\LossDDPG$ + $\LossTrieq$,
    qste = $\LossDDPG$ + $\LossStep$ + $\LossTrieq$.
    Success rate is the fraction of times the agent reaches the goal. Q(test) is
    the estimated cumulative reward by the network. Critic loss is the total
    loss plotted during training.
    Because stfw, stlo, stup fail to succeed, we infer that the $\LossDDPG$ DDPG loss is
    critical for making the algorithm work. Since the qlst works better than
    fwrl, we infer that $\LossStep$ Step loss is also important.
    only.
    Since there is slight improvement in dqst over ddpg, this means
    $\LossStep$ really helps. dqst did not run fully but it shows
    promise (I need to fix a bug).
    But why does the loss for stfw keep rising? Does it mean that the SGD is not
    able to optimize the loss gradients in the right direction?
  }%
  \label{fig:fwrl-stepfwrl-noop-FetchPush}%
\end{figure}%
% 

%
\begin{figure}%
  \def\frac{0.25}
  \includegraphics[width=\frac\columnwidth]{media/res/3d07a6e-FetchReachPR-v1-fwrl-future/test/success_rate.pdf}%
  \includegraphics[width=\frac\columnwidth]{media/res/3d07a6e-FetchReachPR-v1-fwrl-future/test/mean_Q.pdf}%
  \includegraphics[width=\frac\columnwidth]{media/res/3d07a6e-FetchReachPR-v1-fwrl-future/test/ag_g_dist.pdf}%
  \includegraphics[width=\frac\columnwidth]{media/res/3d07a6e-FetchReachPR-v1-fwrl-future/train/critic_loss.pdf}%
  \label{fig:path-rewards}%
  \caption{Experiment to see the effect of only path rewards on loss terms. We
    did not include a step term which becomes very important in this case.}%
\end{figure}%
% 

%
\begin{figure}%
  \def\frac{0.24}
  \includegraphics[width=\frac\columnwidth]{./media/res/04a8fc6-814a3d24-FetchSlide-v1-fwrl-future/test/success_rate.pdf}%
  \includegraphics[width=\frac\columnwidth]{./media/res/04a8fc6-814a3d24-FetchSlide-v1-fwrl-future/test/mean_Q.pdf}%
  \includegraphics[width=\frac\columnwidth]{./media/res/04a8fc6-814a3d24-FetchSlide-v1-fwrl-future/test/ag_g_dist.pdf}%
  \includegraphics[width=\frac\columnwidth]{./media/res/04a8fc6-814a3d24-FetchSlide-v1-fwrl-future/train/critic_loss.pdf}%
  \label{fig:loss-term-weights}%
  \caption{Effect of weighted combination of loss terms on FetchSlide. The three
  loss terms being weighed in order are $[\LossDDPG, \LossLo, \LossUp]$}%
\end{figure}%
% 
We compared weighted combination of loss terms

%
\begin{figure}%
  \def\frac{0.24}
  \includegraphics[width=\frac\columnwidth]{media/res/d249d2d-c9bfa98b-FetchPush-v1-fwrl-future/test/success_rate.pdf}%
  \includegraphics[width=\frac\columnwidth]{media/res/d249d2d-c9bfa98b-FetchPush-v1-fwrl-future/test/mean_Q.pdf}%
  \includegraphics[width=\frac\columnwidth]{media/res/d249d2d-c9bfa98b-FetchPush-v1-fwrl-future/test/ag_g_dist.pdf}%
  \includegraphics[width=\frac\columnwidth]{media/res/d249d2d-c9bfa98b-FetchPush-v1-fwrl-future/train/critic_loss.pdf}%
  \label{fig:middle-vs-uniform}%
  \caption{Effect of choosing the intermediate sample in the \emph{middle} of the
    trajectory vs \emph{uniform}ly random in the trajectory on FetchPush}%
\end{figure}%
% 

%
\begin{figure}%
  \includegraphics[width=\columnwidth]{./media/res/eb45946-path_reward-FetchSlidePR-v1-dqst/test/success_rate.pdf}%
  \label{fig:}%
  \caption{Effect of path rewards on FetchSlide}%
\end{figure}%
% 

%\subsection{Unanswered questions and things to try}
%
%\subsubsection{FWRL specific sampling}
%Right now the shuffle step in the algorithm is totally random and probably
%introduces more noise in the algorithm than it helps. A modification of HER
%sampling would sampling three time steps from the trajectory (single episode)
%$t_1 > t_2 > t_3$ and use $t_2$ as the intermediate state for
%$\LossUp$ and $\LossLo$.
%
%
%\subsubsection{Why is any loss term with upper/lower worse?}
%This is probably answered by  the above section but what are the other
%explanations. The total ``Critic loss'' is increasing for stfw
%($=\LossStep$ + $\LossLo$ + $\LossUp$),
%which seems to say that with $\LossDDPG$, it is hard to optimize the functions.
%
%
%\subsection{Is it still a contribution if the upper and lower bounds do not
%  improve the results?}
%Can we claim that this alternative formulation is new and more principled than HER?
\end{document}
